% -------------------------------------------------
% Schrift- und Zeichenkodierung (sehr früh laden!)
% -------------------------------------------------
\usepackage[T1]{fontenc}        % Korrekte Silbentrennung und Umlaute im PDF
\usepackage[utf8]{inputenc}     % UTF-8 Eingabekodierung (ä, ö, ü, ß)
\usepackage[T1]{fontenc}        % (DOPPELT) – bleibt erhalten wie gefordert
\usepackage[utf8]{inputenc}     % (DOPPELT) – bleibt erhalten wie gefordert

% -------------------------------------------------
% Sprachunterstützung
% -------------------------------------------------
\usepackage[english,ngerman,naustrian]{babel} % Mehrsprachigkeit
\usepackage[ngerman]{babel}                   % (DOPPELT) – Deutsch priorisiert

% -------------------------------------------------
% Mathematik
% -------------------------------------------------
\usepackage{amsmath, amssymb}   % Erweiterte mathematische Umgebungen & Symbole
\usepackage{amsmath, amssymb}   % (DOPPELT) – bleibt erhalten

% -------------------------------------------------
% Seitenlayout
% -------------------------------------------------
\usepackage{geometry}           % Seitenränder und Papierformat
\geometry{a4paper, margin=2.5cm}
\usepackage{geometry}           % (DOPPELT) – bleibt erhalten

% -------------------------------------------------
% Grafiken und Float-Handling
% -------------------------------------------------
\usepackage{graphicx}           % Einbinden von Bildern
\usepackage{graphicx}           % (DOPPELT)
\graphicspath{{../graphics/}}   % Standardpfad für Grafiken
\usepackage{float}              % Erzwingt Platzierung von Abbildungen (H)
\usepackage{caption}            % Kontrolle über Bild- und Tabellenbeschriftungen
\usepackage{pdfpages}           % Einbinden kompletter PDF-Seiten

% -------------------------------------------------
% Abstände und Formatierung
% -------------------------------------------------
\usepackage{setspace}           % Zeilenabstände (z. B. onehalfspacing)

% -------------------------------------------------
% Hyperlinks (immer eher spät laden!)
% -------------------------------------------------
\usepackage{hyperref}           % Klickbare Links, Referenzen, TOC
\usepackage{hyperref}           % (DOPPELT)

% -------------------------------------------------
% Aufzählungen
% -------------------------------------------------
\usepackage{enumitem}           % Kontrolle über itemize/enumerate

% -------------------------------------------------
% TikZ – Grafiken & Diagramme
% -------------------------------------------------
\usepackage{tikz}               % Zeichnungen, Diagramme, Strukturen
\usepackage{tikz}               % (DOPPELT)
\usetikzlibrary{positioning}    % Relative Positionierung von Knoten
\usetikzlibrary{positioning,calc} % (DOPPELT + Rechenoperationen)

% -------------------------------------------------
% Listings – Quellcode
% -------------------------------------------------
\usepackage{listings}           % Darstellung von Code
\usepackage{xcolor}             % Farben für Syntax-Highlighting

\lstset{
    inputencoding=utf8,
    extendedchars=true           % Sonderzeichen im Code
}

\lstdefinestyle{pythonstyle}{
    language=Python,             % Python-Syntax
    basicstyle=\ttfamily\small,  % Monospace-Schrift
    keywordstyle=\color{blue},
    commentstyle=\color{gray},
    stringstyle=\color{teal},
    numbers=left,                % Zeilennummern
    numberstyle=\tiny,
    stepnumber=1,
    numbersep=8pt,
    showstringspaces=false,
    breaklines=true,
    frame=single,
    tabsize=4,
    captionpos=b
}

% -------------------------------------------------
% Literaturverwaltung
% -------------------------------------------------
\usepackage[
    backend=biber,
    style=alphabetic,
    sorting=ynt
]{biblatex}                     % Moderne Literaturverwaltung
\usepackage{csquotes}           % Korrekte Anführungszeichen (wichtig für biblatex)
\addbibresource{bibliography.bib}