
\documentclass[a4paper,12pt]{article}
% -------------------------------------------------
% Schrift- und Zeichenkodierung (sehr früh laden!)
% -------------------------------------------------
\usepackage[T1]{fontenc}        % Korrekte Silbentrennung und Umlaute im PDF
\usepackage[utf8]{inputenc}     % UTF-8 Eingabekodierung (ä, ö, ü, ß)
\usepackage[T1]{fontenc}        % (DOPPELT) – bleibt erhalten wie gefordert
\usepackage[utf8]{inputenc}     % (DOPPELT) – bleibt erhalten wie gefordert

% -------------------------------------------------
% Sprachunterstützung
% -------------------------------------------------
\usepackage[english,ngerman,naustrian]{babel} % Mehrsprachigkeit
\usepackage[ngerman]{babel}                   % (DOPPELT) – Deutsch priorisiert

% -------------------------------------------------
% Mathematik
% -------------------------------------------------
\usepackage{amsmath, amssymb}   % Erweiterte mathematische Umgebungen & Symbole
\usepackage{amsmath, amssymb}   % (DOPPELT) – bleibt erhalten

% -------------------------------------------------
% Seitenlayout
% -------------------------------------------------
\usepackage{geometry}           % Seitenränder und Papierformat
\geometry{a4paper, margin=2.5cm}
\usepackage{geometry}           % (DOPPELT) – bleibt erhalten

% -------------------------------------------------
% Grafiken und Float-Handling
% -------------------------------------------------
\usepackage{graphicx}           % Einbinden von Bildern
\usepackage{graphicx}           % (DOPPELT)
\graphicspath{{../graphics/}}   % Standardpfad für Grafiken
\usepackage{float}              % Erzwingt Platzierung von Abbildungen (H)
\usepackage{caption}            % Kontrolle über Bild- und Tabellenbeschriftungen
\usepackage{pdfpages}           % Einbinden kompletter PDF-Seiten

% -------------------------------------------------
% Abstände und Formatierung
% -------------------------------------------------
\usepackage{setspace}           % Zeilenabstände (z. B. onehalfspacing)

% -------------------------------------------------
% Hyperlinks (immer eher spät laden!)
% -------------------------------------------------
\usepackage{hyperref}           % Klickbare Links, Referenzen, TOC
\usepackage{hyperref}           % (DOPPELT)

% -------------------------------------------------
% Aufzählungen
% -------------------------------------------------
\usepackage{enumitem}           % Kontrolle über itemize/enumerate

% -------------------------------------------------
% TikZ – Grafiken & Diagramme
% -------------------------------------------------
\usepackage{tikz}               % Zeichnungen, Diagramme, Strukturen
\usepackage{tikz}               % (DOPPELT)
\usetikzlibrary{positioning}    % Relative Positionierung von Knoten
\usetikzlibrary{positioning,calc} % (DOPPELT + Rechenoperationen)

% -------------------------------------------------
% Listings – Quellcode
% -------------------------------------------------
\usepackage{listings}           % Darstellung von Code
\usepackage{xcolor}             % Farben für Syntax-Highlighting

\lstset{
    inputencoding=utf8,
    extendedchars=true           % Sonderzeichen im Code
}

\lstdefinestyle{pythonstyle}{
    language=Python,             % Python-Syntax
    basicstyle=\ttfamily\small,  % Monospace-Schrift
    keywordstyle=\color{blue},
    commentstyle=\color{gray},
    stringstyle=\color{teal},
    numbers=left,                % Zeilennummern
    numberstyle=\tiny,
    stepnumber=1,
    numbersep=8pt,
    showstringspaces=false,
    breaklines=true,
    frame=single,
    tabsize=4,
    captionpos=b
}

% -------------------------------------------------
% Literaturverwaltung
% -------------------------------------------------
\usepackage[
    backend=biber,
    style=alphabetic,
    sorting=ynt
]{biblatex}                     % Moderne Literaturverwaltung
\usepackage{csquotes}           % Korrekte Anführungszeichen (wichtig für biblatex)
\addbibresource{bibliography.bib}
\addbibresource{bibliography.bib}

% ------------------ Dokument ------------------
\begin{document}
% ------------------ Titelseite ------------------
\begin{titlepage}
    \centering

    % ------------------ Titel ------------------
    \vspace*{2cm}
    {\LARGE \bfseries Steuerungstechnik \par}
    \vspace{0.5cm}
    {\Large STT 5  \par}

    \vspace{0.8cm}
    \rule{0.8\textwidth}{0.6pt}

    % ------------------ Untertitel ------------------
    \vspace{0.8cm}
    {\large Lernskript \par}

    \vspace{1.2cm}
    \rule{0.6\textwidth}{0.4pt}
    \vspace{1cm}

\begin{figure}[H]
    \centering
    \includegraphics[width=0.8\textwidth]{/Users/danielweindl/_source/LaTex/STT-Lernskript/Bilder/titelbild2.png}
\end{figure}

    % ------------------ Meta-Informationen ------------------
    \vfill
    \begin{tabular}{rl}
        \textbf{Name:} & Daniel Weindl \\
        \textbf{Studiengang:} & Automatisierungstechnik \\
        \textbf{Datum:} & \today \\
    \end{tabular}

    \vspace{1.5cm}

\end{titlepage}
% ------------------ Inhaltsverzeichnis ------------------
\tableofcontents
\newpage
% ------------------ Kapitel ------------------
%\section{Einleitung}

\subsection{Einordnung der Steuerungstechnik}

Die Steuerungstechnik ist eine Teildisziplin der Automatisierungstechnik und beschäftigt sich
mit dem gezielten Beeinflussen technischer Prozesse durch vorgegebene Vorschriften,
Algorithmen oder Gesetzmäßigkeiten.
Sie umfasst den \textbf{Entwurf}, die \textbf{Realisierung}, das \textbf{Testen},
die \textbf{Inbetriebnahme} sowie die \textbf{Wartung} von Steuerungssystemen.

Ein technischer Prozess ist ein Vorgang zur \textbf{Umformung}, \textbf{Speicherung}
oder zum \textbf{Transport} von Material, Energie oder Information.
Der Zustand eines Prozesses wird durch sogenannte \emph{Zustandsgrößen} beschrieben,
die technisch erfassbar und beeinflussbar sind.

Die Automatisierungstechnik gliedert sich unter anderem in:
\begin{itemize}
    \item Sensorik (Informationsgewinnung)
    \item Informationsverarbeitung (z.\,B. SPS, Industrie-PC)
    \item Aktorik (Informationsumsetzung)
    \item Mensch-Maschine-Schnittstelle (Visualisierung, Bedienung)
\end{itemize}

\subsection{Steuern und Regeln}

\paragraph{Steuern}
Unter Steuern versteht man einen Vorgang, bei dem Eingangsgrößen
Ausgangsgrößen eines Systems beeinflussen, \textbf{ohne dass die Ausgangsgrößen
rückgeführt und fortlaufend verglichen werden}.
Kennzeichnend ist der \textbf{offene Wirkungsablauf}.

Die Steuerungstechnik ist die gezielte Anwendung von Steuerungsmechanismen
(Vorschriften, Algorithmen, Gesetzmäßigkeiten) unter Zuhilfenahme der verfügbaren
technischen Mittel zur zielgerichteten Erfüllung von Aufgaben in Prozessen.

\paragraph{Regeln}
Beim Regeln wird eine Regelgröße kontinuierlich erfasst,
mit einer Führungsgröße verglichen und in Richtung dieser Führungsgröße beeinflusst.
Kennzeichnend ist der \textbf{geschlossene Wirkungsablauf (Rückkopplung)}.

\medskip
\textbf{Merksatz:}  
\emph{Steuern arbeitet ohne Rückführung, Regeln immer mit Rückführung.}

\subsection{Offene und geschlossene Steuerungen}

\begin{itemize}
    \item \textbf{Offene Steuerung}:  
    Keine Rückmeldung aus dem Prozess.  
    Beispiel: Zeitgesteuerte Ampel.
    
    \item \textbf{Geschlossene Steuerung}:  
    Rückmeldung beeinflusst den Steuerungsablauf, jedoch nicht kontinuierlich.  
    Beispiel: Abschaltung über Endschalter.
    
    \item \textbf{Regelung}:  
    Kontinuierliche Rückführung und Korrektur.  
    Beispiel: Drehzahlregelung eines Motors.
\end{itemize}

 
\begin{figure}[H]
    \centering
    \includegraphics[page=12,width=0.8\textwidth]{/Users/danielweindl/_source/Repositorys/STT-Lernskript/Data/sttvo-k01-Einführung-4v3-Folien_011025.pdf}    
 \end{figure}


\subsection{Grundstruktur eines Steuerungssystems}

Ein Steuerungssystem besteht grundsätzlich aus:
\begin{itemize}
    \item Führungsgrößen (z.\,B. Start, Stop, Sollwerte)
    \item Steuereinrichtung (z.\,B. SPS)
    \item Steuerstrecke (technischer Prozess)
    \item Stellgrößen (Ausgangsgrößen der Steuerung)
    \item Meldegrößen (Rückmeldungen, Anzeigen)
    \item Störgrößen (unerwünschte Einflüsse)
\end{itemize}

\subsection{Klassifikation von Steuerungen}

\subsubsection{Nach Informationsdarstellung}

\begin{itemize}
    \item \textbf{Analoge Steuerungen}:  
    Stetige Signalverarbeitung (z.\,B. Spannungen, Ströme)
    
    \item \textbf{Digitale Steuerungen}:  
    Verarbeitung von Zahlenwerten und Codes
    
    \item \textbf{Binäre Steuerungen}:  
    Verarbeitung logischer Zustände (TRUE/FALSE)
\end{itemize}

\begin{figure}[H]
    \centering
    \includegraphics[width=0.6\textwidth]{/Users/danielweindl/_source/Repositorys/STT-Lernskript/Bilder/BinäreSignalverarbeitung.png}
\end{figure}

\subsubsection{Nach Art der Signalverarbeitung}
\begin{figure}[H]
    \centering
    \includegraphics[width=0.8\textwidth]{/Users/danielweindl/_source/Repositorys/STT-Lernskript/Bilder/ArtenSignalverarbeitung.png}
\end{figure}

\begin{itemize}
    \item \textbf{Synchrone Steuerungen}:  
    Abtastung und Verarbeitung im festen Zeitraster
    
    \item \textbf{Asynchrone Steuerungen}:  
    Ereignisgesteuerte Verarbeitung
    
\begin{figure}[H]
    \centering
    \includegraphics[width=1\textwidth]{/Users/danielweindl/_source/Repositorys/STT-Lernskript/Bilder/SynchronAsynchron.png}
\end{figure}

    \item \textbf{Verknüpfungssteuerungen}:  
    Ausgang hängt nur vom aktuellen Eingang ab
    
    \item \textbf{Ablaufsteuerungen}:  
    Ausgang hängt von Eingang \emph{und} Zustand ab
\end{itemize}

\subsection{Kategorien von Steuerungen}

Die Realisierung einer Steuerung beschreibt,
\textbf{mit welchen technischen Mitteln}
die Steuerungsfunktion umgesetzt wird.
Man unterscheidet mehrere grundlegende Realisierungsarten,
die historisch und technisch unterschiedliche Eigenschaften besitzen.

\begin{figure}[H]
    \centering
    \includegraphics[width=0.9\textwidth]{/Users/danielweindl/_source/Repositorys/STT-Lernskript/Bilder/UnterscheidungEnergieübertragung.png}
\end{figure}


\subsubsection{Mechanische Steuerungen}

Mechanische Steuerungen realisieren Steuerungsfunktionen
ausschließlich durch mechanische Elemente.
Die Steuerinformation wird durch Kräfte, Wege oder Bewegungen erzeugt.

\paragraph{Typische Elemente}
\begin{itemize}
    \item Anschläge
    \item Nocken
    \item Kurvenscheiben
    \item Gestänge
\end{itemize}

\paragraph{Eigenschaften}
\begin{itemize}
    \item Kein elektrischer Energiebedarf
    \item Ablauf direkt an Bewegung gekoppelt
    \item Sehr robust
    \item Kaum flexibel
\end{itemize}

\paragraph{Beispiele}
\begin{itemize}
    \item Waschmaschinen mit mechanischem Programmwerk
    \item Nockensteuerungen in Werkzeugmaschinen
    \item Drehorgeln, Spieluhren
\end{itemize}

\paragraph{Einordnung}
Mechanische Steuerungen sind historisch bedeutsam,
werden heute jedoch kaum noch neu eingesetzt,
da Änderungen nur mechanisch möglich sind.

\subsubsection{Pneumatische Steuerungen}

Pneumatische Steuerungen nutzen Druckluft
zur Signalübertragung und zur Energieversorgung der Aktoren.

\paragraph{Typische Elemente}
\begin{itemize}
    \item Pneumatikzylinder
    \item Wegeventile
    \item Drosseln
    \item Pneumatische Logikelemente
\end{itemize}

\paragraph{Eigenschaften}
\begin{itemize}
    \item Schnelle Schaltzeiten
    \item Relativ einfache Logik möglich
    \item Begrenzte Genauigkeit
    \item Kompressibilität der Luft
\end{itemize}

\paragraph{Beispiele}
\begin{itemize}
    \item Handhabungssysteme
    \item Einfache Montageautomaten
\end{itemize}

\paragraph{Einordnung}
Pneumatische Steuerungen werden heute meist
in Kombination mit SPS eingesetzt
(SPS steuert Ventile, Pneumatik führt aus).

 

\subsubsection{Hydraulische Steuerungen}

Hydraulische Steuerungen verwenden
nicht oder kaum kompressible Flüssigkeiten
zur Kraft- und Signalübertragung.

\paragraph{Typische Elemente}
\begin{itemize}
    \item Hydraulikzylinder
    \item Steuerventile
    \item Pumpen
    \item Druckspeicher
\end{itemize}

\paragraph{Eigenschaften}
\begin{itemize}
    \item Sehr hohe Stellkräfte möglich
    \item Gute Regelbarkeit
    \item Hoher technischer Aufwand
    \item Wartungsintensiv
\end{itemize}

\paragraph{Beispiele}
\begin{itemize}
    \item Pressen
    \item Spritzgussmaschinen
\end{itemize}

\paragraph{Einordnung}
Hydraulische Steuerungen sind heute
meist Teil der Steuerstrecke,
während die Steuerlogik elektrisch oder per SPS realisiert wird.

 

\subsubsection{Elektrische Steuerungen}

Elektrische Steuerungen realisieren die Steuerungsfunktion
durch elektrische Schaltgeräte und Leitungsverknüpfungen.

\paragraph{Typische Elemente}
\begin{itemize}
    \item Relais
    \item Schütze
    \item Schalter
    \item Zeitrelais
\end{itemize}

\paragraph{Eigenschaften}
\begin{itemize}
    \item Binäre Logik gut realisierbar
    \item Hohe Übersichtlichkeit im Stromlaufplan
    \item Änderungen nur durch Umverdrahtung
    \item Mechanischer Verschleiß
\end{itemize}

\paragraph{Beispiele}
\begin{itemize}
    \item Motorsteuerungen
    \item Aufzugssteuerungen (klassisch)
\end{itemize}

\paragraph{Einordnung}
Elektrische Steuerungen sind die direkte
\textbf{Vorgänger-Technologie der SPS}
und Grundlage für KOP-Darstellungen.

 

\subsubsection{Elektronische Steuerungen}

Elektronische Steuerungen verwenden
kontaktlose elektronische Bauelemente
zur Realisierung der Steuerungslogik.

\paragraph{Typische Elemente}
\begin{itemize}
    \item Transistoren
    \item Logikgatter
    \item Mikrocontroller
    \item FPGA
\end{itemize}

\paragraph{Eigenschaften}
\begin{itemize}
    \item Hohe Schaltgeschwindigkeit
    \item Kein mechanischer Verschleiß
    \item Geringe Leistungsfähigkeit für Aktoren
    \item Hoher Entwicklungsaufwand
\end{itemize}

\paragraph{Einordnung}
Elektronische Steuerungen werden heute
vor allem in eingebetteten Systemen eingesetzt
und bilden die Hardwarebasis moderner SPS.

\subsection{Speicherprogrammierbare Steuerung (SPS)}

\begin{figure}[H]
    \centering
    \includegraphics[width=0.6\textwidth]{/Users/danielweindl/_source/Repositorys/STT-Lernskript/Bilder/Automatisierungspyramide.png}
\end{figure}
\subsubsection{Begriff}

Eine speicherprogrammierbare Steuerung (SPS) ist eine
digital arbeitende Steuerungseinrichtung zur Steuerung von Maschinen
und Anlagen.
Die Steuerungsfunktion wird nicht durch Verdrahtung,
sondern durch ein im Speicher abgelegtes Programm realisiert.


Unter einer SPS versteht man einen speziellen Computer mit mehreren (in der Regel) digitalen Ein‐/Ausgängen,
der logische Verknüpfungen (programmierbar) zwischen den Ein‐ und Ausgängen herstellt.
Im Prinzip können Steuerungsaufgaben auch von ganz normalen PCs erfolgen. Entsprechende Software (Soft‐SPS für
PCs) ist für diesen Zweck auch verfügbar. Vom Standpunkt der Programmierung aus ist es unerheblich, auf welchem
System die SPS‐Funktion umgesetzt wird. Dennoch werden Steuerungen überwiegend auf speziellen SPS‐Computern
verwirklicht. Folgende Gründe sprechen für den Einsatz spezieller SPS‐Geräte:
\subsubsection{Grundelemente einer SPS}

Eine SPS besteht aus folgenden grundlegenden Elementen:

\begin{itemize}
    \item \textbf{Zentraleinheit (CPU)}  
    Führt das Steuerungsprogramm aus, verwaltet Speicher,
    steuert den Zyklus und koordiniert die Kommunikation.

    \item \textbf{Netzteil}  
    Versorgt CPU und Baugruppen mit der erforderlichen Betriebsspannung
    (typisch 24\,V DC).

    \item \textbf{Eingangskarte}  
    Erfassen binäre oder analoge Signale aus der Feldebene
    (Sensoren).

    \item \textbf{Ausgangskarte}  
    Stellen binäre oder analoge Signale zur Ansteuerung von Aktoren bereit.

    \item \textbf{Kommunikationsbaugruppen}  
    Ermöglichen den Datenaustausch mit anderen SPSen,
    Feldgeräten oder Leitsystemen.

    \item \textbf{Rückwandbus / internes Bussystem}  
    Verbindet CPU und Baugruppen und ermöglicht den internen Datenaustausch.
\end{itemize}

\subsubsection{Baugruppen einer SPS}

\paragraph{Digitale Eingangskarte}
\begin{itemize}
    \item Erfassen binäre Zustände (0/1)
    \item Typische Signale: Schalter, Taster, Endschalter
    \item Galvanische Trennung zur Feldebene
\end{itemize}

\paragraph{Digitale Ausgangskarte}
\begin{itemize}
    \item Schalten binäre Lasten
    \item Typisch: Relais-, Transistor- oder Triac-Ausgänge
\end{itemize}

\paragraph{Analoge Eingangskarte}
\begin{itemize}
    \item Erfassen kontinuierliche Messwerte
    \item Typische Signale: 0–10\,V, 4–20\,mA
\end{itemize}

\paragraph{Analoge Ausgangskarte}
\begin{itemize}
    \item Geben kontinuierliche Stellgrößen aus
    \item Ansteuerung von Frequenzumrichtern, Ventilen, Reglern
\end{itemize}

\subsubsection{Aufbauarten von SPS}

Speicherprogrammierbare Steuerungen lassen sich
nach ihrem konstruktiven Aufbau
in verschiedene Aufbauarten einteilen.
Diese unterscheiden sich insbesondere hinsichtlich
Leistungsfähigkeit, Erweiterbarkeit und Einsatzgebiet.

\paragraph{Nano-SPS}

Nano-SPS sind sehr kompakte Steuerungen
mit fest integrierten Ein- und Ausgängen.
Sie besitzen nur eine geringe Rechenleistung
und eingeschränkte Programmiermöglichkeiten.
Typische Einsatzgebiete sind einfache Steuerungsaufgaben,
Gebäudetechnik sowie Ausbildungs- und Übungszwecke.

\paragraph{Kompakt-SPS}

Bei Kompakt-SPS sind CPU, Netzteil
und Ein-/Ausgabebaugruppen
in einem gemeinsamen Gehäuse integriert.
Eine Erweiterung ist nur eingeschränkt möglich.
Kompakt-SPS werden häufig in kleinen bis mittleren Maschinen eingesetzt.

\paragraph{Mini-SPS}

Mini-SPS stellen eine leistungsstärkere Form
kompakter Steuerungen dar.
Sie verfügen über größere Programmspeicher
und erweiterte Kommunikationsmöglichkeiten.
Der Einsatz erfolgt typischerweise
in mittleren Automatisierungsanlagen.

\paragraph{Rackbasierte Klein-SPS}

Rackbasierte Klein-SPS sind modular aufgebaute Systeme,
bei denen CPU und Baugruppen
in einem gemeinsamen Rack angeordnet sind.
Die Kommunikation erfolgt über einen Rückwandbus.
Diese Aufbauart bietet eine gute Flexibilität
bei gleichzeitig kompakter Bauweise.

\paragraph{Rackbasierte SPS}

Rackbasierte SPS stellen die leistungsfähigste
klassische Aufbauart dar.
Sie bestehen aus einer CPU
und frei kombinierbaren Baugruppen
für Ein-/Ausgänge, Kommunikation und Spezialfunktionen.
Sie werden in komplexen Maschinen
und großen Industrieanlagen eingesetzt.

\paragraph{Soft-SPS}

Eine Soft-SPS ist eine softwarebasierte Steuerung,
die auf einem Industrie-PC oder Standard-PC läuft.
Die Anbindung der Peripherie erfolgt
über Feldbusse oder dezentrale I/O-Systeme.
Soft-SPS ermöglichen die Kombination
von Steuerung, Visualisierung und Datenverarbeitung
auf einer Plattform.

\medskip
\textbf{Merksatz:}  
\emph{Die Aufbauart bestimmt Flexibilität,
Leistungsfähigkeit und Einsatzbereich einer SPS.}
\begin{figure}[H]
    \centering
    \includegraphics[page=84,width=0.8\textwidth]{/Users/danielweindl/_source/Repositorys/STT-Lernskript/Data/sttvo-k01-Einführung-4v3-Folien_011025.pdf}    
\end{figure}

\subsubsection{Eigenschaften von SPS}

\begin{itemize}
    \item Zyklische, deterministische Programmausführung
    \item Hohe Zuverlässigkeit im industriellen Umfeld
    \item Gute Diagnose- und Fehlermeldemöglichkeiten
    \item Softwarebasierte Anpassbarkeit
    \item Lange Lebensdauer
\end{itemize}

\paragraph{Determinismus}
Eine deterministische Programmausführung liegt vor,
wenn ein Steuerungsprogramm bei gleichen Eingangszuständen,
gleichem Startzustand und identischen Randbedingungen
bei jedem Durchlauf denselben Ablauf und dieselben Ergebnisse liefert.
In SPS-Systemen wird dies durch den festen Zyklus
(Eingänge einlesen, Programm abarbeiten, Ausgänge schreiben)
und eine definierte Abarbeitungsreihenfolge erreicht.
Determinismus gewährleistet,
dass Reaktionszeiten vorhersehbar sind
und das Systemverhalten reproduzierbar bleibt,
was für sichere und zuverlässige Automatisierung
unverzichtbar ist.

\subsubsection{Abgrenzung zu klassischen Steuerungen}

Im Vergleich zu fest verdrahteten elektrischen Steuerungen
bietet die SPS:
\begin{itemize}
    \item geringeren Änderungsaufwand
    \item bessere Wartbarkeit
    \item höhere Funktionalität
\end{itemize}

\medskip
\textbf{Merksatz:}  
\emph{Die SPS ersetzt die Verdrahtungslogik durch Softwarelogik.}


\subsection{Zentrale und dezentrale Steuerungen}

Steuerungssysteme lassen sich nach der räumlichen Anordnung von
Steuereinheit und Ein-/Ausgabebaugruppen in zentrale und dezentrale
Steuerungen einteilen.

\subsubsection{Zentrale Steuerungen}

Bei zentralen Steuerungen befinden sich
CPU, Netzteil und Ein-/Ausgabebaugruppen
in einem gemeinsamen Schaltschrank.

\paragraph{Merkmale}
\begin{itemize}
    \item Zentrale Verarbeitung und zentrale Peripherie
    \item Direkte Verdrahtung aller Sensoren und Aktoren zur SPS
    \item Übersichtliche Systemstruktur
\end{itemize}

\paragraph{Vorteile}
\begin{itemize}
    \item Einfache Projektierung
    \item Geringe Systemkomplexität
    \item Gute Übersicht im Schaltschrank
\end{itemize}

\paragraph{Nachteile}
\begin{itemize}
    \item Hoher Verdrahtungsaufwand bei großen Anlagen
    \item Lange Leitungswege zur Feldebene
\end{itemize}

\paragraph{Typische Anwendung}
\begin{itemize}
    \item Kleine bis mittlere Maschinen
    \item Kompakte Anlagen
\end{itemize}

\begin{figure}[H]
    \centering
    \includegraphics[page=86,width=0.8\textwidth]{/Users/danielweindl/_source/Repositorys/STT-Lernskript/Data/sttvo-k01-Einführung-4v3-Folien_011025.pdf}    
\end{figure}


\subsubsection{Dezentrale Steuerungen}

Bei dezentralen Steuerungen werden
Ein- und Ausgabebaugruppen räumlich
in die Nähe des Prozesses verlagert.
Die Kommunikation mit der zentralen CPU
erfolgt über Feldbussysteme.

\paragraph{Merkmale}
\begin{itemize}
    \item Verteilte Ein-/Ausgabemodule
    \item Kommunikation über Feldbus (z.\,B. Profinet, Profibus, CAN)
    \item CPU kann zentral oder verteilt angeordnet sein
\end{itemize}

\paragraph{Vorteile}
\begin{itemize}
    \item Reduzierter Verdrahtungsaufwand
    \item Kürzere Sensor- und Aktorleitungen
    \item Hohe Skalierbarkeit
\end{itemize}

\paragraph{Nachteile}
\begin{itemize}
    \item Höherer Planungsaufwand
    \item Abhängigkeit von der Feldbuskommunikation
\end{itemize}

\paragraph{Typische Anwendung}
\begin{itemize}
    \item Große und räumlich verteilte Anlagen
    \item Modulare Maschinenkonzepte
\end{itemize}

\begin{figure}[H]
    \centering
    \includegraphics[page=87,width=0.8\textwidth]{/Users/danielweindl/_source/Repositorys/STT-Lernskript/Data/sttvo-k01-Einführung-4v3-Folien_011025.pdf}    
\end{figure}


\subsubsection{Vergleich zentrale vs. dezentrale Steuerung}

\begin{center}
\begin{tabular}{|l|c|c|}
\hline
\textbf{Merkmal} & \textbf{Zentral} & \textbf{Dezentral} \\ \hline
Verdrahtungsaufwand & hoch & gering \\ \hline
Systemkomplexität & gering & höher \\ \hline
Erweiterbarkeit & begrenzt & sehr gut \\ \hline
Typische Größe & klein/mittel & mittel/groß \\ \hline
\end{tabular}
\end{center}

\medskip
\textbf{Merksatz:}  
\emph{Zentrale Steuerungen sind übersichtlich,
dezentrale Steuerungen sind flexibel und skalierbar.}

\begin{figure}[H]
    \centering
    \includegraphics[page=88,width=0.8\textwidth]{/Users/danielweindl/_source/Repositorys/STT-Lernskript/Data/sttvo-k01-Einführung-4v3-Folien_011025.pdf}    
\end{figure}



\subsection{Zuordnungstabelle}
\begin{itemize}
    \item \textbf{Signalbezeichnung}  
    (eindeutiger, sprechender Name des Signals)

    \item \textbf{Signalart}  
    (digitaler Eingang, digitaler Ausgang, analoger Eingang, analoger Ausgang)

    \item \textbf{Physikalische Adresse}  
    (z.\,B. \texttt{E0.0}, \texttt{A2.3}, Analogkanal)

    \item \textbf{SPS-Baugruppe / Steckplatz / Kanal}  
    (Zuordnung zur konkreten Hardware)

    \item \textbf{Feldgerät}  
    (Sensor oder Aktor, z.\,B. Endschalter, Ventil, Motor)

    \item \textbf{Signalbeschreibung / Funktion}  
    (was bewirkt das Signal im Prozess)
\end{itemize}
\begin{figure}[H]
    \centering
    \includegraphics[width=0.9\textwidth]{/Users/danielweindl/_source/Repositorys/STT-Lernskript/Bilder/Zuordnungstabelle.png}
\end{figure}

\subsection{Feldebene}

Die Feldebene bildet die unterste Ebene der Automatisierungspyramide
und umfasst alle technischen Einrichtungen,
die \textbf{direkt mit dem Prozess gekoppelt} sind.
Dazu zählen Sensoren zur Erfassung von Prozessgrößen
sowie Aktoren zur Beeinflussung des Prozesses.
Die Kommunikation auf der Feldebene erfolgt über
binäre oder analoge Signale,
die von Ein-/Ausgabebaugruppen oder Feldbussystemen
(z.\,B. Profibus, Profinet, CAN)
zur übergeordneten Steuerung übertragen werden.
Die Feldebene stellt damit die Schnittstelle
zwischen realem Prozess und Steuerungssystem dar.

\begin{itemize}
    \item \textbf{Sensoren}  
    (z.\,B. Endschalter, Näherungsschalter, Lichtschranken, Druck-, Temperatur- und Wegsensoren)
    
    \item \textbf{Aktoren}  
    (z.\,B. Elektromotoren, Ventile, Zylinder, Relais, Schütze)
    
    \item \textbf{Ein- und Ausgabegeräte}  
    (digitale und analoge E/A-Baugruppen, dezentrale I/O-Module)
    
    \item \textbf{Feldgeräte}  
    (Messumformer, Stellgeräte, intelligente Sensoren)
    
    \item \textbf{Feldbussysteme und Signalübertragung}  
    (z.\,B. Profibus, Profinet, CAN, IO-Link)
\end{itemize}
\section{Einleitung}

\subsection{Einordnung der Steuerungstechnik}

Die Steuerungstechnik ist eine Teildisziplin der Automatisierungstechnik und beschäftigt sich
mit dem gezielten Beeinflussen technischer Prozesse durch vorgegebene Vorschriften,
Algorithmen oder Gesetzmäßigkeiten.
Sie umfasst den \textbf{Entwurf}, die \textbf{Realisierung}, das \textbf{Testen},
die \textbf{Inbetriebnahme} sowie die \textbf{Wartung} von Steuerungssystemen.

Ein technischer Prozess ist ein Vorgang zur \textbf{Umformung}, \textbf{Speicherung}
oder zum \textbf{Transport} von Material, Energie oder Information.
Der Zustand eines Prozesses wird durch sogenannte \emph{Zustandsgrößen} beschrieben,
die technisch erfassbar und beeinflussbar sind.

Die Automatisierungstechnik gliedert sich unter anderem in:
\begin{itemize}
    \item Sensorik (Informationsgewinnung)
    \item Informationsverarbeitung (z.\,B. SPS, Industrie-PC)
    \item Aktorik (Informationsumsetzung)
    \item Mensch-Maschine-Schnittstelle (Visualisierung, Bedienung)
\end{itemize}

\subsection{Steuern und Regeln}

\paragraph{Steuern}
Unter Steuern versteht man einen Vorgang, bei dem Eingangsgrößen
Ausgangsgrößen eines Systems beeinflussen, \textbf{ohne dass die Ausgangsgrößen
rückgeführt und fortlaufend verglichen werden}.
Kennzeichnend ist der \textbf{offene Wirkungsablauf}.

Die Steuerungstechnik ist die gezielte Anwendung von Steuerungsmechanismen
(Vorschriften, Algorithmen, Gesetzmäßigkeiten) unter Zuhilfenahme der verfügbaren
technischen Mittel zur zielgerichteten Erfüllung von Aufgaben in Prozessen.

\paragraph{Regeln}
Beim Regeln wird eine Regelgröße kontinuierlich erfasst,
mit einer Führungsgröße verglichen und in Richtung dieser Führungsgröße beeinflusst.
Kennzeichnend ist der \textbf{geschlossene Wirkungsablauf (Rückkopplung)}.

\medskip
\textbf{Merksatz:}  
\emph{Steuern arbeitet ohne Rückführung, Regeln immer mit Rückführung.}

\subsection{Offene und geschlossene Steuerungen}

\begin{itemize}
    \item \textbf{Offene Steuerung}:  
    Keine Rückmeldung aus dem Prozess.  
    Beispiel: Zeitgesteuerte Ampel.
    
    \item \textbf{Geschlossene Steuerung}:  
    Rückmeldung beeinflusst den Steuerungsablauf, jedoch nicht kontinuierlich.  
    Beispiel: Abschaltung über Endschalter.
    
    \item \textbf{Regelung}:  
    Kontinuierliche Rückführung und Korrektur.  
    Beispiel: Drehzahlregelung eines Motors.
\end{itemize}

 
\begin{figure}[H]
    \centering
    \includegraphics[page=12,width=0.8\textwidth]{/Users/danielweindl/_source/Repositorys/STT-Lernskript/Data/sttvo-k01-Einführung-4v3-Folien_011025.pdf}    
 \end{figure}


\subsection{Grundstruktur eines Steuerungssystems}

Ein Steuerungssystem besteht grundsätzlich aus:
\begin{itemize}
    \item Führungsgrößen (z.\,B. Start, Stop, Sollwerte)
    \item Steuereinrichtung (z.\,B. SPS)
    \item Steuerstrecke (technischer Prozess)
    \item Stellgrößen (Ausgangsgrößen der Steuerung)
    \item Meldegrößen (Rückmeldungen, Anzeigen)
    \item Störgrößen (unerwünschte Einflüsse)
\end{itemize}

\subsection{Klassifikation von Steuerungen}

\subsubsection{Nach Informationsdarstellung}

\begin{itemize}
    \item \textbf{Analoge Steuerungen}:  
    Stetige Signalverarbeitung (z.\,B. Spannungen, Ströme)
    
    \item \textbf{Digitale Steuerungen}:  
    Verarbeitung von Zahlenwerten und Codes
    
    \item \textbf{Binäre Steuerungen}:  
    Verarbeitung logischer Zustände (TRUE/FALSE)
\end{itemize}

\begin{figure}[H]
    \centering
    \includegraphics[width=0.6\textwidth]{/Users/danielweindl/_source/Repositorys/STT-Lernskript/Bilder/BinäreSignalverarbeitung.png}
\end{figure}

\subsubsection{Nach Art der Signalverarbeitung}
\begin{figure}[H]
    \centering
    \includegraphics[width=0.8\textwidth]{/Users/danielweindl/_source/Repositorys/STT-Lernskript/Bilder/ArtenSignalverarbeitung.png}
\end{figure}

\begin{itemize}
    \item \textbf{Synchrone Steuerungen}:  
    Abtastung und Verarbeitung im festen Zeitraster
    
    \item \textbf{Asynchrone Steuerungen}:  
    Ereignisgesteuerte Verarbeitung
    
\begin{figure}[H]
    \centering
    \includegraphics[width=1\textwidth]{/Users/danielweindl/_source/Repositorys/STT-Lernskript/Bilder/SynchronAsynchron.png}
\end{figure}

    \item \textbf{Verknüpfungssteuerungen}:  
    Ausgang hängt nur vom aktuellen Eingang ab
    
    \item \textbf{Ablaufsteuerungen}:  
    Ausgang hängt von Eingang \emph{und} Zustand ab
\end{itemize}

\subsection{Kategorien von Steuerungen}

Die Realisierung einer Steuerung beschreibt,
\textbf{mit welchen technischen Mitteln}
die Steuerungsfunktion umgesetzt wird.
Man unterscheidet mehrere grundlegende Realisierungsarten,
die historisch und technisch unterschiedliche Eigenschaften besitzen.

\begin{figure}[H]
    \centering
    \includegraphics[width=0.9\textwidth]{/Users/danielweindl/_source/Repositorys/STT-Lernskript/Bilder/UnterscheidungEnergieübertragung.png}
\end{figure}


\subsubsection{Mechanische Steuerungen}

Mechanische Steuerungen realisieren Steuerungsfunktionen
ausschließlich durch mechanische Elemente.
Die Steuerinformation wird durch Kräfte, Wege oder Bewegungen erzeugt.

\paragraph{Typische Elemente}
\begin{itemize}
    \item Anschläge
    \item Nocken
    \item Kurvenscheiben
    \item Gestänge
\end{itemize}

\paragraph{Eigenschaften}
\begin{itemize}
    \item Kein elektrischer Energiebedarf
    \item Ablauf direkt an Bewegung gekoppelt
    \item Sehr robust
    \item Kaum flexibel
\end{itemize}

\paragraph{Beispiele}
\begin{itemize}
    \item Waschmaschinen mit mechanischem Programmwerk
    \item Nockensteuerungen in Werkzeugmaschinen
    \item Drehorgeln, Spieluhren
\end{itemize}

\paragraph{Einordnung}
Mechanische Steuerungen sind historisch bedeutsam,
werden heute jedoch kaum noch neu eingesetzt,
da Änderungen nur mechanisch möglich sind.

\subsubsection{Pneumatische Steuerungen}

Pneumatische Steuerungen nutzen Druckluft
zur Signalübertragung und zur Energieversorgung der Aktoren.

\paragraph{Typische Elemente}
\begin{itemize}
    \item Pneumatikzylinder
    \item Wegeventile
    \item Drosseln
    \item Pneumatische Logikelemente
\end{itemize}

\paragraph{Eigenschaften}
\begin{itemize}
    \item Schnelle Schaltzeiten
    \item Relativ einfache Logik möglich
    \item Begrenzte Genauigkeit
    \item Kompressibilität der Luft
\end{itemize}

\paragraph{Beispiele}
\begin{itemize}
    \item Handhabungssysteme
    \item Einfache Montageautomaten
\end{itemize}

\paragraph{Einordnung}
Pneumatische Steuerungen werden heute meist
in Kombination mit SPS eingesetzt
(SPS steuert Ventile, Pneumatik führt aus).

 

\subsubsection{Hydraulische Steuerungen}

Hydraulische Steuerungen verwenden
nicht oder kaum kompressible Flüssigkeiten
zur Kraft- und Signalübertragung.

\paragraph{Typische Elemente}
\begin{itemize}
    \item Hydraulikzylinder
    \item Steuerventile
    \item Pumpen
    \item Druckspeicher
\end{itemize}

\paragraph{Eigenschaften}
\begin{itemize}
    \item Sehr hohe Stellkräfte möglich
    \item Gute Regelbarkeit
    \item Hoher technischer Aufwand
    \item Wartungsintensiv
\end{itemize}

\paragraph{Beispiele}
\begin{itemize}
    \item Pressen
    \item Spritzgussmaschinen
\end{itemize}

\paragraph{Einordnung}
Hydraulische Steuerungen sind heute
meist Teil der Steuerstrecke,
während die Steuerlogik elektrisch oder per SPS realisiert wird.

 

\subsubsection{Elektrische Steuerungen}

Elektrische Steuerungen realisieren die Steuerungsfunktion
durch elektrische Schaltgeräte und Leitungsverknüpfungen.

\paragraph{Typische Elemente}
\begin{itemize}
    \item Relais
    \item Schütze
    \item Schalter
    \item Zeitrelais
\end{itemize}

\paragraph{Eigenschaften}
\begin{itemize}
    \item Binäre Logik gut realisierbar
    \item Hohe Übersichtlichkeit im Stromlaufplan
    \item Änderungen nur durch Umverdrahtung
    \item Mechanischer Verschleiß
\end{itemize}

\paragraph{Beispiele}
\begin{itemize}
    \item Motorsteuerungen
    \item Aufzugssteuerungen (klassisch)
\end{itemize}

\paragraph{Einordnung}
Elektrische Steuerungen sind die direkte
\textbf{Vorgänger-Technologie der SPS}
und Grundlage für KOP-Darstellungen.

 

\subsubsection{Elektronische Steuerungen}

Elektronische Steuerungen verwenden
kontaktlose elektronische Bauelemente
zur Realisierung der Steuerungslogik.

\paragraph{Typische Elemente}
\begin{itemize}
    \item Transistoren
    \item Logikgatter
    \item Mikrocontroller
    \item FPGA
\end{itemize}

\paragraph{Eigenschaften}
\begin{itemize}
    \item Hohe Schaltgeschwindigkeit
    \item Kein mechanischer Verschleiß
    \item Geringe Leistungsfähigkeit für Aktoren
    \item Hoher Entwicklungsaufwand
\end{itemize}

\paragraph{Einordnung}
Elektronische Steuerungen werden heute
vor allem in eingebetteten Systemen eingesetzt
und bilden die Hardwarebasis moderner SPS.

\subsection{Speicherprogrammierbare Steuerung (SPS)}

\begin{figure}[H]
    \centering
    \includegraphics[width=0.6\textwidth]{/Users/danielweindl/_source/Repositorys/STT-Lernskript/Bilder/Automatisierungspyramide.png}
\end{figure}
\subsubsection{Begriff}

Eine speicherprogrammierbare Steuerung (SPS) ist eine
digital arbeitende Steuerungseinrichtung zur Steuerung von Maschinen
und Anlagen.
Die Steuerungsfunktion wird nicht durch Verdrahtung,
sondern durch ein im Speicher abgelegtes Programm realisiert.


Unter einer SPS versteht man einen speziellen Computer mit mehreren (in der Regel) digitalen Ein‐/Ausgängen,
der logische Verknüpfungen (programmierbar) zwischen den Ein‐ und Ausgängen herstellt.
Im Prinzip können Steuerungsaufgaben auch von ganz normalen PCs erfolgen. Entsprechende Software (Soft‐SPS für
PCs) ist für diesen Zweck auch verfügbar. Vom Standpunkt der Programmierung aus ist es unerheblich, auf welchem
System die SPS‐Funktion umgesetzt wird. Dennoch werden Steuerungen überwiegend auf speziellen SPS‐Computern
verwirklicht. Folgende Gründe sprechen für den Einsatz spezieller SPS‐Geräte:
\subsubsection{Grundelemente einer SPS}

Eine SPS besteht aus folgenden grundlegenden Elementen:

\begin{itemize}
    \item \textbf{Zentraleinheit (CPU)}  
    Führt das Steuerungsprogramm aus, verwaltet Speicher,
    steuert den Zyklus und koordiniert die Kommunikation.

    \item \textbf{Netzteil}  
    Versorgt CPU und Baugruppen mit der erforderlichen Betriebsspannung
    (typisch 24\,V DC).

    \item \textbf{Eingangskarte}  
    Erfassen binäre oder analoge Signale aus der Feldebene
    (Sensoren).

    \item \textbf{Ausgangskarte}  
    Stellen binäre oder analoge Signale zur Ansteuerung von Aktoren bereit.

    \item \textbf{Kommunikationsbaugruppen}  
    Ermöglichen den Datenaustausch mit anderen SPSen,
    Feldgeräten oder Leitsystemen.

    \item \textbf{Rückwandbus / internes Bussystem}  
    Verbindet CPU und Baugruppen und ermöglicht den internen Datenaustausch.
\end{itemize}

\subsubsection{Baugruppen einer SPS}

\paragraph{Digitale Eingangskarte}
\begin{itemize}
    \item Erfassen binäre Zustände (0/1)
    \item Typische Signale: Schalter, Taster, Endschalter
    \item Galvanische Trennung zur Feldebene
\end{itemize}

\paragraph{Digitale Ausgangskarte}
\begin{itemize}
    \item Schalten binäre Lasten
    \item Typisch: Relais-, Transistor- oder Triac-Ausgänge
\end{itemize}

\paragraph{Analoge Eingangskarte}
\begin{itemize}
    \item Erfassen kontinuierliche Messwerte
    \item Typische Signale: 0–10\,V, 4–20\,mA
\end{itemize}

\paragraph{Analoge Ausgangskarte}
\begin{itemize}
    \item Geben kontinuierliche Stellgrößen aus
    \item Ansteuerung von Frequenzumrichtern, Ventilen, Reglern
\end{itemize}

\subsubsection{Aufbauarten von SPS}

Speicherprogrammierbare Steuerungen lassen sich
nach ihrem konstruktiven Aufbau
in verschiedene Aufbauarten einteilen.
Diese unterscheiden sich insbesondere hinsichtlich
Leistungsfähigkeit, Erweiterbarkeit und Einsatzgebiet.

\paragraph{Nano-SPS}

Nano-SPS sind sehr kompakte Steuerungen
mit fest integrierten Ein- und Ausgängen.
Sie besitzen nur eine geringe Rechenleistung
und eingeschränkte Programmiermöglichkeiten.
Typische Einsatzgebiete sind einfache Steuerungsaufgaben,
Gebäudetechnik sowie Ausbildungs- und Übungszwecke.

\paragraph{Kompakt-SPS}

Bei Kompakt-SPS sind CPU, Netzteil
und Ein-/Ausgabebaugruppen
in einem gemeinsamen Gehäuse integriert.
Eine Erweiterung ist nur eingeschränkt möglich.
Kompakt-SPS werden häufig in kleinen bis mittleren Maschinen eingesetzt.

\paragraph{Mini-SPS}

Mini-SPS stellen eine leistungsstärkere Form
kompakter Steuerungen dar.
Sie verfügen über größere Programmspeicher
und erweiterte Kommunikationsmöglichkeiten.
Der Einsatz erfolgt typischerweise
in mittleren Automatisierungsanlagen.

\paragraph{Rackbasierte Klein-SPS}

Rackbasierte Klein-SPS sind modular aufgebaute Systeme,
bei denen CPU und Baugruppen
in einem gemeinsamen Rack angeordnet sind.
Die Kommunikation erfolgt über einen Rückwandbus.
Diese Aufbauart bietet eine gute Flexibilität
bei gleichzeitig kompakter Bauweise.

\paragraph{Rackbasierte SPS}

Rackbasierte SPS stellen die leistungsfähigste
klassische Aufbauart dar.
Sie bestehen aus einer CPU
und frei kombinierbaren Baugruppen
für Ein-/Ausgänge, Kommunikation und Spezialfunktionen.
Sie werden in komplexen Maschinen
und großen Industrieanlagen eingesetzt.

\paragraph{Soft-SPS}

Eine Soft-SPS ist eine softwarebasierte Steuerung,
die auf einem Industrie-PC oder Standard-PC läuft.
Die Anbindung der Peripherie erfolgt
über Feldbusse oder dezentrale I/O-Systeme.
Soft-SPS ermöglichen die Kombination
von Steuerung, Visualisierung und Datenverarbeitung
auf einer Plattform.

\medskip
\textbf{Merksatz:}  
\emph{Die Aufbauart bestimmt Flexibilität,
Leistungsfähigkeit und Einsatzbereich einer SPS.}
\begin{figure}[H]
    \centering
    \includegraphics[page=84,width=0.8\textwidth]{/Users/danielweindl/_source/Repositorys/STT-Lernskript/Data/sttvo-k01-Einführung-4v3-Folien_011025.pdf}    
\end{figure}

\subsubsection{Eigenschaften von SPS}

\begin{itemize}
    \item Zyklische, deterministische Programmausführung
    \item Hohe Zuverlässigkeit im industriellen Umfeld
    \item Gute Diagnose- und Fehlermeldemöglichkeiten
    \item Softwarebasierte Anpassbarkeit
    \item Lange Lebensdauer
\end{itemize}

\paragraph{Determinismus}
Eine deterministische Programmausführung liegt vor,
wenn ein Steuerungsprogramm bei gleichen Eingangszuständen,
gleichem Startzustand und identischen Randbedingungen
bei jedem Durchlauf denselben Ablauf und dieselben Ergebnisse liefert.
In SPS-Systemen wird dies durch den festen Zyklus
(Eingänge einlesen, Programm abarbeiten, Ausgänge schreiben)
und eine definierte Abarbeitungsreihenfolge erreicht.
Determinismus gewährleistet,
dass Reaktionszeiten vorhersehbar sind
und das Systemverhalten reproduzierbar bleibt,
was für sichere und zuverlässige Automatisierung
unverzichtbar ist.

\subsubsection{Abgrenzung zu klassischen Steuerungen}

Im Vergleich zu fest verdrahteten elektrischen Steuerungen
bietet die SPS:
\begin{itemize}
    \item geringeren Änderungsaufwand
    \item bessere Wartbarkeit
    \item höhere Funktionalität
\end{itemize}

\medskip
\textbf{Merksatz:}  
\emph{Die SPS ersetzt die Verdrahtungslogik durch Softwarelogik.}


\subsection{Zentrale und dezentrale Steuerungen}

Steuerungssysteme lassen sich nach der räumlichen Anordnung von
Steuereinheit und Ein-/Ausgabebaugruppen in zentrale und dezentrale
Steuerungen einteilen.

\subsubsection{Zentrale Steuerungen}

Bei zentralen Steuerungen befinden sich
CPU, Netzteil und Ein-/Ausgabebaugruppen
in einem gemeinsamen Schaltschrank.

\paragraph{Merkmale}
\begin{itemize}
    \item Zentrale Verarbeitung und zentrale Peripherie
    \item Direkte Verdrahtung aller Sensoren und Aktoren zur SPS
    \item Übersichtliche Systemstruktur
\end{itemize}

\paragraph{Vorteile}
\begin{itemize}
    \item Einfache Projektierung
    \item Geringe Systemkomplexität
    \item Gute Übersicht im Schaltschrank
\end{itemize}

\paragraph{Nachteile}
\begin{itemize}
    \item Hoher Verdrahtungsaufwand bei großen Anlagen
    \item Lange Leitungswege zur Feldebene
\end{itemize}

\paragraph{Typische Anwendung}
\begin{itemize}
    \item Kleine bis mittlere Maschinen
    \item Kompakte Anlagen
\end{itemize}

\begin{figure}[H]
    \centering
    \includegraphics[page=86,width=0.8\textwidth]{/Users/danielweindl/_source/Repositorys/STT-Lernskript/Data/sttvo-k01-Einführung-4v3-Folien_011025.pdf}    
\end{figure}


\subsubsection{Dezentrale Steuerungen}

Bei dezentralen Steuerungen werden
Ein- und Ausgabebaugruppen räumlich
in die Nähe des Prozesses verlagert.
Die Kommunikation mit der zentralen CPU
erfolgt über Feldbussysteme.

\paragraph{Merkmale}
\begin{itemize}
    \item Verteilte Ein-/Ausgabemodule
    \item Kommunikation über Feldbus (z.\,B. Profinet, Profibus, CAN)
    \item CPU kann zentral oder verteilt angeordnet sein
\end{itemize}

\paragraph{Vorteile}
\begin{itemize}
    \item Reduzierter Verdrahtungsaufwand
    \item Kürzere Sensor- und Aktorleitungen
    \item Hohe Skalierbarkeit
\end{itemize}

\paragraph{Nachteile}
\begin{itemize}
    \item Höherer Planungsaufwand
    \item Abhängigkeit von der Feldbuskommunikation
\end{itemize}

\paragraph{Typische Anwendung}
\begin{itemize}
    \item Große und räumlich verteilte Anlagen
    \item Modulare Maschinenkonzepte
\end{itemize}

\begin{figure}[H]
    \centering
    \includegraphics[page=87,width=0.8\textwidth]{/Users/danielweindl/_source/Repositorys/STT-Lernskript/Data/sttvo-k01-Einführung-4v3-Folien_011025.pdf}    
\end{figure}


\subsubsection{Vergleich zentrale vs. dezentrale Steuerung}

\begin{center}
\begin{tabular}{|l|c|c|}
\hline
\textbf{Merkmal} & \textbf{Zentral} & \textbf{Dezentral} \\ \hline
Verdrahtungsaufwand & hoch & gering \\ \hline
Systemkomplexität & gering & höher \\ \hline
Erweiterbarkeit & begrenzt & sehr gut \\ \hline
Typische Größe & klein/mittel & mittel/groß \\ \hline
\end{tabular}
\end{center}

\medskip
\textbf{Merksatz:}  
\emph{Zentrale Steuerungen sind übersichtlich,
dezentrale Steuerungen sind flexibel und skalierbar.}

\begin{figure}[H]
    \centering
    \includegraphics[page=88,width=0.8\textwidth]{/Users/danielweindl/_source/Repositorys/STT-Lernskript/Data/sttvo-k01-Einführung-4v3-Folien_011025.pdf}    
\end{figure}



\subsection{Zuordnungstabelle}
\begin{itemize}
    \item \textbf{Signalbezeichnung}  
    (eindeutiger, sprechender Name des Signals)

    \item \textbf{Signalart}  
    (digitaler Eingang, digitaler Ausgang, analoger Eingang, analoger Ausgang)

    \item \textbf{Physikalische Adresse}  
    (z.\,B. \texttt{E0.0}, \texttt{A2.3}, Analogkanal)

    \item \textbf{SPS-Baugruppe / Steckplatz / Kanal}  
    (Zuordnung zur konkreten Hardware)

    \item \textbf{Feldgerät}  
    (Sensor oder Aktor, z.\,B. Endschalter, Ventil, Motor)

    \item \textbf{Signalbeschreibung / Funktion}  
    (was bewirkt das Signal im Prozess)
\end{itemize}
\begin{figure}[H]
    \centering
    \includegraphics[width=0.9\textwidth]{/Users/danielweindl/_source/Repositorys/STT-Lernskript/Bilder/Zuordnungstabelle.png}
\end{figure}

\subsection{Feldebene}

Die Feldebene bildet die unterste Ebene der Automatisierungspyramide
und umfasst alle technischen Einrichtungen,
die \textbf{direkt mit dem Prozess gekoppelt} sind.
Dazu zählen Sensoren zur Erfassung von Prozessgrößen
sowie Aktoren zur Beeinflussung des Prozesses.
Die Kommunikation auf der Feldebene erfolgt über
binäre oder analoge Signale,
die von Ein-/Ausgabebaugruppen oder Feldbussystemen
(z.\,B. Profibus, Profinet, CAN)
zur übergeordneten Steuerung übertragen werden.
Die Feldebene stellt damit die Schnittstelle
zwischen realem Prozess und Steuerungssystem dar.

\begin{itemize}
    \item \textbf{Sensoren}  
    (z.\,B. Endschalter, Näherungsschalter, Lichtschranken, Druck-, Temperatur- und Wegsensoren)
    
    \item \textbf{Aktoren}  
    (z.\,B. Elektromotoren, Ventile, Zylinder, Relais, Schütze)
    
    \item \textbf{Ein- und Ausgabegeräte}  
    (digitale und analoge E/A-Baugruppen, dezentrale I/O-Module)
    
    \item \textbf{Feldgeräte}  
    (Messumformer, Stellgeräte, intelligente Sensoren)
    
    \item \textbf{Feldbussysteme und Signalübertragung}  
    (z.\,B. Profibus, Profinet, CAN, IO-Link)
\end{itemize}
\newpage
\section{Diskrete Steuerungen}

\subsection{Begriff der diskreten Steuerung}

Diskrete Steuerungen sind Steuerungen,
bei denen Ein- und Ausgangsgrößen
nur \textbf{diskrete Zustände} annehmen.
In der Praxis handelt es sich überwiegend
um binäre Zustände (0/1, FALSE/TRUE).

Die Verarbeitung erfolgt auf Basis
logischer Verknüpfungen, Speicherwirkungen
und zeitlicher Abfolgen.

\subsection{Hierarchische Abstraktion}
In der Abbildung soll ersichtlich werden wie die Begriffe zusammenhängen. Die Abstraktion nimmt nach unten hin zu.

\vspace*{1cm}
\begin{tikzpicture}[
    box/.style={
        draw,
        rectangle,
        rounded corners,
        align=center,
        minimum width=8cm,
        minimum height=1cm
    },
    node distance=1.2cm
]

% Hierarchie
\node[box] (se) {\textbf{Schaltelemente}\\
Relais, Kontakte, Transistoren};

\node[box, below=of se] (ls) {\textbf{Logische Schaltungen}\\
Boolesche Verknüpfungen};

\node[box, below=of ls] (sn) {\textbf{Schaltnetze}\\
speicherlose logische Systeme};

\node[box, below=of sn] (f) {\textbf{Funktionen}\\
abstrahierte Schaltnetze\\
(ein Ausgang, kein Speicher)};

\node[box, below=of f] (sw) {\textbf{Schaltwerke}\\
logische Systeme mit Speicher};

\node[box, below=of sw, thick] (fb) {\textbf{Funktionsbausteine}\\
gekapselte Schaltwerke};

% Pfeile
\draw[->] (se) -- (ls);
\draw[->] (ls) -- (sn);
\draw[->] (sn) -- (f);
\draw[->] (f) -- (sw);
\draw[->] (sw) -- (fb);

% Schaltsysteme-Klammer (Sammelbegriff)
\draw[dashed, thick, rounded corners]
    ($(sn.north west)+(-0.6,0.3)$) rectangle
    ($(sw.south east)+(0.6,-0.3)$);

\node[align=center] at ($(sn.west)+(-2.3,-1.2)$)
{\textbf{Schaltsysteme}\\
(Oberbegriff)};

\end{tikzpicture}

\subsection{Schaltelemente}

Schaltelemente sind die elementaren
Bausteine diskreter Steuerungen.
Sie besitzen klar definierte Eingänge
und einen oder mehrere Ausgänge.

\paragraph{Grundlegende Schaltelemente}
\begin{itemize}
    \item Schalter
    \item Relais
    \item Schütze
    \item Elektronische Schaltelemente (Transistoren)
\end{itemize}

Schaltelemente können Signale:
\begin{itemize}
    \item schalten
    \item verknüpfen
    \item speichern
\end{itemize}

\subsection{Grundlagen logischer Schaltungen}

Logische Schaltungen verarbeiten binäre Signale
nach den Regeln der Booleschen Algebra.

\paragraph{Logische Grundfunktionen}
\begin{itemize}
    \item TRUE / FALSE
    \item AND
    \item OR
    \item NOT
\end{itemize}

Diese Grundfunktionen können durch
Schaltelemente realisiert werden
(z.\,B. Reihenschaltung, Parallelschaltung).

\paragraph{Erweiterte logische Funktionen}
\begin{itemize}
    \item Exklusiv-ODER (XOR)
    \item NAND
    \item NOR
\end{itemize}

\subsection{Darstellung von Logikfunktionen durch Wertetabellen}

Eine Logikfunktion beschreibt den funktionalen Zusammenhang
zwischen binären Eingangsgrößen und einer binären Ausgangsgröße.
Die Eingänge und der Ausgang können ausschließlich die Werte
\texttt{0} (FALSE) oder \texttt{1} (TRUE) annehmen.

Logikfunktionen bilden die Grundlage aller diskreten Steuerungen
und werden mithilfe der Booleschen Algebra beschrieben.

\subsection{Boolesche Grundfunktionen}

\subsubsection{NICHT-Funktion (NOT)}

Die NICHT-Funktion invertiert den Eingang.

\[
y = \lnot a
\]

\paragraph{Wertetabelle}
\begin{center}
\begin{tabular}{|c|c|}
\hline
$a$ & $y = \lnot a$ \\ \hline
0 & 1 \\ \hline
1 & 0 \\ \hline
\end{tabular}
\end{center}

\subsubsection{UND-Funktion (AND)}

Die UND-Funktion liefert nur dann den Wert 1,
wenn \textbf{alle Eingänge} den Wert 1 besitzen.

\[
y = a \land b
\]

\paragraph{Wertetabelle}
\begin{center}
\begin{tabular}{|c|c|c|}
\hline
$a$ & $b$ & $y = a \land b$ \\ \hline
0 & 0 & 0 \\ \hline
0 & 1 & 0 \\ \hline
1 & 0 & 0 \\ \hline
1 & 1 & 1 \\ \hline
\end{tabular}
\end{center}

\subsubsection{ODER-Funktion (OR)}

Die ODER-Funktion liefert den Wert 1,
wenn \textbf{mindestens ein Eingang} den Wert 1 besitzt.

\[
y = a \lor b
\]

\paragraph{Wertetabelle}
\begin{center}
\begin{tabular}{|c|c|c|}
\hline
$a$ & $b$ & $y = a \lor b$ \\ \hline
0 & 0 & 0 \\ \hline
0 & 1 & 1 \\ \hline
1 & 0 & 1 \\ \hline
1 & 1 & 1 \\ \hline
\end{tabular}
\end{center}

\subsection{Erweiterte Logikfunktionen}

\subsubsection{Exklusiv-ODER-Funktion (XOR)}

Die Exklusiv-ODER-Funktion liefert genau dann den Wert 1,
wenn die Eingänge \textbf{unterschiedlich} sind.

\[
y = a \oplus b
\]

\paragraph{Wertetabelle}
\begin{center}
\begin{tabular}{|c|c|c|}
\hline
$a$ & $b$ & $y = a \oplus b$ \\ \hline
0 & 0 & 0 \\ \hline
0 & 1 & 1 \\ \hline
1 & 0 & 1 \\ \hline
1 & 1 & 0 \\ \hline
\end{tabular}
\end{center}

\subsubsection{NAND-Funktion}

Die NAND-Funktion ist die Negation der UND-Funktion.

\[
y = \lnot (a \land b)
\]

\paragraph{Wertetabelle}
\begin{center}
\begin{tabular}{|c|c|c|}
\hline
$a$ & $b$ & $y$ \\ \hline
0 & 0 & 1 \\ \hline
0 & 1 & 1 \\ \hline
1 & 0 & 1 \\ \hline
1 & 1 & 0 \\ \hline
\end{tabular}
\end{center}

\subsubsection{NOR-Funktion}

Die NOR-Funktion ist die Negation der ODER-Funktion.

\[
y = \lnot (a \lor b)
\]

\paragraph{Wertetabelle}
\begin{center}
\begin{tabular}{|c|c|c|}
\hline
$a$ & $b$ & $y$ \\ \hline
0 & 0 & 1 \\ \hline
0 & 1 & 0 \\ \hline
1 & 0 & 0 \\ \hline
1 & 1 & 0 \\ \hline
\end{tabular}
\end{center}

\subsubsection{Zusammenhang zu Schaltelementen}

Logikfunktionen können physikalisch
durch Schaltelemente realisiert werden:

\begin{itemize}
    \item UND-Funktion → Reihenschaltung von Kontakten
    \item ODER-Funktion → Parallelschaltung von Kontakten
    \item NICHT-Funktion → Öffnerkontakt
\end{itemize}

Diese Abbildungen bilden die Grundlage
für die Darstellung in KOP
sowie für elektrische Steuerungen.

\subsubsection{Zusammenfassung}

Logikfunktionen beschreiben die grundlegenden Zusammenhänge
binärer Steuerungssysteme.
Sie sind Grundlage für Schaltnetze,
Schaltwerke und Funktionsbausteine.

\medskip
\textbf{Merksatz:}  
\emph{Jede diskrete Steuerung lässt sich auf Logikfunktionen zurückführen.}

\subsection{Minterme und Maxterme}

\subsubsection{Grundidee}

Minterme und Maxterme sind standardisierte Darstellungsformen
logischer Funktionen auf Basis von Wertetabellen.
Sie ermöglichen eine systematische und eindeutige Beschreibung
beliebiger Logikfunktionen mithilfe der Booleschen Algebra.

\subsubsection{Minterm}

Ein \textbf{Minterm} ist eine logische UND-Verknüpfung
aller Eingangsvariablen oder ihrer Negationen,
die \textbf{genau für eine bestimmte Eingangskombination den Wert 1} liefert.

\paragraph{Eigenschaften eines Minterms}
\begin{itemize}
    \item Enthält alle Eingangsvariablen genau einmal
    \item Nicht negierte Variable entspricht Eingang = 1
    \item Negierte Variable entspricht Eingang = 0
    \item Ergebnis ist nur für eine Kombination TRUE
\end{itemize}

\paragraph{Beispiel}

Für zwei Eingänge $a$ und $b$:

\[
m_2 = a \land \lnot b
\]

Dieser Minterm ist genau dann TRUE, wenn:
\[
a = 1 \;\land\; b = 0
\]

\subsubsection{Minterme aus der Wertetabelle}

Gegeben sei folgende Wertetabelle:

\begin{center}
\begin{tabular}{|c|c|c|}
\hline
$a$ & $b$ & $y$ \\ \hline
0 & 0 & 0 \\ \hline
0 & 1 & 1 \\ \hline
1 & 0 & 1 \\ \hline
1 & 1 & 0 \\ \hline
\end{tabular}
\end{center}

Die Funktion ist für die Eingangskombinationen
$(0,1)$ und $(1,0)$ gleich 1.

Daraus ergeben sich die Minterme:
\[
m_1 = \lnot a \land b
\]
\[
m_2 = a \land \lnot b
\]

\paragraph{Disjunktive Normalform (DNF)}

Die vollständige Darstellung der Funktion lautet:
\[
y = m_1 \lor m_2
\]

\medskip
\textbf{Merksatz:}  
\emph{Minterme beschreiben alle Eingangskombinationen,
für die die Funktion den Wert 1 annimmt.}

\subsubsection{Maxterm}

Ein \textbf{Maxterm} ist eine logische ODER-Verknüpfung
aller Eingangsvariablen oder ihrer Negationen,
die \textbf{genau für eine bestimmte Eingangskombination den Wert 0} liefert.

\paragraph{Eigenschaften eines Maxterms}
\begin{itemize}
    \item Enthält alle Eingangsvariablen genau einmal
    \item Nicht negierte Variable entspricht Eingang = 0
    \item Negierte Variable entspricht Eingang = 1
    \item Ergebnis ist nur für eine Kombination FALSE
\end{itemize}

\paragraph{Beispiel}

\[
M_2 = a \lor \lnot b
\]

Dieser Maxterm ist genau dann FALSE, wenn:
\[
a = 0 \;\land\; b = 1
\]

\subsubsection{Maxterme aus der Wertetabelle}

Aus der obigen Wertetabelle ergibt sich:
Die Funktion ist für die Eingangskombinationen
$(0,0)$ und $(1,1)$ gleich 0.

Daraus ergeben sich die Maxterme:
\[
M_0 = a \lor b
\]
\[
M_3 = \lnot a \lor \lnot b
\]

\paragraph{Konjunktive Normalform (KNF)}

Die vollständige Darstellung der Funktion lautet:
\[
y = M_0 \land M_3
\]

\medskip
\textbf{Merksatz:}  
\emph{Maxterme beschreiben alle Eingangskombinationen,
für die die Funktion den Wert 0 annimmt.}

\subsubsection{Vergleich Minterm und Maxterm}

\begin{center}
\begin{tabular}{|l|c|c|}
\hline
\textbf{Merkmal} & \textbf{Minterm} & \textbf{Maxterm} \\ \hline
Basierend auf & $y = 1$ & $y = 0$ \\ \hline
Verknüpfung & UND & ODER \\ \hline
Normalform & DNF & KNF \\ \hline
Negation & bei Eingängen mit 0 & bei Eingängen mit 1 \\ \hline
\end{tabular}
\end{center}

\subsubsection{Bedeutung für diskrete Steuerungen}

Minterme und Maxterme bilden die theoretische Grundlage für:
\begin{itemize}
    \item systematische Schaltungsentwürfe
    \item Vereinfachung logischer Funktionen
    \item Umsetzung von Schaltnetzen
    \item Analyse von Steuerungslogik
\end{itemize}

\medskip
\textbf{Zusammenfassender Merksatz:}  
\emph{Minterme führen von der Wertetabelle zur logischen Gleichung,
Maxterme von der logischen Gleichung zur Wertetabelle.}

\begin{figure}[H]
    \centering
    \includegraphics[page=16,width=0.8\textwidth]{/Users/danielweindl/_source/Repositorys/STT-Lernskript/Data/sttvo-k02-diskrete Steuerungen-4v2-Folien_221025.pdf}    
 \end{figure}

\subsection{Schaltnetze}

Schaltnetze sind \textbf{speicherlose}
Schaltsysteme.

\paragraph{Eigenschaft}
Der Ausgang eines Schaltnetzes
hängt ausschließlich von den
aktuellen Eingangswerten ab.

\[
a = f(e)
\]

\paragraph{Merkmale}
\begin{itemize}
    \item Keine Speicherwirkung
    \item Keine Zustandsabhängigkeit
    \item Rein kombinatorisches Verhalten
\end{itemize}

\paragraph{Beispiele}
\begin{itemize}
    \item Logische Verriegelungen
    \item Freigabeschaltungen
    \item Sicherheitsabfragen
\end{itemize}

\begin{figure}[H]
    \centering
    \includegraphics[page=37,width=0.8\textwidth]{/Users/danielweindl/_source/Repositorys/STT-Lernskript/Data/sttvo-k02-diskrete Steuerungen-4v2-Folien_221025.pdf}    
\end{figure}

\subsection{Funktionen}

Funktionen sind abstrahierte logische
Zusammenfassungen von Schaltnetzen.

Sie besitzen:
\begin{itemize}
    \item definierte Eingänge
    \item genau einen Ausgang
    \item keine Speicherwirkung
\end{itemize}

Funktionen beschreiben damit
eine reine logische Abbildung
von Eingang auf Ausgang.

\subsection{Schaltwerke}

Schaltwerke sind \textbf{schaltende Systeme mit Speicherwirkung}.

\paragraph{Eigenschaft}
Der Ausgang hängt von:
\begin{itemize}
    \item den aktuellen Eingängen
    \item dem internen Zustand
\end{itemize}
ab.

\[
a = f(e, s)
\]

\paragraph{Merkmale}
\begin{itemize}
    \item Speicherwirkung vorhanden
    \item Zustandsabhängiges Verhalten
    \item Grundlage für Ablaufsteuerungen
\end{itemize}

\paragraph{Typische Speicher}
\begin{itemize}
    \item Selbsthaltung
    \item Flip-Flops (RS, SR)
    \item Merker
\end{itemize}


\begin{figure}[H]
    \centering
    \includegraphics[page=48,width=0.8\textwidth]{/Users/danielweindl/_source/Repositorys/STT-Lernskript/Data/sttvo-k02-diskrete Steuerungen-4v2-Folien_221025.pdf}    
\end{figure}
\begin{figure}[H]
    \centering
    \includegraphics[page=49,width=0.8\textwidth]{/Users/danielweindl/_source/Repositorys/STT-Lernskript/Data/sttvo-k02-diskrete Steuerungen-4v2-Folien_221025.pdf}    
\end{figure}
\begin{figure}[H]
    \centering
    \includegraphics[page=50,width=0.8\textwidth]{/Users/danielweindl/_source/Repositorys/STT-Lernskript/Data/sttvo-k02-diskrete Steuerungen-4v2-Folien_221025.pdf}    
\end{figure}
\begin{figure}[H]
    \centering
    \includegraphics[page=51,width=0.8\textwidth]{/Users/danielweindl/_source/Repositorys/STT-Lernskript/Data/sttvo-k02-diskrete Steuerungen-4v2-Folien_221025.pdf}    
\end{figure}
\begin{figure}[H]
    \centering
    \includegraphics[page=52,width=0.8\textwidth]{/Users/danielweindl/_source/Repositorys/STT-Lernskript/Data/sttvo-k02-diskrete Steuerungen-4v2-Folien_221025.pdf}    
\end{figure}

\subsection{Schaltsysteme}

Ein Schaltsystem ist die
\textbf{Zusammenfassung mehrerer Schaltelemente}
zu einer funktionalen Einheit.

Schaltsysteme lassen sich grundsätzlich
in zwei Klassen einteilen:
\begin{itemize}
    \item Schaltnetze
    \item Schaltwerke
\end{itemize}

Diese Unterscheidung ist grundlegend
für das Verständnis diskreter Steuerungen.
\begin{figure}[H]
    \centering
    \includegraphics[page=5,width=0.8\textwidth]{/Users/danielweindl/_source/Repositorys/STT-Lernskript/Data/sttvo-k02-diskrete Steuerungen-4v2-Folien_221025.pdf}    
\end{figure}

\subsection{Funktionsbausteine}

Funktionsbausteine sind
\textbf{gekapselte Schaltwerke}
mit definierten Schnittstellen.

Sie kombinieren:
\begin{itemize}
    \item logische Verknüpfungen
    \item Speicherfunktionen
    \item gegebenenfalls Zeitfunktionen
\end{itemize}

\paragraph{Eigenschaften}
\begin{itemize}
    \item Mehrere Ein- und Ausgänge
    \item Interner Zustand
    \item Wiederverwendbarkeit
\end{itemize}

\paragraph{Beispiele}
\begin{itemize}
    \item Zeitglieder
    \item Zähler
    \item Speicherbausteine
\end{itemize}

\subsection{Zusammenhang der Begriffe}

\begin{center}
\begin{tabular}{|l|l|}
\hline
\textbf{Begriff} & \textbf{Charakteristik} \\ \hline
Schaltelement & Physikalisches Grundelement \\ \hline
Logische Schaltung & Logische Verknüpfung \\ \hline
Schaltnetz & Speicherloses System \\ \hline
Funktion & Abstraktes Schaltnetz \\ \hline
Schaltwerk & System mit Speicher \\ \hline
Funktionsbaustein & Gekapseltes Schaltwerk \\ \hline
\end{tabular}
\end{center}

\medskip
\textbf{Merksatz:}  
\emph{Schaltnetze kennen keinen Zustand,
Schaltwerke benötigen Speicher.}

\subsection{Zahlensysteme}

\begin{figure}[H]
    \centering
    \includegraphics[page=24,width=0.8\textwidth]{/Users/danielweindl/_source/Repositorys/STT-Lernskript/Data/sttvo-k02-diskrete Steuerungen-4v2-Folien_221025.pdf}    
\end{figure}\begin{figure}[H]
    \centering
    \includegraphics[page=26,width=0.8\textwidth]{/Users/danielweindl/_source/Repositorys/STT-Lernskript/Data/sttvo-k02-diskrete Steuerungen-4v2-Folien_221025.pdf}    
\end{figure}

\newpage
\section{Programmiersprachen}

\subsection{Allgemein}
\subsection*{Allgemeines und Norm IEC 61131-3}

Die Programmierung speicherprogrammierbarer Steuerungen (SPS) erfolgt normiert
nach der internationalen Norm \textbf{IEC 61131-3}.
Diese Norm definiert sowohl die zulässigen Programmiersprachen als auch
das Softwaremodell und grundlegende Konzepte der SPS-Programmierung.

Ziel der Norm ist eine herstellerunabhängige, strukturierte und wartbare
Automatisierungssoftware.

\subsection*{Programmiersprachen nach IEC 61131-3}

Die IEC 61131-3 definiert fünf standardisierte Programmiersprachen:

\begin{figure}[H]
    \centering
    \includegraphics[width=0.7\textwidth]{/Users/danielweindl/_source/Repositorys/STT-Lernskript/Bilder/Programmiersprachen.png}
\end{figure}

\renewcommand{\arraystretch}{1.3} % größerer Zeilenabstand
\begin{center}
\begin{tabular}{|l|c|l|}
\hline
\textbf{Deutsch} & \textbf{Abkürzung} & \textbf{English} \\ \hline
Anweisungsliste & AWL -- IL & Instruction List \\ \hline
Strukturierter Text & ST & Structured Text \\ \hline
Funktionsbausteinsprache & FBS -- FBD & Function Block Diagram \\ \hline
Kontaktplan & KOP -- LD & Ladder Diagram \\ \hline
Ablaufsprache & AS -- SFC & Sequential Function Chart \\ \hline
\end{tabular}
\end{center}

Die Sprachen lassen sich in zwei Gruppen einteilen:
\begin{itemize}
    \item \textbf{textuelle Sprachen}: AWL, ST
    \item \textbf{grafische Sprachen}: KOP, FBS, AS
\end{itemize}

\subsection*{Softwaremodell einer SPS}

Nach IEC 61131-3 ist die Software einer SPS hierarchisch aufgebaut.
Diese Hierarchie dient der Strukturierung, Wiederverwendbarkeit und Wartbarkeit.

\subsubsection*{Konfiguration}

Die Konfiguration bildet die oberste Ebene des Softwaremodells.
Sie entspricht einem vollständigen SPS-System und kann eine oder mehrere
Ressourcen enthalten.

In komplexen Anlagen können mehrere Konfigurationen existieren,
die miteinander kommunizieren.

\subsubsection*{Ressource}

Eine Ressource ist einer Konfiguration untergeordnet und umfasst typischerweise
eine CPU eines SPS-Systems.

Jede Ressource enthält:
\begin{itemize}
    \item Tasks
    \item Programm-Organisationseinheiten (POEs)
\end{itemize}

\subsubsection*{Task}

Eine Task definiert die \textbf{Abarbeitungseigenschaften} von Programmen.
Sie legt fest:
\begin{itemize}
    \item Art der Ausführung (zyklisch, ereignisgesteuert)
    \item Zykluszeit
    \item Priorität
\end{itemize}

Die Task selbst enthält keine Programmlogik, sondern steuert deren Ausführung.

\subsubsection*{Programm-Organisationseinheiten (POE)}

POEs sind die eigentlichen Träger der Programmlogik.
Die IEC 61131-3 unterscheidet drei POE-Typen:

\begin{itemize}
    \item Programme
    \item Funktionsbausteine
    \item Funktionen
\end{itemize}

\paragraph{Programme}
Programme besitzen kein eigenes Gedächtnis über mehrere Aufrufe hinaus.
Sie werden einer Task zugeordnet und zyklisch oder ereignisgesteuert ausgeführt.

\paragraph{Funktionsbausteine}
Funktionsbausteine besitzen ein internes Gedächtnis.
Sie werden über Instanzen aufgerufen und eignen sich zur Abbildung
zustandsbehafteter Funktionen.

\paragraph{Funktionen}
Funktionen besitzen kein Gedächtnis.
Sie liefern bei gleichen Eingängen immer gleiche Ausgänge
und eignen sich für Berechnungen und logische Verknüpfungen.

\subsection*{Aufrufhierarchie von POEs}

POEs können sich gegenseitig aufrufen, jedoch gelten folgende Regeln:

\begin{itemize}
    \item Rekursive Aufrufe sind nicht erlaubt
    \item Programme dürfen Funktionen und Funktionsbausteine aufrufen
    \item Funktionsbausteine dürfen Funktionen und Funktionsbausteine aufrufen
    \item Funktionen dürfen ausschließlich Funktionen aufrufen
\end{itemize}

\medskip
\textbf{Merksatz:}  
\emph{Funktionen haben kein Gedächtnis – daher dürfen sie keine
Funktionsbausteine oder Programme aufrufen.}

\subsection*{Zyklisches Abarbeitungsmodell}

Die Programmausführung einer SPS erfolgt typischerweise zyklisch:

\begin{enumerate}
    \item Einlesen der Eingänge
    \item Programmausführung
    \item Schreiben der Ausgänge
\end{enumerate}

Dieses Modell gewährleistet deterministisches Verhalten
und ist Grundlage für die zeitliche Bewertung von Steuerungsprogrammen.

\subsection*{Variablen und Sichtbarkeit}

In POEs können Variablen unterschiedlicher Sichtbarkeit verwendet werden:

\begin{itemize}
    \item Lokale Variablen: nur innerhalb der POE sichtbar
    \item Globale Variablen: im gesamten Projekt sichtbar
\end{itemize}

Globale Variablen ermöglichen die Kommunikation zwischen Programmen,
bergen jedoch die Gefahr unübersichtlicher Abhängigkeiten.

\subsection*{Zusammenfassung}

Die IEC 61131-3 stellt ein einheitliches Modell zur Verfügung, um
Steuerungsprogramme strukturiert, wartbar und normkonform zu erstellen.
Die Wahl der Programmiersprache richtet sich nach der Aufgabenstellung,
der Komplexität und den Anforderungen an Übersichtlichkeit und Wartbarkeit.

%%%%%%%%%%%%%%%%%%%%%%%%%%%%%%%%%%%%%%%%%%%%%%%%%%%%%%%%%%%%%%%%%%%%%%%%%%%%%%%%%%%%%%%%%%%%%%%

\newpage
\subsection{AWL}

\subsubsection{Grundprinzip}

Die Anweisungsliste (AWL, engl. Instruction List – IL) ist eine textuelle
Programmiersprache der IEC~61131-3.
Sie ist maschinennah aufgebaut und lehnt sich konzeptionell an die
Assemblerprogrammierung an.

AWL arbeitet nach dem Prinzip einer \textbf{Ein-Adress-Maschine}.
Alle Operationen erfolgen über ein zentrales Rechenregister,
den sogenannten \textbf{Akkumulator (AKKU)}.

\subsubsection{Ein-Adress-Maschine}

Bei einer Ein-Adress-Maschine besitzt jede Rechenoperation genau einen expliziten
Operanden.
Der zweite Operand ist implizit der Inhalt des Akkumulators.

\medskip
Arbeitsweise:
\begin{enumerate}
    \item Ein Wert wird in den Akku geladen
    \item Der Akku wird mit weiteren Operanden verarbeitet
    \item Das Ergebnis wird aus dem Akku gespeichert
\end{enumerate}

\paragraph{Beispiel}
Berechnung:
\[
a = (x + y) \cdot c
\]

AWL-Code:
\begin{verbatim}
LD   x
ADD  y
MUL  c
ST   a
\end{verbatim}

\subsubsection{Aufbau eines AWL-Programms}

Ein AWL-Programm besteht aus einer Folge von Anweisungen.
Jede Anweisung steht in einer eigenen Zeile und besteht aus:
\begin{itemize}
    \item einem Operator
    \item optional einem oder mehreren Operanden
\end{itemize}

Vor einer Anweisung kann optional eine Sprungmarke (Label) stehen.

\subsubsection{Akkumulator}

Der Akkumulator speichert stets das Ergebnis der letzten Operation.
Jede neue Operation überschreibt den vorherigen Inhalt des Akkus.

Der Datentyp des Akkumulators wird durch die ausgeführte Operation bestimmt.
Daher müssen die Datentypen aufeinanderfolgender Anweisungen kompatibel sein.

\medskip
\textbf{Wichtig:}  
Ein falscher Datentyp im Akku führt zu Laufzeit- oder Compilerfehlern.

\subsubsection{Operatoren und Operanden}

\paragraph{Operatoren}
Operatoren beschreiben die auszuführende Operation.
Typische Operatoren sind:
\begin{itemize}
    \item LD (Load)
    \item ST (Store)
    \item AND, OR, XOR
    \item ADD, SUB, MUL, DIV
    \item EQ, GT, LT
\end{itemize}

\paragraph{Operanden}
Operanden können sein:
\begin{itemize}
    \item Variablen
    \item Literale
    \item Instanznamen von Funktionsbausteinen
\end{itemize}

\subsubsection{Modifier in AWL}

Operatoren können durch sogenannte Modifier erweitert werden.

\paragraph{Negationsmodifier N}
Der Modifier \texttt{N} negiert den Operanden vor der Ausführung der Operation.

\begin{verbatim}
LD   Var1
ANDN Var2
ST   Var3
\end{verbatim}

Dies entspricht logisch:
\[
Var3 := Var1 \land \lnot Var2
\]

\subsubsection{Klammerung und Schachtelung}

AWL erlaubt die Klammerung von Anweisungsfolgen.
Der geklammerte Ausdruck wird zuerst ausgewertet und anschließend
mit dem vorherigen Akkumulatorwert verknüpft.

\begin{verbatim}
LD   Var1
AND(
     Var2
     OR Var3
   )
ST   Var4
\end{verbatim}

Ergebnis:
\[
Var4 := Var1 \land (Var2 \lor Var3)
\]

Klammern können geschachtelt werden, müssen jedoch stets
einen gültigen Datentyp liefern.

\subsubsection{Bedingte Ausführung}

AWL ermöglicht die bedingte Ausführung von Anweisungen.
Hierzu wird das aktuelle Ergebnis im Akku als Bedingung verwendet.

\paragraph{Modifier C}
Mit dem Modifier \texttt{C} wird eine Anweisung nur ausgeführt,
wenn der Akkumulator den Wert TRUE enthält.

\begin{verbatim}
LD    Var1
GT    10
JMPC  label
LD    20
ST    Var2
label:
\end{verbatim}

Ist \texttt{Var1 > 10}, wird der Sprung ausgeführt und
die Zuweisung an \texttt{Var2} übersprungen.

\subsubsection{Sprünge und Marken}

AWL unterstützt bedingte und unbedingte Sprünge.
Sprungziele werden durch Marken (Labels) definiert.

\begin{itemize}
    \item JMP  – unbedingter Sprung
    \item JMPC – bedingter Sprung (TRUE)
    \item JMPCN – bedingter Sprung (FALSE)
\end{itemize}

Sprünge ermöglichen einfache Ablaufsteuerungen,
ersetzen jedoch keine strukturierte Ablaufbeschreibung.

\subsubsection{Kommentare}

Kommentare dienen ausschließlich der Dokumentation
und haben keinen Einfluss auf die Programmausführung.

\begin{itemize}
    \item Mehrzeilige Kommentare: \texttt{(* Kommentar *)}
    \item Zeilenkommentare (herstellerspezifisch): \texttt{// Kommentar}
\end{itemize}

In AWL sind Kommentare besonders wichtig,
da die Lesbarkeit geringer ist als bei grafischen Sprachen.

\subsubsection{Vor- und Nachteile von AWL}

\paragraph{Vorteile}
\begin{itemize}
    \item Maschinennah und effizient
    \item Gute Kontrolle über den Programmablauf
    \item Binäre Logik sehr einfach abbildbar
    \item Analogwertverarbeitung einfacher als in KOP
\end{itemize}

\paragraph{Nachteile}
\begin{itemize}
    \item Geringe Übersichtlichkeit
    \item Schleifen und Verzweigungen nur umständlich realisierbar
    \item Hoher Dokumentationsaufwand notwendig
\end{itemize}

\subsubsection{Einordnung}

AWL ist besonders geeignet für:
\begin{itemize}
    \item zeitkritische Programmteile
    \item maschinennahe Steuerungslogik
    \item Analyse und Optimierung bestehender Programme
\end{itemize}

In modernen Projekten wird AWL häufig durch ST oder grafische Sprachen ergänzt.

\newpage
\subsection{ST}

\subsubsection{Grundprinzip}

Strukturierter Text (ST, engl. Structured Text) ist eine textuelle
Hochsprache nach IEC~61131-3.
Sie ähnelt klassischen Programmiersprachen wie Pascal oder C
und eignet sich besonders für komplexe Algorithmen,
Berechnungen und strukturierte Ablaufbeschreibungen.

ST ist deterministisch und zyklisch in das SPS-Abarbeitungsmodell eingebettet.

\begin{figure}[H]
    \centering
    \includegraphics[page=114,width=0.8\textwidth]{/Users/danielweindl/_source/Repositorys/STT-Lernskript/Data/sttvo-k03-Programmiersprachen-4v5-Folien.pdf}    
 \end{figure}


\subsubsection{Aufbau und Syntax}

Ein ST-Programm besteht aus Anweisungen, die sequenziell abgearbeitet werden.
Die Syntax ist blockorientiert und verwendet Schlüsselwörter zur
Strukturierung des Programms.

Beispiel:
\begin{verbatim}
IF a > b THEN
    c := a;
ELSE
    c := b;
END_IF;
\end{verbatim}

\subsubsection{Datentypen}

ST unterstützt alle in IEC~61131-3 definierten Datentypen.

\paragraph{Elementare Datentypen}
\begin{itemize}
    \item BOOL
    \item INT, DINT
    \item REAL
    \item TIME
\end{itemize}

\paragraph{Zusammengesetzte Datentypen}
\begin{itemize}
    \item ARRAY
    \item STRUCT
    \item ENUM (Aufzählungstypen)
\end{itemize}

\paragraph{Beispiel: ARRAY}
\begin{verbatim}
TYPE
    messwert : ARRAY[1..50] OF REAL;
END_TYPE
\end{verbatim}

\paragraph{Beispiel: STRUCT}
\begin{verbatim}
TYPE
    ventil : STRUCT
        vorhanden : BOOL;
        ausschuss : BOOL;
        zylinder  : BOOL;
        farbe     : INT;
    END_STRUCT;
END_TYPE
\end{verbatim}

\subsubsection{Typkonversion}

Bei gemischter Verwendung von Datentypen ist eine explizite
Typkonversion erforderlich.

\begin{verbatim}
VAR
    i : INT;
    r : REAL;
END_VAR

r := INT_TO_REAL(i);
i := TRUNC(r);
\end{verbatim}

\subsubsection{Kontrollstrukturen}

\paragraph{IF / ELSIF / ELSE}
\begin{verbatim}
IF x = 0 THEN
    y := 0;
ELSIF x > 0 THEN
    y := 1;
ELSE
    y := -1;
END_IF;
\end{verbatim}
\begin{figure}[H]
    \centering
    \includegraphics[page=123,width=0.8\textwidth]{/Users/danielweindl/_source/Repositorys/STT-Lernskript/Data/sttvo-k03-Programmiersprachen-4v5-Folien.pdf}    
 \end{figure}

\paragraph{CASE}
\begin{verbatim}
CASE state OF
    0: y := 0;
    1: y := 10;
    2: y := 20;
ELSE
    y := -1;
END_CASE;
\end{verbatim}
\begin{figure}[H]
    \centering
    \includegraphics[page=125,width=0.8\textwidth]{/Users/danielweindl/_source/Repositorys/STT-Lernskript/Data/sttvo-k03-Programmiersprachen-4v5-Folien.pdf}    
 \end{figure}
\begin{figure}[H]
    \centering
    \includegraphics[page=126,width=0.8\textwidth]{/Users/danielweindl/_source/Repositorys/STT-Lernskript/Data/sttvo-k03-Programmiersprachen-4v5-Folien.pdf}    
 \end{figure}

\subsubsection{Schleifen}

\paragraph{WHILE}
\begin{verbatim}
WHILE x < 10 DO
    x := x + 1;
END_WHILE;
\end{verbatim}

\paragraph{REPEAT}
\begin{verbatim}
REPEAT
    x := x + 1;
UNTIL x >= 10
END_REPEAT;
\end{verbatim}

\paragraph{FOR-Schleife}

Die FOR-Schleife ist die wichtigste Schleifenform in ST
und in den Folien detailliert behandelt.

\begin{verbatim}
FOR cnt := startVal TO endVal BY stepVal DO
    (* Anweisungen *)
END_FOR;
\end{verbatim}
\begin{figure}[H]
    \centering
    \includegraphics[page=129,width=0.8\textwidth]{/Users/danielweindl/_source/Repositorys/STT-Lernskript/Data/sttvo-k03-Programmiersprachen-4v5-Folien.pdf}    
 \end{figure}
Hinweis: Bei Schleifen mit unbestimmter Laufzeit (WHILE, REPEAT)
sollte ein Zähler mit EXIT-Abbruch eingebaut werden, damit es im
Extremfall zu keiner Zykluszeitverletzung kommt, sondern der Fehler
programmatisch abgefangen werden kann.
\subsubsection{Abarbeitungsverhalten der FOR-Schleife}

\begin{itemize}
    \item Wird \texttt{startVal > endVal} bei positivem \texttt{stepVal},  
    wird die Schleife \textbf{nicht ausgeführt}.
    \item Bei \texttt{startVal = endVal} wird die Schleife \textbf{genau einmal} ausgeführt.
    \item Der Abbruch erfolgt bei \textbf{größer} bzw. \textbf{kleiner},
    nicht bei größer/kleiner gleich.
    \item Der Zählerwert nach Verlassen der Schleife ist
    \texttt{letzter Wert + stepVal}.
    \item Änderungen von \texttt{startVal}, \texttt{endVal} oder \texttt{stepVal}
    innerhalb der Schleife werden berücksichtigt.
    \item \texttt{stepVal = 0} führt zu einer Endlosschleife
    und damit zu einer Zykluszeitverletzung.
    \item Der Schleifenzähler muss ein ganzzahliger Datentyp sein.
\end{itemize}

\subsubsection{EXIT}

Mit \texttt{EXIT} kann eine Schleife vorzeitig verlassen werden.

\begin{verbatim}
FOR i := 1 TO 10 DO
    IF i = 5 THEN
        EXIT;
    END_IF;
END_FOR;
\end{verbatim}
\begin{figure}[H]
    \centering
    \includegraphics[page=127,width=0.8\textwidth]{/Users/danielweindl/_source/Repositorys/STT-Lernskript/Data/sttvo-k03-Programmiersprachen-4v5-Folien.pdf}    
 \end{figure}
\begin{figure}[H]
    \centering
    \includegraphics[page=131,width=0.8\textwidth]{/Users/danielweindl/_source/Repositorys/STT-Lernskript/Data/sttvo-k03-Programmiersprachen-4v5-Folien.pdf}    
 \end{figure}

\subsubsection{Grundstrukturen}
\begin{figure}[H]
    \centering
    \includegraphics[page=134,width=0.8\textwidth]{/Users/danielweindl/_source/Repositorys/STT-Lernskript/Data/sttvo-k03-Programmiersprachen-4v5-Folien.pdf}    
 \end{figure}
\begin{figure}[H]
    \centering
    \includegraphics[page=135,width=0.8\textwidth]{/Users/danielweindl/_source/Repositorys/STT-Lernskript/Data/sttvo-k03-Programmiersprachen-4v5-Folien.pdf}    
 \end{figure}
\begin{figure}[H]
    \centering
    \includegraphics[page=136,width=0.8\textwidth]{/Users/danielweindl/_source/Repositorys/STT-Lernskript/Data/sttvo-k03-Programmiersprachen-4v5-Folien.pdf}    
 \end{figure}


\subsubsection{Determinismus und Laufzeit}

ST-Anweisungen werden innerhalb eines SPS-Zyklus vollständig abgearbeitet.
Endlosschleifen oder sehr große Schleifen können
eine \textbf{Zykluszeitverletzung} verursachen.

\subsubsection{Vor- und Nachteile von ST}

\paragraph{Vorteile}
\begin{itemize}
    \item Sehr gut lesbar und strukturiert
    \item Mächtige Kontrollstrukturen
    \item Ideal für Berechnungen und Algorithmen
    \item Gute Wartbarkeit
\end{itemize}

\paragraph{Nachteile}
\begin{itemize}
    \item Weniger anschaulich für binäre Logik
    \item Fehler durch Endlosschleifen möglich
\end{itemize}

\subsubsection{Einordnung}

ST ist besonders geeignet für:
\begin{itemize}
    \item komplexe Ablauf- und Rechenalgorithmen
    \item Datenverarbeitung
    \item Umsetzung von Hochsprachenkonstrukten
\end{itemize}

In der Praxis wird ST häufig mit grafischen Sprachen kombiniert.

\newpage
\subsection{FBS}

\subsubsection{Grundprinzip}

Die Funktionsbausteinsprache (FBS, engl. Function Block Diagram – FBD)
ist eine grafische Programmiersprache nach IEC~61131-3.
Sie basiert auf der Darstellung von Funktionen und Funktionsbausteinen
als grafische Blöcke, die über Signalverbindungen miteinander verknüpft sind.

Im Mittelpunkt steht der \textbf{Signalfluss} zwischen Ein- und Ausgängen.

\subsubsection{Signalflussorientierung}

In FBS wird die Programmlogik durch die Verbindung von Ausgängen
zu Eingängen anderer Bausteine beschrieben.

\begin{itemize}
    \item Daten fließen von links nach rechts
    \item Ein Ausgang kann mehrere Eingänge speisen
    \item Der Signalfluss bestimmt die logische Abhängigkeit
\end{itemize}

Die zeitliche Ausführung erfolgt dennoch innerhalb des SPS-Zyklus
und wird durch die zugeordnete Task bestimmt.

\subsubsection{Bausteintypen in FBS}

In FBS können folgende Bausteintypen verwendet werden:

\begin{itemize}
    \item Funktionen (stateless)
    \item Funktionsbausteine (mit Gedächtnis)
    \item Erweiterbare Funktionen
\end{itemize}

\paragraph{Funktionen}
Funktionen besitzen kein internes Gedächtnis.
Bei gleichen Eingängen liefern sie immer gleiche Ausgänge.

\paragraph{Funktionsbausteine}
Funktionsbausteine besitzen ein internes Gedächtnis.
Sie benötigen eine Instanzvariable und sind für
zustandsbehaftete Aufgaben geeignet (z.\,B. Timer, Zähler).

\begin{figure}[H]
    \centering
    \includegraphics[page=95,width=0.8\textwidth]{/Users/danielweindl/_source/Repositorys/STT-Lernskript/Data/sttvo-k03-Programmiersprachen-4v5-Folien.pdf}    
 \end{figure}


\subsubsection{Bausteininstanzen}

Jeder Funktionsbaustein wird über eine Instanz aufgerufen.
Die Instanz speichert die internen Zustände des Bausteins
über mehrere SPS-Zyklen hinweg.

\begin{itemize}
    \item Eine Instanz pro Funktionsbaustein-Aufruf
    \item Mehrere Instanzen desselben Bausteintyps möglich
\end{itemize}

Ohne Instanz ist keine Speicherwirkung möglich.

\subsubsection{Abarbeitungsreihenfolge}
\begin{figure}[H]
    \centering
    \includegraphics[page=97,width=0.8\textwidth]{/Users/danielweindl/_source/Repositorys/STT-Lernskript/Data/sttvo-k03-Programmiersprachen-4v5-Folien.pdf}    
 \end{figure}



\subsubsection{EN/ENO-Logik}

Viele Funktionsbausteine und Funktionen besitzen optionale
Enable-Eingänge (EN) und Enable-Ausgänge (ENO).

\begin{itemize}
    \item EN = FALSE: Baustein wird nicht ausgeführt
    \item EN = TRUE: Baustein wird ausgeführt
    \item ENO zeigt an, ob der Baustein aktiv war
\end{itemize}

Ist EN FALSE, bleiben die Ausgänge des Bausteins unverändert.
Dies ist insbesondere bei der Weiterverschaltung zu beachten.

\subsubsection{Weiterverschaltung von Bausteinen}

\begin{itemize}
    \item Bausteine ohne EN/ENO können direkt weiter verschaltet werden
    \item Bausteine mit aktiviertem EN/ENO dürfen
    ausgangsseitig nicht direkt weiter verschaltet werden
    \item Zwischenspeicherung über Variablen ist erforderlich
\end{itemize}

Diese Einschränkung dient der Vermeidung undefinierter Zustände,
wenn Bausteine deaktiviert sind.

\subsubsection{Analoge Signalverarbeitung}

FBS eignet sich besonders gut zur Verarbeitung analoger Werte.

\begin{itemize}
    \item Analoge Werte werden über Linien dargestellt
    \item Keine analogen Kontakte wie in KOP
    \item Verbindung nur über Funktions- oder FB-Ein-/Ausgänge
\end{itemize}

Typische Anwendungen:
\begin{itemize}
    \item Skalierung
    \item Vergleich
    \item Berechnung
    \item Filterung
\end{itemize}

\subsubsection{Erweiterbare Funktionen}

Die IEC~61131-3 definiert erweiterbare Funktionen,
die mehr als zwei Eingänge besitzen können.

Beispiele:
\begin{itemize}
    \item ADD
    \item MUL
    \item AND
    \item OR
    \item MAX / MIN
\end{itemize}

Nicht beschaltete Eingänge sind nicht zulässig.

\subsubsection{Programmflusssteuerung}
\begin{figure}[H]
    \centering
    \includegraphics[page=102,width=0.8\textwidth]{/Users/danielweindl/_source/Repositorys/STT-Lernskript/Data/sttvo-k03-Programmiersprachen-4v5-Folien.pdf}    
 \end{figure}
\begin{figure}[H]
    \centering
    \includegraphics[page=103,width=0.8\textwidth]{/Users/danielweindl/_source/Repositorys/STT-Lernskript/Data/sttvo-k03-Programmiersprachen-4v5-Folien.pdf}    
 \end{figure}

\subsubsection{Vergleich FBS zu KOP und ST}

\begin{center}
\begin{tabular}{|l|c|c|c|}
\hline
\textbf{Merkmal} & \textbf{FBS} & \textbf{KOP} & \textbf{ST} \\ \hline
Darstellung & grafisch & grafisch & textuell \\ \hline
Binäre Logik & gut & sehr gut & gut \\ \hline
Analoge Werte & sehr gut & eingeschränkt & sehr gut \\ \hline
Übersichtlichkeit & hoch & hoch & mittel \\ \hline
Komplexe Algorithmen & eingeschränkt & schlecht & sehr gut \\ \hline
\end{tabular}
\end{center}

\subsubsection{Vor- und Nachteile von FBS}

\paragraph{Vorteile}
\begin{itemize}
    \item Sehr anschauliche Darstellung
    \item Ideal für Signalverarbeitung
    \item Gute Diagnosemöglichkeiten
\end{itemize}

\paragraph{Nachteile}
\begin{itemize}
    \item Bei großen Programmen schnell unübersichtlich
    \item Schleifen und komplexe Abläufe schwer darstellbar
\end{itemize}

\subsubsection{Einordnung}

FBS eignet sich besonders für:
\begin{itemize}
    \item Signalflussorientierte Aufgaben
    \item Analogwertverarbeitung
    \item Kombination mit KOP und ST
\end{itemize}

\newpage
\subsection{KOP}

\subsubsection{Grundprinzip}

Der Kontaktplan (KOP, engl. Ladder Diagram – LD) ist eine grafische
Programmiersprache nach IEC~61131-3.
Er lehnt sich an die Darstellung klassischer Stromlaufpläne an
und kann als um 90° gegen den Uhrzeigersinn gedrehter Stromlaufplan
interpretiert werden.

Eine KOP-Zeile (horizontal) entspricht einem Strompfad (vertikal)
im elektrischen Stromlaufplan.

\begin{itemize}
    \item Sehr gut lesbar für Elektrotechniker
    \item Besonders geeignet für binäre Logik
    \item Weit verbreitet im industriellen Umfeld
\end{itemize}

\subsubsection{Virtueller Stromfluss}

Im KOP existiert kein realer Strom, sondern ein \textbf{virtueller Stromfluss}.
Dieser fließt:
\begin{itemize}
    \item immer von links nach rechts
    \item innerhalb eines Netzwerks
\end{itemize}

Der virtuelle Stromfluss repräsentiert den logischen Zustand TRUE.

\subsubsection{Struktur eines KOP-Netzwerks}

Ein KOP-Programm besteht aus mehreren Netzwerken.
Jedes Netzwerk wird separat ausgewertet.

\paragraph{Linke Stromschiene}
Die linke Stromschiene liefert den logischen Wert TRUE
und ist Startpunkt jedes Netzwerks.

\paragraph{Rechte Stromschiene}
Die rechte Stromschiene schließt das Netzwerk ab.

\paragraph{Horizontale Linien}
Horizontale Linien übertragen den logischen Zustand
von einem Element zum nächsten.

\paragraph{Vertikale Linien}
Vertikale Linien verbinden mehrere horizontale Linien
und implizieren eine logische ODER-Verknüpfung.

\subsubsection{Berechnungsteil und Zuweisungsteil}

Ein KOP-Netzwerk lässt sich logisch in zwei Bereiche gliedern:
\begin{itemize}
    \item \textbf{Berechnungsteil (links)}:  
    Kontakte lesen Variablen und verknüpfen sie logisch.
    
    \item \textbf{Zuweisungsteil (rechts)}:  
    Spulen weisen den berechneten Wert einer Variablen zu.
\end{itemize}

\subsubsection{Kontakte}

Kontakte lesen den Wert einer Variablen
und beeinflussen den virtuellen Stromfluss.
Der Wert der Variablen wird durch Kontakte \textbf{nicht verändert}.

\paragraph{Schließer (NO)}
Der Kontakt leitet Strom, wenn die Variable TRUE ist.

Logisch:
\[
Rechts := Links \land Var
\]

\paragraph{Öffner (NC)}
Der Kontakt invertiert den Variablenwert.

Logisch:
\[
Rechts := Links \land \lnot Var
\]

\subsubsection{Flankenkontakte}

Flankenkontakte liefern TRUE nur für einen SPS-Zyklus
bei einer Zustandsänderung der Variable.

\paragraph{Positive Flanke}
TRUE bei Übergang von FALSE auf TRUE.

\paragraph{Negative Flanke}
TRUE bei Übergang von TRUE auf FALSE.

\paragraph{Positive und negative Flanke}
TRUE bei jeder Zustandsänderung.

Flankenkontakte werden häufig zum Zählen,
Setzen oder Rücksetzen verwendet.

\subsubsection{Spulen}

Spulen weisen den Wert des virtuellen Stromflusses
einer Variablen zu.

\paragraph{Normale Spule}
\[
Var := Links
\]

\paragraph{Negierte Spule}
\[
Var := \lnot Links
\]

\subsubsection{Setz- und Rücksetzspulen}

\paragraph{Setzspule (S)}
Setzt eine Variable auf TRUE,
solange der virtuelle Stromfluss anliegt.
Der Zustand bleibt gespeichert,
bis eine Rücksetzspule aktiv wird.

\paragraph{Rücksetzspule (R)}
Setzt eine Variable auf FALSE,
solange der virtuelle Stromfluss anliegt.

Setz- und Rücksetzspulen realisieren Speicherwirkung
und sind Grundlage für Ablaufsteuerungen.

\subsubsection{Flankenspulen}

\paragraph{Positive Flankenspule}
Setzt die Variable für genau einen Zyklus auf TRUE,
wenn eine positive Flanke am Eingang auftritt.

\paragraph{Negative Flankenspule}
Setzt die Variable für genau einen Zyklus auf TRUE,
wenn eine negative Flanke am Eingang auftritt.

\paragraph{Beide Flanken}
Reagiert auf jede Zustandsänderung.

\subsubsection{Ausführungsreihenfolge}

\begin{itemize}
    \item Der Signalfluss innerhalb eines Netzwerks erfolgt von links nach rechts
    \item Netzwerke werden von oben nach unten abgearbeitet
    \item Kleinere Netzwerknummern vor größeren
\end{itemize}

Explizite Rückkopplungen innerhalb eines Netzwerks
sind nicht erlaubt und werden als Kurzschluss abgelehnt.

Implizite Rückkopplungen sind über Variablen
im nächsten SPS-Zyklus möglich.

\subsubsection{Programmflusssteuerung}

KOP erlaubt die Beeinflussung der Programmausführung durch:
\begin{itemize}
    \item Bedingte Sprünge
    \item Unbedingte Sprünge
    \item RETURN (Verlassen der POE)
\end{itemize}

Sprünge können auf Netzwerknummern oder Labels erfolgen.

\subsubsection{Verwendung von Funktionen und Funktionsbausteinen}

In KOP können Funktionen und Funktionsbausteine verwendet werden.

\begin{itemize}
    \item Funktionsbausteine benötigen Instanzen
    \item Funktionen besitzen kein Gedächtnis
    \item Analoge Werte nur über Block-Ein- und Ausgänge
\end{itemize}

Analoge Werte können nicht über Kontakte verarbeitet werden.

\subsubsection{Vor- und Nachteile von KOP}

\paragraph{Vorteile}
\begin{itemize}
    \item Sehr übersichtlich für binäre Logik
    \item Gute Diagnose durch Anzeige des Stromflusses
    \item Einfache Übertragung von Relaissteuerungen
\end{itemize}

\paragraph{Nachteile}
\begin{itemize}
    \item Numerische Logik nur über Bausteine
    \item Zeichen- und Stringverarbeitung umständlich
    \item Komplexe Algorithmen schwer darstellbar
\end{itemize}

\subsubsection{Einordnung}

KOP eignet sich besonders für:
\begin{itemize}
    \item Binäre Steuerungslogik
    \item Verriegelungen
    \item Klassische Maschinensteuerungen
\end{itemize}
\newpage
\subsection{AS}

\subsubsection{Grundprinzip}

Die Ablaufsprache (AS, engl. Sequential Function Chart – SFC)
ist eine grafische Programmiersprache nach IEC~61131-3
zur Beschreibung \textbf{sequentieller Abläufe}.
Sie eignet sich besonders zur Darstellung von Ablaufsteuerungen
mit klar definierten Zuständen und Übergängen.

SFC basiert auf dem Zustandsmodell und trennt explizit zwischen:
\begin{itemize}
    \item Schritten (Zustände)
    \item Transitionen (Übergangsbedingungen)
    \item Aktionen (auszuführende Operationen)
\end{itemize}

\subsubsection{Schritte}

Ein Schritt repräsentiert einen Zustand des Steuerungsablaufs.
Zu jedem Zeitpunkt ist mindestens ein Schritt aktiv.

\begin{itemize}
    \item Schritte werden als Rechtecke dargestellt
    \item Aktive Schritte sind logisch TRUE
    \item Ein Initialschritt ist beim Start aktiv
\end{itemize}

Ein Schritt selbst führt keine Logik aus,
sondern aktiviert zugeordnete Aktionen.

\subsubsection{Transitionen}

Transitionen beschreiben die Bedingung für den Übergang
von einem oder mehreren Schritten zu einem oder mehreren Folgeschritten.

\begin{itemize}
    \item Transitionen werden logisch ausgewertet
    \item Sie besitzen ausschließlich BOOL-Ergebnisse
    \item Ein Übergang erfolgt nur, wenn alle vorgeschalteten Schritte aktiv sind
\end{itemize}

\medskip
\textbf{Regel:}  
Ein Schritt wird verlassen, wenn die zugehörige Transition TRUE wird.

\subsubsection{Aktivierungs- und Deaktivierungsregeln}

\begin{itemize}
    \item Wird eine Transition TRUE, werden die vorherigen Schritte deaktiviert
    \item Die nachfolgenden Schritte werden gleichzeitig aktiviert
    \item Mehrere Schritte können parallel aktiv sein
\end{itemize}

SFC erlaubt damit auch parallele Abläufe.

\subsubsection{Aktionen}

Aktionen sind die eigentlichen Programmanweisungen
und werden einem Schritt zugeordnet.
Sie können in KOP, FBS, ST oder AWL implementiert sein.

Eine Aktion wird ausgeführt, solange ihr Schritt aktiv ist
und die Aktionsbedingung erfüllt ist.

\subsubsection{Aktionsblöcke}

Aktionen werden in Aktionsblöcken dargestellt.
Ein Aktionsblock besteht aus:
\begin{itemize}
    \item einem Bestimmungszeichen (Qualifier)
    \item optional einer Zeitangabe
    \item dem Aktionsnamen oder einer BOOL-Variablen
\end{itemize}

\subsubsection{Aktionsqualifier}

Der Qualifier bestimmt das zeitliche Verhalten einer Aktion.

\begin{itemize}
    \item \textbf{N} (Normal):  
    Aktion ist aktiv, solange der Schritt aktiv ist
    
    \item \textbf{S} (Set):  
    Aktion wird beim Aktivieren des Schrittes gesetzt
    und bleibt aktiv, bis sie explizit zurückgesetzt wird
    
    \item \textbf{R} (Reset):  
    Setzt eine zuvor gesetzte Aktion zurück
    
    \item \textbf{L} (Limit):  
    Aktion ist nur für eine begrenzte Zeit aktiv
    
    \item \textbf{D} (Delay):  
    Aktion wird zeitverzögert aktiviert
\end{itemize}

Zeitqualifier verwenden entweder konstante Zeiten
oder Variablen vom Typ \texttt{TIME}.

\subsubsection{Parallele und alternative Abläufe}

SFC unterstützt:
\begin{itemize}
    \item alternative Verzweigungen (ODER-Verzweigung)
    \item parallele Verzweigungen (UND-Verzweigung)
\end{itemize}

Bei parallelen Abläufen müssen alle parallelen Zweige
abgeschlossen sein, bevor der Ablauf fortgesetzt wird.

\subsubsection{Einbindung in das SPS-Zyklusmodell}

\begin{itemize}
    \item Schritte und Transitionen werden zyklisch ausgewertet
    \item Aktionen werden innerhalb des SPS-Zyklus ausgeführt
    \item Zeitfunktionen basieren auf der Zykluszeit
\end{itemize}

SFC selbst beschreibt nur die Ablaufstruktur,
nicht die konkrete Ausführungslogik der Aktionen.

\subsubsection{Vor- und Nachteile von SFC}

\paragraph{Vorteile}
\begin{itemize}
    \item Sehr übersichtliche Darstellung komplexer Abläufe
    \item Klare Trennung von Zustand und Aktion
    \item Sehr gut geeignet für Ablaufsteuerungen
\end{itemize}

\paragraph{Nachteile}
\begin{itemize}
    \item Für reine Logik ungeeignet
    \item Zusätzlicher Implementierungsaufwand für Aktionen
\end{itemize}

\subsubsection{Einordnung}

SFC eignet sich besonders für:
\begin{itemize}
    \item Schrittketten
    \item Ablauf- und Prozesssteuerungen
    \item Strukturierung komplexer Programme
\end{itemize}

In der Praxis wird SFC häufig mit KOP, FBS oder ST kombiniert,
wobei SFC die Ablaufstruktur vorgibt
und die Aktionen in anderen Sprachen implementiert werden.

%%%%%%%%%%%%%%%%%%%%%%%%%%%%%%%%%%%%%%%%%%%%%%%%%%%%%%%%%%%%%%%%%%%%%%%%%%%%%%%%%%%%%%%%%%%%%%%

\subsection{Vergleich der Programmiersprachen anhand von Hochsprachenstrukturen}

In diesem Abschnitt werden typische Hochsprachenstrukturen
systematisch den IEC~61131-3-Sprachen gegenübergestellt.
Dadurch wird klar ersichtlich, welche Sprachmittel sich für
welche Aufgaben eignen.

\subsubsection{IF -- THEN -- ELSE -- ELSIF}

\paragraph{Struktur}
\begin{verbatim}
IF cond1 THEN
    x := 1;
ELSIF cond2 THEN
    x := 2;
ELSE
    x := 0;
END_IF;
\end{verbatim}

\paragraph{ST}
\begin{verbatim}
IF cond1 THEN
    x := 1;
ELSIF cond2 THEN
    x := 2;
ELSE
    x := 0;
END_IF;
\end{verbatim}

\paragraph{AWL}
\begin{verbatim}
LD   cond1
JMPC L1
LD   cond2
JMPC L2
LD   0
ST   x
JMP  L_END
L1: LD 1
    ST x
    JMP L_END
L2: LD 2
    ST x
L_END:
\end{verbatim}

\paragraph{KOP}
\begin{itemize}
    \item Mehrere Netzwerke
    \item Parallele Zweige mit Verriegelung
    \item ELSE-Zweig über negierte Bedingungen
\end{itemize}

\paragraph{FBS}
\begin{itemize}
    \item Vergleichsblöcke (GT, EQ, LT)
    \item MUX- oder SELECT-Baustein
\end{itemize}

\paragraph{SFC}
\begin{itemize}
    \item Alternative Transitionen
    \item Je ein Schritt pro IF-Zweig
\end{itemize}

---

\subsubsection{CASE}

\paragraph{Struktur}
\begin{verbatim}
CASE state OF
    0: x := 0;
    1: x := 10;
    2: x := 20;
ELSE
    x := -1;
END_CASE;
\end{verbatim}

\paragraph{ST}
\begin{verbatim}
CASE state OF
    0: x := 0;
    1: x := 10;
    2: x := 20;
ELSE
    x := -1;
END_CASE;
\end{verbatim}

\paragraph{AWL}
\begin{verbatim}
LD state
EQ 0
JMPC S0
LD state
EQ 1
JMPC S1
LD state
EQ 2
JMPC S2
LD -1
ST x
JMP END
S0: LD 0
    ST x
    JMP END
S1: LD 10
    ST x
    JMP END
S2: LD 20
    ST x
END:
\end{verbatim}

\paragraph{KOP}
\begin{itemize}
    \item Parallele Vergleichsnetzwerke
    \item Gegenseitige Verriegelung notwendig
\end{itemize}

\paragraph{FBS}
\begin{itemize}
    \item MUX-Baustein
    \item state als Auswahlvariable
\end{itemize}

\paragraph{SFC}
\begin{itemize}
    \item Zustand entspricht direkt CASE-Wert
\end{itemize}

---

\subsubsection{WHILE -- DO}

\paragraph{Struktur}
\begin{verbatim}
WHILE x < 10 DO
    x := x + 1;
END_WHILE;
\end{verbatim}

\paragraph{ST}
\begin{verbatim}
WHILE x < 10 DO
    x := x + 1;
END_WHILE;
\end{verbatim}

\paragraph{AWL}
\begin{verbatim}
LOOP:
LD x
LT 10
JMPC BODY
JMP END
BODY:
LD x
ADD 1
ST x
JMP LOOP
END:
\end{verbatim}

\paragraph{KOP}
\begin{itemize}
    \item Nicht direkt möglich
    \item Nur über Zustandsmerker und Sprünge
\end{itemize}

\paragraph{FBS}
\begin{itemize}
    \item Nicht vorgesehen
    \item Zyklische Wiederholung über Rückkopplung
\end{itemize}

\paragraph{SFC}
\begin{itemize}
    \item Schleife über Rücksprung-Transition
\end{itemize}

---

\subsubsection{REPEAT -- UNTIL}

\paragraph{Struktur}
\begin{verbatim}
REPEAT
    x := x + 1;
UNTIL x >= 10
END_REPEAT;
\end{verbatim}

\paragraph{ST}
\begin{verbatim}
REPEAT
    x := x + 1;
UNTIL x >= 10
END_REPEAT;
\end{verbatim}

\paragraph{AWL}
\begin{verbatim}
LOOP:
LD x
ADD 1
ST x
LD x
GE 10
JMPC END
JMP LOOP
END:
\end{verbatim}

\paragraph{KOP / FBS}
\begin{itemize}
    \item Nur über Zustandslogik realisierbar
\end{itemize}

\paragraph{SFC}
\begin{itemize}
    \item Schritt bleibt aktiv bis Transition TRUE wird
\end{itemize}

---

\subsubsection{FOR}

\paragraph{Struktur}
\begin{verbatim}
FOR i := 1 TO 5 DO
    sum := sum + i;
END_FOR;
\end{verbatim}

\paragraph{ST}
\begin{verbatim}
FOR i := 1 TO 5 DO
    sum := sum + i;
END_FOR;
\end{verbatim}

\paragraph{AWL}
\begin{verbatim}
LD 1
ST i
LD 0
ST sum
LOOP:
LD sum
ADD i
ST sum
LD i
ADD 1
ST i
LD i
LE 5
JMPC LOOP
\end{verbatim}

\paragraph{KOP}
\begin{itemize}
    \item Sehr aufwendig
    \item Zähler + Vergleich + Rücksprung
\end{itemize}

\paragraph{FBS}
\begin{itemize}
    \item Zählerbaustein + Rückkopplung
\end{itemize}

\paragraph{SFC}
\begin{itemize}
    \item Schleifenstruktur mit Zählvariable
\end{itemize}

---

\subsubsection{Zusammenfassender Vergleich}

\begin{center}
\begin{tabular}{|l|c|c|c|c|c|}
\hline
\textbf{Struktur} & \textbf{ST} & \textbf{AWL} & \textbf{KOP} & \textbf{FBS} & \textbf{SFC} \\ \hline
IF / ELSE & sehr gut & möglich & umständlich & gut & gut \\ \hline
CASE & sehr gut & aufwendig & schlecht & sehr gut & sehr gut \\ \hline
WHILE & gut & möglich & nein & nein & gut \\ \hline
REPEAT & gut & möglich & nein & nein & gut \\ \hline
FOR & sehr gut & möglich & schlecht & schlecht & gut \\ \hline
\end{tabular}
\end{center}

\medskip
\textbf{Merksatz:}  
\emph{ST ist die einzige Sprache, die Hochsprachenstrukturen direkt und sauber abbildet.
KOP und FBS benötigen Zustands- und Speicherlogik, SFC bildet Abläufe strukturell ab.}

%%%%%%%%%%%%%%%%%%%%%%%%%%%%%%%%%%%%%%%%%%%%%%%%%%%%%%%%%%%%%%%%%%%%%%%%%%%%%%%%%%%%%%%%%%%%%%%

\subsection{Umsetzungsmuster}

In diesem Abschnitt werden typische Hochsprachenkonzepte
auf ihre praktische Umsetzung in den IEC~61131-3-Sprachen abgebildet.
Der Schwerpunkt liegt auf wiederkehrenden Mustern
für binäre Logik, Ablaufsteuerungen und Speicherwirkungen.

\subsubsection{Bedingte Logik (IF-Struktur)}

\paragraph{Ziel}
Ausführung einer Aktion nur bei erfüllter Bedingung.

\paragraph{ST}
\begin{verbatim}
IF cond THEN
    y := TRUE;
ELSE
    y := FALSE;
END_IF;
\end{verbatim}

\paragraph{KOP}
\begin{itemize}
    \item Schließerkontakt \texttt{cond}
    \item Normale Spule \texttt{y}
\end{itemize}

\paragraph{AWL}
\begin{verbatim}
LD  cond
ST  y
\end{verbatim}

\paragraph{FBS}
\begin{itemize}
    \item Vergleichs- oder BOOL-Signal
    \item Direkt auf Ausgang verschaltet
\end{itemize}

\paragraph{SFC}
\begin{itemize}
    \item Transition = Bedingung
    \item Aktion im Folgeschritt
\end{itemize}

---

\subsubsection{Speicher (Selbsthaltung)}

\paragraph{Ziel}
Zustand soll erhalten bleiben, auch wenn Bedingung nicht mehr anliegt.

\paragraph{ST}
\begin{verbatim}
IF set THEN
    mem := TRUE;
ELSIF reset THEN
    mem := FALSE;
END_IF;
\end{verbatim}

\paragraph{KOP}
\begin{itemize}
    \item Setzspule (S)
    \item Rücksetzspule (R)
\end{itemize}

\paragraph{AWL}
\begin{verbatim}
LD   set
JMPC SET
LD   reset
JMPC RESET
JMP  END
SET:   LD TRUE
       ST mem
       JMP END
RESET: LD FALSE
       ST mem
END:
\end{verbatim}

\paragraph{FBS}
\begin{itemize}
    \item RS- oder SR-Flipflop-Baustein
\end{itemize}

\paragraph{SFC}
\begin{itemize}
    \item Schritt selbst repräsentiert den Speicher
\end{itemize}

---

\subsubsection{Flankenerkennung}

\paragraph{Ziel}
Aktion soll nur bei Zustandsänderung ausgelöst werden.

\paragraph{ST}
\begin{verbatim}
edge := signal AND NOT signal_old;
signal_old := signal;
\end{verbatim}

\paragraph{KOP}
\begin{itemize}
    \item Positiver oder negativer Flankenkontakt
\end{itemize}

\paragraph{AWL}
\begin{verbatim}
LD   signal
ANDN signal_old
ST   edge
LD   signal
ST   signal_old
\end{verbatim}

\paragraph{FBS}
\begin{itemize}
    \item Flankenbaustein (\texttt{R\_TRIG} / \texttt{F\_TRIG})
\end{itemize}

\paragraph{SFC}
\begin{itemize}
    \item Transition mit Flankenbedingung
\end{itemize}

---

\subsubsection{Zähler}

\paragraph{Ziel}
Ereignisse zählen und auswerten.

\paragraph{ST}
\begin{verbatim}
IF edge THEN
    cnt := cnt + 1;
END_IF;
\end{verbatim}

\paragraph{KOP}
\begin{itemize}
    \item Zählerbaustein (CTU, CTD)
\end{itemize}

\paragraph{AWL}
\begin{verbatim}
LD edge
JMPC INC
JMP END
INC:
LD cnt
ADD 1
ST cnt
END:
\end{verbatim}

\paragraph{FBS}
\begin{itemize}
    \item CTU / CTD / CTUD
\end{itemize}

\paragraph{SFC}
\begin{itemize}
    \item Zähler in Aktionen
\end{itemize}

---

\subsubsection{Zeitverhalten}

\paragraph{Ziel}
Zeitabhängige Steuerung.

\paragraph{ST}
\begin{verbatim}
ton(IN := start, PT := T#5s);
done := ton.Q;
\end{verbatim}

\paragraph{KOP}
\begin{itemize}
    \item Zeitrelais (TON, TOF, TP)
\end{itemize}

\paragraph{AWL}
\begin{verbatim}
LD start
TON t1, PT := T#5s
\end{verbatim}

\paragraph{FBS}
\begin{itemize}
    \item Timer-Funktionsbausteine
\end{itemize}

\paragraph{SFC}
\begin{itemize}
    \item Zeitqualifier (L, D)
\end{itemize}

---

\subsubsection{Ablaufsteuerung}

\paragraph{Ziel}
Mehrere Schritte in definierter Reihenfolge.

\paragraph{ST}
\begin{verbatim}
CASE state OF
    0: state := 1;
    1: state := 2;
END_CASE;
\end{verbatim}

\paragraph{KOP}
\begin{itemize}
    \item Zustandsmerker
    \item Verriegelte Netzwerke
\end{itemize}

\paragraph{AWL}
\begin{itemize}
    \item Sprungmarken
    \item Zustandsvariable
\end{itemize}

\paragraph{FBS}
\begin{itemize}
    \item Zustandsvariable + Logik
\end{itemize}

\paragraph{SFC}
\begin{itemize}
    \item Natürliche Abbildung (Schritte/Transitionen)
\end{itemize}

---

\subsubsection{Zusammenfassung der Umsetzungsmuster}

\begin{center}
\begin{tabular}{|l|c|c|c|c|c|}
\hline
\textbf{Muster} & \textbf{ST} & \textbf{AWL} & \textbf{KOP} & \textbf{FBS} & \textbf{SFC} \\ \hline
IF / Bedingung & sehr gut & gut & sehr gut & gut & gut \\ \hline
Speicher & gut & gut & sehr gut & sehr gut & sehr gut \\ \hline
Flanke & gut & gut & sehr gut & sehr gut & gut \\ \hline
Zähler & gut & gut & sehr gut & sehr gut & gut \\ \hline
Zeit & sehr gut & gut & sehr gut & sehr gut & sehr gut \\ \hline
Ablauf & gut & gut & mittel & mittel & sehr gut \\ \hline
\end{tabular}
\end{center}

\medskip
\textbf{Merksatz:}  
\emph{ST bildet Logik und Algorithmen ab,
KOP/FBS bilden Signale ab,
SFC bildet Abläufe ab.}

%%%%%%%%%%%%%%%%%%%%%%%%%%%%%%%%%%%%%%%%%%%%%%%%%%%%%%%%%%%%%%%%%%%%%%%%%%%%%%%%%%%%%%%%%%%%%%%
\newpage
\section{SPS-Konfiguration und Programmierung}

\subsection{Programmierung nach Norm}

Die Programmierung speicherprogrammierbarer Steuerungen erfolgt
nach der internationalen Norm \textbf{IEC~61131-3}.
Diese Norm definiert sowohl die zulässigen Programmiersprachen
als auch das zugrunde liegende Softwaremodell.

Ziel der Norm ist:
\begin{itemize}
    \item herstellerunabhängige Programmierung
    \item strukturierte und wartbare Software
    \item klare Trennung von Konfiguration und Programm
\end{itemize}

Die IEC~61131-3 unterscheidet fünf Programmiersprachen:
\begin{itemize}
    \item Anweisungsliste (AWL / IL)
    \item Strukturierter Text (ST)
    \item Kontaktplan (KOP / LD)
    \item Funktionsbausteinsprache (FBS / FBD)
    \item Ablaufsprache (SFC)
\end{itemize}

Die Norm legt außerdem fest,
wie Programme organisiert, aufgerufen und ausgeführt werden.

---

\subsection{Konfigurationselemente}

Die Software einer SPS ist hierarchisch aufgebaut.
Diese Struktur ist unabhängig von der konkreten Hardware
und wird als Konfiguration bezeichnet.

\begin{figure}[H]
    \centering
    \includegraphics[width=0.8\textwidth]{/Users/danielweindl/_source/Repositorys/STT-Lernskript/Bilder/KonfigElemente.png}
\end{figure}


\subsubsection{Konfiguration}

Die Konfiguration beschreibt ein vollständiges SPS-System.
Sie bildet die oberste Ebene des Softwaremodells.

Eine Konfiguration umfasst:
\begin{itemize}
    \item eine oder mehrere Ressourcen
    \item globale Variablen
    \item Kommunikationsbeziehungen
\end{itemize}

\subsubsection{Ressource}

Eine Ressource stellt eine logische Verarbeitungseinheit dar,
meist eine CPU.

Jede Ressource enthält:
\begin{itemize}
    \item Tasks
    \item zugeordnete Programme
\end{itemize}

\subsubsection{Task}

Tasks steuern die zeitliche Ausführung der Programme.

Sie legen fest:
\begin{itemize}
    \item Ausführungsart (zyklisch, ereignisgesteuert)
    \item Zykluszeit
    \item Priorität
\end{itemize}

Tasks enthalten selbst keine Programmlogik.

\begin{figure}[H]
    \centering
    \includegraphics[width=0.4\textwidth]{/Users/danielweindl/_source/Repositorys/STT-Lernskript/Bilder/RessourceTask.png}
\end{figure}



\subsection{Bausteine (POE – Program Organisation Elements)}

Die eigentliche Programmlogik wird in
\textbf{Programm-Organisationseinheiten (POE)}
realisiert.


\begin{figure}[H]
    \centering
    \includegraphics[width=1\textwidth]{/Users/danielweindl/_source/Repositorys/STT-Lernskript/Bilder/POE.png}
\end{figure}

Die IEC~61131-3 unterscheidet drei POE-Typen.

\subsubsection{Programme}

Programme werden einer Task zugeordnet
und zyklisch oder ereignisgesteuert ausgeführt.

\begin{itemize}
    \item Kein eigenes Gedächtnis über mehrere Aufrufe
    \item Einstiegspunkt der Programmausführung
\end{itemize}

\subsubsection{Funktionsbausteine}

Funktionsbausteine besitzen ein internes Gedächtnis
und müssen als Instanz verwendet werden.

\begin{itemize}
    \item Speicherung von Zuständen
    \item Wiederverwendbar
    \item Geeignet für zustandsbehaftete Funktionen
\end{itemize}

\subsubsection{Funktionen}

Funktionen besitzen kein Gedächtnis.
Bei gleichen Eingängen liefern sie immer gleiche Ausgänge.

\begin{itemize}
    \item Keine Speicherwirkung
    \item Ideal für Berechnungen und logische Verknüpfungen
\end{itemize}


\begin{figure}[H]
    \centering
    \includegraphics[width=1\textwidth]{/Users/danielweindl/_source/Repositorys/STT-Lernskript/Bilder/POE2.png}
\end{figure}


\subsubsection{Aufrufregeln für POE}

\begin{itemize}
    \item Programme dürfen Funktionen und Funktionsbausteine aufrufen
    \item Funktionsbausteine dürfen Funktionen und Funktionsbausteine aufrufen
    \item Funktionen dürfen nur Funktionen aufrufen
    \item Rekursive Aufrufe sind nicht erlaubt
\end{itemize}

\subsection{Datentypen}

Datentypen legen fest,
welche Werte eine Variable annehmen kann
und wie diese interpretiert werden.

\subsubsection{Elementare Datentypen}

\begin{itemize}
    \item BOOL
    \item INT, DINT
    \item REAL
    \item TIME
\end{itemize}

\subsubsection{Abgeleitete und zusammengesetzte Datentypen}

\begin{itemize}
    \item ARRAY
    \item STRUCT
    \item ENUM
\end{itemize}

Diese Datentypen ermöglichen
strukturierte und übersichtliche Programme.

\subsubsection{Typkonversion}

Bei der Verarbeitung unterschiedlicher Datentypen
sind explizite Typumwandlungen erforderlich.

Beispiel:
\begin{verbatim}
r := INT_TO_REAL(i);
\end{verbatim}

---

\subsection{Variablen}

Variablen repräsentieren Speicherplätze
für Daten innerhalb eines SPS-Programms.

\subsubsection{Lokale Variablen}

\begin{itemize}
    \item Nur innerhalb einer POE sichtbar
    \item Fördern Modularität
    \item Keine Seiteneffekte
\end{itemize}

\subsubsection{Globale Variablen}

\begin{itemize}
    \item Projektweit sichtbar
    \item Ermöglichen Kommunikation zwischen Programmen
    \item Erhöhen die Kopplung
\end{itemize}

\subsubsection{Initialisierung von Variablen}

Variablen können beim Start der SPS
oder beim Neustart einer Ressource
initialisiert werden.

Initialwerte sind insbesondere für:
\begin{itemize}
    \item Zustandsvariablen
    \item Zähler
    \item Merker
\end{itemize}
relevant.

\subsection{Generische Datentypen}

\subsubsection{Begriff}

Generische Datentypen sind \textbf{platzhalterartige Datentypen},
die nicht auf einen konkreten elementaren Datentyp festgelegt sind.
Sie werden verwendet, um Funktionen und Funktionsbausteine
\textbf{datentypunabhängig} zu definieren.

Der konkrete Datentyp wird erst beim Aufruf
durch die tatsächlich übergebenen Variablen festgelegt.

\subsubsection{Motivation und Zweck}

Generische Datentypen dienen:
\begin{itemize}
    \item der Wiederverwendbarkeit von Bausteinen
    \item der Reduktion von mehrfach identischem Code
    \item der allgemeinen Beschreibung von Operationen
\end{itemize}

Typische Beispiele sind Vergleichs- oder logische Funktionen,
die mit unterschiedlichen Datentypen arbeiten sollen.

\subsubsection{Generische Datentypklassen}

Die IEC~61131-3 fasst generische Datentypen
in sogenannte \textbf{Typklassen} zusammen.


\begin{figure}[H]
    \centering
    \includegraphics[width=1\textwidth]{/Users/danielweindl/_source/Repositorys/STT-Lernskript/Bilder/GenerischeDatentypen.png}
\end{figure}


\subsubsection{Beispiel einer generischen Funktion}

\begin{verbatim}
FUNCTION MAX : ANY_NUM
VAR_INPUT
    a : ANY_NUM;
    b : ANY_NUM;
END_VAR
\end{verbatim}


\subsection{Variablen}
\subsubsection{Deklaration}
\begin{figure}[H]
    \centering
    \includegraphics[width=1\textwidth]{/Users/danielweindl/_source/Repositorys/STT-Lernskript/Bilder/Deklaration.png}
\end{figure}
\begin{figure}[H]
    \centering
    \includegraphics[width=0.6\textwidth]{/Users/danielweindl/_source/Repositorys/STT-Lernskript/Bilder/Deklaration2.png}
\end{figure}

\subsubsection{Globale Variablen}

\begin{figure}[H]
    \centering
    \includegraphics[width=1\textwidth]{/Users/danielweindl/_source/Repositorys/STT-Lernskript/Bilder/GlobaleVariablen.png}
\end{figure}



\subsection{Zusammenfassung}

Die Konfiguration und Programmierung nach IEC~61131-3
trennt klar zwischen Struktur, Ablauf und Logik.
Durch Konfigurationselemente, POEs, Datentypen und Variablen
entstehen strukturierte, wartbare und normkonforme
Steuerungsprogramme.

\medskip
\textbf{Merksatz:}  
\emph{Die Norm legt nicht fest, was programmiert wird,
sondern wie Programme strukturiert sind.}
\newpage
\section{Projektierung}
Foliensatz muss nicht gekürzt werden.

\includepdf[pages=2-13]{/Users/danielweindl/_source/Repositorys/STT-Lernskript/Data/sttvo-k05-AT-Projektierung von Steuerungsprogrammen-4v1-Folien.pdf}
\newpage
\section{Entwurf von Steuerungsprogrammen -- Statecharts}

\subsection{Begriff und Ziele}

Der Entwurf von Steuerungsprogrammen mithilfe von Statecharts
ist eine modellbasierte Methode zur Beschreibung
zustandsorientierter Systeme.
Im Mittelpunkt steht die formale Darstellung von Zuständen,
Übergängen und zugehörigen Aktionen.

Ziel des Statechart-basierten Entwurfs ist es,
komplexe Steuerungsabläufe:
\begin{itemize}
    \item übersichtlich darzustellen,
    \item eindeutig zu spezifizieren,
    \item systematisch zu strukturieren,
    \item und fehlerarm zu implementieren.
\end{itemize}

Statecharts dienen dabei als \textbf{Entwurfsmodell}
und sind unabhängig von der späteren Programmiersprache.

\begin{figure}[H]
    \centering
    \includegraphics[page=3,width=\textwidth]{/Users/danielweindl/_source/LaTex/STT-Lernskript/Data/sttvo-k06-Entwurf von Steuerungsprogrammen-4v1-Folien_gesamt.pdf}
\end{figure}
\begin{figure}[H]
    \centering
    \includegraphics[page=6,width=\textwidth]{/Users/danielweindl/_source/LaTex/STT-Lernskript/Data/sttvo-k06-Entwurf von Steuerungsprogrammen-4v1-Folien_gesamt.pdf}
\end{figure}

\subsection{Kodierrichtlinien}

Kodierrichtlinien legen verbindliche Regeln für die Struktur,
Benennung und Umsetzung von Steuerungsprogrammen fest.
Ziel ist es, Programme \textbf{lesbar, wartbar und erweiterbar}
zu gestalten – unabhängig davon, wer sie erstellt oder später wartet.

Wesentliche Inhalte von Kodierrichtlinien sind:
\begin{itemize}
    \item einheitliche Namenskonventionen für Variablen, Bausteine und Zustände
    \item klare Strukturierung von Programmen und Funktionsbausteinen
    \item konsequente Trennung von Ablauf, Logik und Peripherie
    \item aussagekräftige Kommentare an relevanten Stellen
\end{itemize}

Die Einhaltung von Kodierrichtlinien reduziert Fehler,
erleichtert die Fehlersuche und ist insbesondere
bei größeren Projekten und Teamarbeit unverzichtbar.

\subsection{Logikarten}

Logikarten beschreiben,
wie Ein- und Ausgangsgrößen in einer Steuerung
miteinander verknüpft sind.
Grundsätzlich unterscheidet man
zwischen kombinatorischer und sequentieller Logik.
In realen Anwendungen treten beide Logikarten gemeinsam auf.

\subsubsection{Kombinatorische Logik}

Bei der kombinatorischen Logik hängt das Ergebnis
ausschließlich von den aktuellen Eingangsgrößen ab.
Ein Zustands- oder Speicherbegriff existiert nicht.

\paragraph{Resultat}
\begin{itemize}
    \item \textbf{Verknüpfungssteuerung}
\end{itemize}

\paragraph{Modellierung}
\begin{itemize}
    \item \textbf{Wahrheitstabelle}  
    Vollständige Auflistung aller Eingangskombinationen
    und der zugehörigen Ausgänge.

    \item \textbf{Boolesche Logik / KV-Diagramm}  
    Mathematische Beschreibung und Vereinfachung
    logischer Zusammenhänge.
\end{itemize}

\subsubsection{Sequentielle Logik}

Bei der sequentiellen Logik hängt das Ergebnis
von den aktuellen Eingängen \emph{und}
vom internen Zustand des Systems ab.
Eine Speicherwirkung ist vorhanden.

\paragraph{Resultat}
\begin{itemize}
    \item \textbf{Ablaufsteuerung}
\end{itemize}

\paragraph{Modellierung}
\begin{itemize}
    \item \textbf{Flussdiagramm}  
    Lineare Ablaufdarstellung,
    für komplexe Steuerungen jedoch nicht empfohlen.

    \item \textbf{Zustandsautomat}  
    Formale Beschreibung durch Zustände und Übergänge;
    wird in einem eigenen Kapitel behandelt.
\end{itemize}

\subsubsection{Realität industrieller Steuerungen}

In realen Anwendungen werden kombinatorische
und sequentielle Logik nahezu immer gemeinsam eingesetzt.

Typische Eigenschaften:
\begin{itemize}
    \item kombinatorische Logik für Freigaben und Verriegelungen
    \item sequentielle Logik für Abläufe und Prozessschritte
\end{itemize}

Diese Mischung erfordert eine
\textbf{klare Softwarearchitektur},
um Übersichtlichkeit, Wartbarkeit
und Erweiterbarkeit sicherzustellen.

\medskip
\textbf{Merksatz:}  
\emph{Verknüpfungssteuerung entsteht aus kombinatorischer Logik,
Ablaufsteuerung aus sequentieller Logik.}



\subsection{Auswahl der Modellierungstechnik}

\begin{figure}[H]
    \centering
    \includegraphics[page=11,width=\textwidth]{/Users/danielweindl/_source/LaTex/STT-Lernskript/Data/sttvo-k06-Entwurf von Steuerungsprogrammen-4v1-Folien_gesamt.pdf}
\end{figure}


\subsection{Modellierungstechniken}

Zur Beschreibung und zum Entwurf diskreter Steuerungen
stehen unterschiedliche Modellierungstechniken zur Verfügung.
Die Auswahl der Technik hängt davon ab,
ob das System \textbf{kombinatorisch} oder \textbf{sequentiell}
arbeitet.


%##############
\begin{center}
\resizebox{\textwidth}{!}{%
\begin{tikzpicture}[
    box/.style={
        draw,
        rectangle,
        rounded corners,
        align=center,
        minimum width=5.2cm,
        minimum height=1.1cm
    },
    node distance=1.4cm
]

% Hauptknoten
\node[box, thick] (logic) {\textbf{Modellierungstechniken}};

% Kombinatorisch
\node[box, below left=of logic, xshift=-1.2cm] (komb) {\textbf{Kombinatorische Logik}\\
Verknüpfungssteuerung};

\node[box, below=of komb] (wt) {Wahrheitstabelle\\
RS-Tabelle};

\node[box, below=of wt] (bool) {Boolesche Logik\\
KV-Diagramm};

\node[box, below=of bool] (wz1) {Weg-/Zeitdiagramm};

% Sequentiell
\node[box, below right=of logic, xshift=1.2cm] (seq) {\textbf{Sequentielle Logik}\\
Ablaufsteuerung};

\node[box, below=of seq] (wz2) {Weg-/Zeitdiagramm};

\node[box, below=of wz2] (flow) {Flussdiagramm\\
(nicht empfohlen)};

\node[box, below=of flow] (state) {Zustandsautomat\\
Statecharts};

% Realität
\node[box, below=of wt, xshift=7.5cm, thick] (real) {\textbf{Reale Steuerungen}\\
Mischung aus beidem\\
$\Rightarrow$ Softwarearchitektur};

% Pfeile
\draw[->] (logic) -- (komb);
\draw[->] (logic) -- (seq);

\draw[->] (komb) -- (wt);
\draw[->] (wt) -- (bool);
\draw[->] (bool) -- (wz1);

\draw[->] (seq) -- (wz2);
\draw[->] (wz2) -- (flow);
\draw[->] (flow) -- (state);

\draw[->, thick] (wz1.east) -- (real.west);
\draw[->, thick] (state.west) -- (real.east);

\end{tikzpicture}%
}
\end{center}

\subsubsection{Kombinatorische Logik}

Bei der kombinatorischen Logik hängen die Ausgänge
ausschließlich von den aktuellen Eingängen ab.
Es existiert kein Zustands- oder Speicherbegriff.

Typische Modellierungstechniken sind:
\begin{itemize}
    \item \textbf{Wertetabellen / RS-Tabellen}  
    Systematische Auflistung aller möglichen Eingangskombinationen
    und der zugehörigen Ausgänge.

    \item \textbf{Weg-Zeit-Diagramme}  
    Darstellung des Signalverlaufs über der Zeit,
    ohne Berücksichtigung interner Zustände.

    \item \textbf{Stromlaufpläne}  
    Grafische Darstellung logischer Verknüpfungen
    mithilfe elektrischer Schaltelemente,
    insbesondere bei bestehenden oder klassischen Anlagen.
\end{itemize}

\subsubsection{Sequentielle Logik}

Bei der sequentiellen Logik hängt das Systemverhalten
von den aktuellen Eingängen \emph{und} vom internen Zustand ab.
Eine Speicherwirkung ist vorhanden.

Geeignete Modellierungstechniken sind:
\begin{itemize}
    \item \textbf{Weg-Zeit-Diagramme}  
    Darstellung zeitlicher Abfolgen unter Berücksichtigung
    von Zustandswechseln.

    \item \textbf{R\&I-Fließschema}  
    Darstellung von verfahrenstechnischen Abläufen
    mit Zustands- und Signalbezug,
    insbesondere in der Prozessindustrie.

    \item \textbf{Zustandsbasierte Modellierung}  
    Beschreibung des Systems durch Zustände,
    Übergänge und Aktionen,
    z.\,B. mithilfe von Zustandsdiagrammen oder Statecharts.
\end{itemize}

\medskip
\textbf{Merksatz:}  
\emph{Kombinatorische Modelle kennen keinen Zustand,
sequentielle Modelle benötigen einen Zustandsbegriff.}

In den nächsten Kapiteln werden einige Techniken näher betrachtet.


\subsection{RS-Wertetabelle}
\begin{figure}[H]
    \centering
    \includegraphics[page=15,width=\textwidth]{/Users/danielweindl/_source/LaTex/STT-Lernskript/Data/sttvo-k06-Entwurf von Steuerungsprogrammen-4v1-Folien_gesamt.pdf}
\end{figure}
\begin{figure}[H]
    \centering
    \includegraphics[page=16,width=0.6\textwidth]{/Users/danielweindl/_source/LaTex/STT-Lernskript/Data/sttvo-k06-Entwurf von Steuerungsprogrammen-4v1-Folien_gesamt.pdf}
\end{figure}
\begin{figure}[H]
    \centering
    \includegraphics[page=17,width=\textwidth]{/Users/danielweindl/_source/LaTex/STT-Lernskript/Data/sttvo-k06-Entwurf von Steuerungsprogrammen-4v1-Folien_gesamt.pdf}
\end{figure}

\subsection{Funktionsdiagramme}

\begin{figure}[H]
    \centering
    \includegraphics[page=21,width=0.8\textwidth]{/Users/danielweindl/_source/LaTex/STT-Lernskript/Data/sttvo-k06-Entwurf von Steuerungsprogrammen-4v1-Folien_gesamt.pdf}
\end{figure}
\begin{figure}[H]
    \centering
    \includegraphics[page=22,width=0.8\textwidth]{/Users/danielweindl/_source/LaTex/STT-Lernskript/Data/sttvo-k06-Entwurf von Steuerungsprogrammen-4v1-Folien_gesamt.pdf}
\end{figure}

Sie bilden den Übergang zu Statecharts.

\subsection{Weg-/Zeitdiagramm}
\begin{figure}[H]
    \centering
    \includegraphics[page=25,width=\textwidth]{/Users/danielweindl/_source/LaTex/STT-Lernskript/Data/sttvo-k06-Entwurf von Steuerungsprogrammen-4v1-Folien_gesamt.pdf}
\end{figure}
\begin{figure}[H]
    \centering
    \includegraphics[page=26,width=\textwidth]{/Users/danielweindl/_source/LaTex/STT-Lernskript/Data/sttvo-k06-Entwurf von Steuerungsprogrammen-4v1-Folien_gesamt.pdf}
\end{figure}

%%%%%%%%%%%%%%%%%%%%%%%%%%%%%%%%%%%%%%%%%%%%%%%%%%%%%%%%%%%%%%%%%%%%%%%%%%%%%%%%%%%%%%%%%%%

\subsection{Zustandsautomaten - Statecharts}

Statecharts sind eine Erweiterung klassischer Zustandsdiagramme
und wurden von David Harel eingeführt.
Sie ermöglichen die strukturierte Beschreibung
komplexer zustandsbasierter Systeme.

\begin{figure}[H]
    \centering
    \includegraphics[width=\textwidth]{/Users/danielweindl/_source/LaTex/STT-Lernskript/Bilder/Zustandsgraphen.png}
\end{figure}
\begin{figure}[H]
    \centering
    \includegraphics[page=43,width=\textwidth]{/Users/danielweindl/_source/LaTex/STT-Lernskript/Data/sttvo-k06-Entwurf von Steuerungsprogrammen-4v1-Folien_gesamt.pdf}
\end{figure}



\subsubsection{Vorgehensweise}
\begin{enumerate}
  \item Zustände identifizieren
  \item Aktionen in den jeweiligen Zuständen identifizieren
  \item Übergänge und Transitions identifizieren
  \item Start
\end{enumerate}

\subsubsection{States}

Ein \textbf{State} (Zustand) beschreibt den Zustand,
in dem sich ein Objekt oder System
zu einem bestimmten Zeitpunkt befindet.

Ein Zustand repräsentiert eine stabile Phase,
in der das System definierte Eigenschaften besitzt
und auf Ereignisse wartet.

\subsubsection{Transitions}

\textbf{Transitions} beschreiben den Übergang
von einem Zustand in einen anderen.

Ein Übergang legt fest,
\begin{itemize}
    \item von welchem Zustand ausgegangen wird,
    \item in welchen Zustand gewechselt wird,
    \item unter welchen Bedingungen der Wechsel erfolgt.
\end{itemize}
\begin{figure}[H]
    \centering
    \includegraphics[page=46,width=0.8\textwidth]{/Users/danielweindl/_source/LaTex/STT-Lernskript/Data/sttvo-k06-Entwurf von Steuerungsprogrammen-4v1-Folien_gesamt.pdf}
\end{figure}
\begin{figure}[H]
    \centering
    \includegraphics[page=47,width=0.8\textwidth]{/Users/danielweindl/_source/LaTex/STT-Lernskript/Data/sttvo-k06-Entwurf von Steuerungsprogrammen-4v1-Folien_gesamt.pdf}
\end{figure}

\subsubsection{Events}

\textbf{Events} sind Ereignisse,
die einen Zustandsübergang auslösen können.

Sie bestimmen,
welcher Übergang von einem Zustand aus
ausgeführt wird,
wenn mehrere Übergänge möglich sind.

Typische Events sind:
\begin{itemize}
    \item Signaländerungen
    \item Zeitereignisse
    \item Benutzeraktionen
\end{itemize}

\subsubsection{Actions}

\textbf{Actions} sind Aktionen,
die beim Übergang zwischen Zuständen
oder innerhalb eines Zustands ausgeführt werden.

Sie sind optional
und dienen der Ausführung konkreter Operationen,
z.\,B. Setzen von Ausgängen oder Initialisieren von Variablen.

\begin{figure}[H]
    \centering
    \includegraphics[page=48,width=0.8\textwidth]{/Users/danielweindl/_source/LaTex/STT-Lernskript/Data/sttvo-k06-Entwurf von Steuerungsprogrammen-4v1-Folien_gesamt.pdf}
\end{figure}


\subsubsection{Beispiel}
\begin{figure}[H]
    \centering
    \includegraphics[page=49,width=0.8\textwidth]{/Users/danielweindl/_source/LaTex/STT-Lernskript/Data/sttvo-k06-Entwurf von Steuerungsprogrammen-4v1-Folien_gesamt.pdf}
\end{figure}
\begin{figure}[H]
    \centering
    \includegraphics[page=50,width=0.8\textwidth]{/Users/danielweindl/_source/LaTex/STT-Lernskript/Data/sttvo-k06-Entwurf von Steuerungsprogrammen-4v1-Folien_gesamt.pdf}
\end{figure}

\subsubsection{Übergang zu Harel}
Die Anzahl der Zustände und Transitionen wächst bei komplexen Steuerungsaufgaben sehr schnell stark an, was zu unübersichtlichen und schwer wartbaren Modellen führt. Dieses Problem tritt nicht nur bei Zustandsdiagrammen, sondern in ähnlicher Weise auch bei Flussdiagrammen auf. Klassische Mealy- und Moore-Automaten bieten hierfür keine wirklich geeignete Lösung, da sie die Modellierungsmöglichkeiten stark einschränken. Mealy-Automaten erlauben die Ausführung von Aktionen ausschließlich während einer aktiven Transition, während Moore-Automaten Aktionen nur beim Eintritt in einen Zustand zulassen. Für die Anforderungen der Steuerungstechnik sind diese Einschränkungen in der Praxis meist ungeeignet. Aus diesem Grund werden in UML und in der modernen Steuerungstechnik erweiterte Zustandsmodelle eingesetzt, insbesondere die sogenannten Harel-Automaten, die eine strukturiertere und leistungsfähigere Modellierung komplexer Abläufe ermöglichen.

\subsubsection{Harel-Automat}

Der Harel-Automat erweitert klassische endliche Automaten
um zusätzliche Strukturierungsmöglichkeiten.

Er unterstützt:
\begin{itemize}
    \item Hierarchie
    \item Parallelität
    \item Synchronisation
\end{itemize}

Dadurch lassen sich komplexe Steuerungen
übersichtlich und modular modellieren.

\begin{figure}[H]
    \centering
    \includegraphics[page=52,width=0.8\textwidth]{/Users/danielweindl/_source/LaTex/STT-Lernskript/Data/sttvo-k06-Entwurf von Steuerungsprogrammen-4v1-Folien_gesamt.pdf}
\end{figure}
\begin{figure}[H]
    \centering
    \includegraphics[page=53,width=0.8\textwidth]{/Users/danielweindl/_source/LaTex/STT-Lernskript/Data/sttvo-k06-Entwurf von Steuerungsprogrammen-4v1-Folien_gesamt.pdf}
\end{figure}
\begin{figure}[H]
    \centering
    \includegraphics[page=54,width=0.8\textwidth]{/Users/danielweindl/_source/LaTex/STT-Lernskript/Data/sttvo-k06-Entwurf von Steuerungsprogrammen-4v1-Folien_gesamt.pdf}
\end{figure}


\subsubsection{Super States}

Super States (übergeordnete Zustände)
fassen mehrere Einzelzustände zusammen.

Vorteile:
\begin{itemize}
    \item Reduktion der Komplexität
    \item gemeinsame Übergänge
    \item bessere Lesbarkeit
\end{itemize}

Super States ermöglichen hierarchische Statecharts.

\begin{figure}[H]
    \centering
    \includegraphics[page=55,width=0.8\textwidth]{/Users/danielweindl/_source/LaTex/STT-Lernskript/Data/sttvo-k06-Entwurf von Steuerungsprogrammen-4v1-Folien_gesamt.pdf}
\end{figure}
\begin{figure}[H]
    \centering
    \includegraphics[page=56,width=0.8\textwidth]{/Users/danielweindl/_source/LaTex/STT-Lernskript/Data/sttvo-k06-Entwurf von Steuerungsprogrammen-4v1-Folien_gesamt.pdf}
\end{figure}
\begin{figure}[H]
    \centering
    \includegraphics[page=57,width=0.8\textwidth]{/Users/danielweindl/_source/LaTex/STT-Lernskript/Data/sttvo-k06-Entwurf von Steuerungsprogrammen-4v1-Folien_gesamt.pdf}
\end{figure}





\subsubsection{Synchronisations Messages}

Synchronisationsnachrichten dienen der Koordination
paralleler Zustände oder Teilautomaten.

Eigenschaften:
\begin{itemize}
    \item ereignisbasiert
    \item entkoppeln Zustandsbereiche
    \item ermöglichen parallele Abläufe
\end{itemize}

Sie sind insbesondere bei verteilten oder modularen Systemen relevant.

\begin{figure}[H]
    \centering
    \includegraphics[page=60,width=0.8\textwidth]{/Users/danielweindl/_source/LaTex/STT-Lernskript/Data/sttvo-k06-Entwurf von Steuerungsprogrammen-4v1-Folien_gesamt.pdf}
\end{figure}


\subsubsection{Implementierung}

Statecharts werden nicht direkt ausgeführt,
sondern in eine konkrete Implementierung überführt.

Typische Umsetzungen:
\begin{itemize}
    \item SFC (Ablaufsprache)
    \item Zustandsvariable mit CASE-Struktur
    \item Funktionsbausteine mit Zustandslogik
\end{itemize}

Wichtig ist die saubere Abbildung:
\begin{itemize}
    \item der Zustände,
    \item der Übergangsbedingungen,
    \item und der Aktionen.
\end{itemize}

\medskip
\textbf{Merksatz:}  
\emph{Statecharts sind Entwurfsmodelle,
keine Programmiersprachen.}


\begin{figure}[H]
    \centering
    \includegraphics[page=61,width=0.8\textwidth]{/Users/danielweindl/_source/LaTex/STT-Lernskript/Data/sttvo-k06-Entwurf von Steuerungsprogrammen-4v1-Folien_gesamt.pdf}
\end{figure}
\begin{figure}[H]
    \centering
    \includegraphics[page=62,width=0.8\textwidth]{/Users/danielweindl/_source/LaTex/STT-Lernskript/Data/sttvo-k06-Entwurf von Steuerungsprogrammen-4v1-Folien_gesamt.pdf}
\end{figure}
\begin{figure}[H]
    \centering
    \includegraphics[page=63,width=0.8\textwidth]{/Users/danielweindl/_source/LaTex/STT-Lernskript/Data/sttvo-k06-Entwurf von Steuerungsprogrammen-4v1-Folien_gesamt.pdf}
\end{figure}
\begin{figure}[H]
    \centering
    \includegraphics[page=64,width=0.8\textwidth]{/Users/danielweindl/_source/LaTex/STT-Lernskript/Data/sttvo-k06-Entwurf von Steuerungsprogrammen-4v1-Folien_gesamt.pdf}
\end{figure}



\newpage
\section{Elektrische Steuerungen}

\subsection{Zuordnung von Steuerungsteilen}

\subsection{Schaltzeichen und Betriebsmittelkennzeichen}

\subsection{Schaltungsunterlagen}

\subsection{Geräte und Bauelemente der elektrischen Steuerungstechnik}

\input{chapters/Betriebsmittelkennzeichnung.tex}
\input{chapters/IO-Baugruppen.tex}
\input{chapters/Gebersysteme.tex}
% ------------------ Literaturverzeichnis ------------------
%\printbibliography
\end{document}