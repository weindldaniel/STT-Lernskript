\newpage
\section{Programmiersprachen}

\subsection{Allgemein}
\subsection*{Allgemeines und Norm IEC 61131-3}

Die Programmierung speicherprogrammierbarer Steuerungen (SPS) erfolgt normiert
nach der internationalen Norm \textbf{IEC 61131-3}.
Diese Norm definiert sowohl die zulässigen Programmiersprachen als auch
das Softwaremodell und grundlegende Konzepte der SPS-Programmierung.

Ziel der Norm ist eine herstellerunabhängige, strukturierte und wartbare
Automatisierungssoftware.

\subsection*{Programmiersprachen nach IEC 61131-3}

Die IEC 61131-3 definiert fünf standardisierte Programmiersprachen:

\begin{figure}[H]
    \centering
    \includegraphics[width=0.7\textwidth]{/Users/danielweindl/_source/Repositorys/STT-Lernskript/Bilder/Programmiersprachen.png}
\end{figure}

\renewcommand{\arraystretch}{1.3} % größerer Zeilenabstand
\begin{center}
\begin{tabular}{|l|c|l|}
\hline
\textbf{Deutsch} & \textbf{Abkürzung} & \textbf{English} \\ \hline
Anweisungsliste & AWL -- IL & Instruction List \\ \hline
Strukturierter Text & ST & Structured Text \\ \hline
Funktionsbausteinsprache & FBS -- FBD & Function Block Diagram \\ \hline
Kontaktplan & KOP -- LD & Ladder Diagram \\ \hline
Ablaufsprache & AS -- SFC & Sequential Function Chart \\ \hline
\end{tabular}
\end{center}

Die Sprachen lassen sich in zwei Gruppen einteilen:
\begin{itemize}
    \item \textbf{textuelle Sprachen}: AWL, ST
    \item \textbf{grafische Sprachen}: KOP, FBS, AS
\end{itemize}

\subsection*{Softwaremodell einer SPS}

Nach IEC 61131-3 ist die Software einer SPS hierarchisch aufgebaut.
Diese Hierarchie dient der Strukturierung, Wiederverwendbarkeit und Wartbarkeit.

\subsubsection*{Konfiguration}

Die Konfiguration bildet die oberste Ebene des Softwaremodells.
Sie entspricht einem vollständigen SPS-System und kann eine oder mehrere
Ressourcen enthalten.

In komplexen Anlagen können mehrere Konfigurationen existieren,
die miteinander kommunizieren.

\subsubsection*{Ressource}

Eine Ressource ist einer Konfiguration untergeordnet und umfasst typischerweise
eine CPU eines SPS-Systems.

Jede Ressource enthält:
\begin{itemize}
    \item Tasks
    \item Programm-Organisationseinheiten (POEs)
\end{itemize}

\subsubsection*{Task}

Eine Task definiert die \textbf{Abarbeitungseigenschaften} von Programmen.
Sie legt fest:
\begin{itemize}
    \item Art der Ausführung (zyklisch, ereignisgesteuert)
    \item Zykluszeit
    \item Priorität
\end{itemize}

Die Task selbst enthält keine Programmlogik, sondern steuert deren Ausführung.

\subsubsection*{Programm-Organisationseinheiten (POE)}

POEs sind die eigentlichen Träger der Programmlogik.
Die IEC 61131-3 unterscheidet drei POE-Typen:

\begin{itemize}
    \item Programme
    \item Funktionsbausteine
    \item Funktionen
\end{itemize}

\paragraph{Programme}
Programme besitzen kein eigenes Gedächtnis über mehrere Aufrufe hinaus.
Sie werden einer Task zugeordnet und zyklisch oder ereignisgesteuert ausgeführt.

\paragraph{Funktionsbausteine}
Funktionsbausteine besitzen ein internes Gedächtnis.
Sie werden über Instanzen aufgerufen und eignen sich zur Abbildung
zustandsbehafteter Funktionen.

\paragraph{Funktionen}
Funktionen besitzen kein Gedächtnis.
Sie liefern bei gleichen Eingängen immer gleiche Ausgänge
und eignen sich für Berechnungen und logische Verknüpfungen.

\subsection*{Aufrufhierarchie von POEs}

POEs können sich gegenseitig aufrufen, jedoch gelten folgende Regeln:

\begin{itemize}
    \item Rekursive Aufrufe sind nicht erlaubt
    \item Programme dürfen Funktionen und Funktionsbausteine aufrufen
    \item Funktionsbausteine dürfen Funktionen und Funktionsbausteine aufrufen
    \item Funktionen dürfen ausschließlich Funktionen aufrufen
\end{itemize}

\medskip
\textbf{Merksatz:}  
\emph{Funktionen haben kein Gedächtnis – daher dürfen sie keine
Funktionsbausteine oder Programme aufrufen.}

\subsection*{Zyklisches Abarbeitungsmodell}

Die Programmausführung einer SPS erfolgt typischerweise zyklisch:

\begin{enumerate}
    \item Einlesen der Eingänge
    \item Programmausführung
    \item Schreiben der Ausgänge
\end{enumerate}

Dieses Modell gewährleistet deterministisches Verhalten
und ist Grundlage für die zeitliche Bewertung von Steuerungsprogrammen.

\subsection*{Variablen und Sichtbarkeit}

In POEs können Variablen unterschiedlicher Sichtbarkeit verwendet werden:

\begin{itemize}
    \item Lokale Variablen: nur innerhalb der POE sichtbar
    \item Globale Variablen: im gesamten Projekt sichtbar
\end{itemize}

Globale Variablen ermöglichen die Kommunikation zwischen Programmen,
bergen jedoch die Gefahr unübersichtlicher Abhängigkeiten.

\subsection*{Zusammenfassung}

Die IEC 61131-3 stellt ein einheitliches Modell zur Verfügung, um
Steuerungsprogramme strukturiert, wartbar und normkonform zu erstellen.
Die Wahl der Programmiersprache richtet sich nach der Aufgabenstellung,
der Komplexität und den Anforderungen an Übersichtlichkeit und Wartbarkeit.

%%%%%%%%%%%%%%%%%%%%%%%%%%%%%%%%%%%%%%%%%%%%%%%%%%%%%%%%%%%%%%%%%%%%%%%%%%%%%%%%%%%%%%%%%%%%%%%

\newpage
\subsection{AWL}

\subsubsection{Grundprinzip}

Die Anweisungsliste (AWL, engl. Instruction List – IL) ist eine textuelle
Programmiersprache der IEC~61131-3.
Sie ist maschinennah aufgebaut und lehnt sich konzeptionell an die
Assemblerprogrammierung an.

AWL arbeitet nach dem Prinzip einer \textbf{Ein-Adress-Maschine}.
Alle Operationen erfolgen über ein zentrales Rechenregister,
den sogenannten \textbf{Akkumulator (AKKU)}.

\subsubsection{Ein-Adress-Maschine}

Bei einer Ein-Adress-Maschine besitzt jede Rechenoperation genau einen expliziten
Operanden.
Der zweite Operand ist implizit der Inhalt des Akkumulators.

\medskip
Arbeitsweise:
\begin{enumerate}
    \item Ein Wert wird in den Akku geladen
    \item Der Akku wird mit weiteren Operanden verarbeitet
    \item Das Ergebnis wird aus dem Akku gespeichert
\end{enumerate}

\paragraph{Beispiel}
Berechnung:
\[
a = (x + y) \cdot c
\]

AWL-Code:
\begin{verbatim}
LD   x
ADD  y
MUL  c
ST   a
\end{verbatim}

\subsubsection{Aufbau eines AWL-Programms}

Ein AWL-Programm besteht aus einer Folge von Anweisungen.
Jede Anweisung steht in einer eigenen Zeile und besteht aus:
\begin{itemize}
    \item einem Operator
    \item optional einem oder mehreren Operanden
\end{itemize}

Vor einer Anweisung kann optional eine Sprungmarke (Label) stehen.

\subsubsection{Akkumulator}

Der Akkumulator speichert stets das Ergebnis der letzten Operation.
Jede neue Operation überschreibt den vorherigen Inhalt des Akkus.

Der Datentyp des Akkumulators wird durch die ausgeführte Operation bestimmt.
Daher müssen die Datentypen aufeinanderfolgender Anweisungen kompatibel sein.

\medskip
\textbf{Wichtig:}  
Ein falscher Datentyp im Akku führt zu Laufzeit- oder Compilerfehlern.

\subsubsection{Operatoren und Operanden}

\paragraph{Operatoren}
Operatoren beschreiben die auszuführende Operation.
Typische Operatoren sind:
\begin{itemize}
    \item LD (Load)
    \item ST (Store)
    \item AND, OR, XOR
    \item ADD, SUB, MUL, DIV
    \item EQ, GT, LT
\end{itemize}

\paragraph{Operanden}
Operanden können sein:
\begin{itemize}
    \item Variablen
    \item Literale
    \item Instanznamen von Funktionsbausteinen
\end{itemize}

\subsubsection{Modifier in AWL}

Operatoren können durch sogenannte Modifier erweitert werden.

\paragraph{Negationsmodifier N}
Der Modifier \texttt{N} negiert den Operanden vor der Ausführung der Operation.

\begin{verbatim}
LD   Var1
ANDN Var2
ST   Var3
\end{verbatim}

Dies entspricht logisch:
\[
Var3 := Var1 \land \lnot Var2
\]

\subsubsection{Klammerung und Schachtelung}

AWL erlaubt die Klammerung von Anweisungsfolgen.
Der geklammerte Ausdruck wird zuerst ausgewertet und anschließend
mit dem vorherigen Akkumulatorwert verknüpft.

\begin{verbatim}
LD   Var1
AND(
     Var2
     OR Var3
   )
ST   Var4
\end{verbatim}

Ergebnis:
\[
Var4 := Var1 \land (Var2 \lor Var3)
\]

Klammern können geschachtelt werden, müssen jedoch stets
einen gültigen Datentyp liefern.

\subsubsection{Bedingte Ausführung}

AWL ermöglicht die bedingte Ausführung von Anweisungen.
Hierzu wird das aktuelle Ergebnis im Akku als Bedingung verwendet.

\paragraph{Modifier C}
Mit dem Modifier \texttt{C} wird eine Anweisung nur ausgeführt,
wenn der Akkumulator den Wert TRUE enthält.

\begin{verbatim}
LD    Var1
GT    10
JMPC  label
LD    20
ST    Var2
label:
\end{verbatim}

Ist \texttt{Var1 > 10}, wird der Sprung ausgeführt und
die Zuweisung an \texttt{Var2} übersprungen.

\subsubsection{Sprünge und Marken}

AWL unterstützt bedingte und unbedingte Sprünge.
Sprungziele werden durch Marken (Labels) definiert.

\begin{itemize}
    \item JMP  – unbedingter Sprung
    \item JMPC – bedingter Sprung (TRUE)
    \item JMPCN – bedingter Sprung (FALSE)
\end{itemize}

Sprünge ermöglichen einfache Ablaufsteuerungen,
ersetzen jedoch keine strukturierte Ablaufbeschreibung.

\subsubsection{Kommentare}

Kommentare dienen ausschließlich der Dokumentation
und haben keinen Einfluss auf die Programmausführung.

\begin{itemize}
    \item Mehrzeilige Kommentare: \texttt{(* Kommentar *)}
    \item Zeilenkommentare (herstellerspezifisch): \texttt{// Kommentar}
\end{itemize}

In AWL sind Kommentare besonders wichtig,
da die Lesbarkeit geringer ist als bei grafischen Sprachen.

\subsubsection{Vor- und Nachteile von AWL}

\paragraph{Vorteile}
\begin{itemize}
    \item Maschinennah und effizient
    \item Gute Kontrolle über den Programmablauf
    \item Binäre Logik sehr einfach abbildbar
    \item Analogwertverarbeitung einfacher als in KOP
\end{itemize}

\paragraph{Nachteile}
\begin{itemize}
    \item Geringe Übersichtlichkeit
    \item Schleifen und Verzweigungen nur umständlich realisierbar
    \item Hoher Dokumentationsaufwand notwendig
\end{itemize}

\subsubsection{Einordnung}

AWL ist besonders geeignet für:
\begin{itemize}
    \item zeitkritische Programmteile
    \item maschinennahe Steuerungslogik
    \item Analyse und Optimierung bestehender Programme
\end{itemize}

In modernen Projekten wird AWL häufig durch ST oder grafische Sprachen ergänzt.

\newpage
\subsection{ST}

\subsubsection{Grundprinzip}

Strukturierter Text (ST, engl. Structured Text) ist eine textuelle
Hochsprache nach IEC~61131-3.
Sie ähnelt klassischen Programmiersprachen wie Pascal oder C
und eignet sich besonders für komplexe Algorithmen,
Berechnungen und strukturierte Ablaufbeschreibungen.

ST ist deterministisch und zyklisch in das SPS-Abarbeitungsmodell eingebettet.

\begin{figure}[H]
    \centering
    \includegraphics[page=114,width=0.8\textwidth]{/Users/danielweindl/_source/Repositorys/STT-Lernskript/Data/sttvo-k03-Programmiersprachen-4v5-Folien.pdf}    
 \end{figure}


\subsubsection{Aufbau und Syntax}

Ein ST-Programm besteht aus Anweisungen, die sequenziell abgearbeitet werden.
Die Syntax ist blockorientiert und verwendet Schlüsselwörter zur
Strukturierung des Programms.

Beispiel:
\begin{verbatim}
IF a > b THEN
    c := a;
ELSE
    c := b;
END_IF;
\end{verbatim}

\subsubsection{Datentypen}

ST unterstützt alle in IEC~61131-3 definierten Datentypen.

\paragraph{Elementare Datentypen}
\begin{itemize}
    \item BOOL
    \item INT, DINT
    \item REAL
    \item TIME
\end{itemize}

\paragraph{Zusammengesetzte Datentypen}
\begin{itemize}
    \item ARRAY
    \item STRUCT
    \item ENUM (Aufzählungstypen)
\end{itemize}

\paragraph{Beispiel: ARRAY}
\begin{verbatim}
TYPE
    messwert : ARRAY[1..50] OF REAL;
END_TYPE
\end{verbatim}

\paragraph{Beispiel: STRUCT}
\begin{verbatim}
TYPE
    ventil : STRUCT
        vorhanden : BOOL;
        ausschuss : BOOL;
        zylinder  : BOOL;
        farbe     : INT;
    END_STRUCT;
END_TYPE
\end{verbatim}

\subsubsection{Typkonversion}

Bei gemischter Verwendung von Datentypen ist eine explizite
Typkonversion erforderlich.

\begin{verbatim}
VAR
    i : INT;
    r : REAL;
END_VAR

r := INT_TO_REAL(i);
i := TRUNC(r);
\end{verbatim}

\subsubsection{Kontrollstrukturen}

\paragraph{IF / ELSIF / ELSE}
\begin{verbatim}
IF x = 0 THEN
    y := 0;
ELSIF x > 0 THEN
    y := 1;
ELSE
    y := -1;
END_IF;
\end{verbatim}
\begin{figure}[H]
    \centering
    \includegraphics[page=123,width=0.8\textwidth]{/Users/danielweindl/_source/Repositorys/STT-Lernskript/Data/sttvo-k03-Programmiersprachen-4v5-Folien.pdf}    
 \end{figure}

\paragraph{CASE}
\begin{verbatim}
CASE state OF
    0: y := 0;
    1: y := 10;
    2: y := 20;
ELSE
    y := -1;
END_CASE;
\end{verbatim}
\begin{figure}[H]
    \centering
    \includegraphics[page=125,width=0.8\textwidth]{/Users/danielweindl/_source/Repositorys/STT-Lernskript/Data/sttvo-k03-Programmiersprachen-4v5-Folien.pdf}    
 \end{figure}
\begin{figure}[H]
    \centering
    \includegraphics[page=126,width=0.8\textwidth]{/Users/danielweindl/_source/Repositorys/STT-Lernskript/Data/sttvo-k03-Programmiersprachen-4v5-Folien.pdf}    
 \end{figure}

\subsubsection{Schleifen}

\paragraph{WHILE}
\begin{verbatim}
WHILE x < 10 DO
    x := x + 1;
END_WHILE;
\end{verbatim}

\paragraph{REPEAT}
\begin{verbatim}
REPEAT
    x := x + 1;
UNTIL x >= 10
END_REPEAT;
\end{verbatim}

\paragraph{FOR-Schleife}

Die FOR-Schleife ist die wichtigste Schleifenform in ST
und in den Folien detailliert behandelt.

\begin{verbatim}
FOR cnt := startVal TO endVal BY stepVal DO
    (* Anweisungen *)
END_FOR;
\end{verbatim}
\begin{figure}[H]
    \centering
    \includegraphics[page=129,width=0.8\textwidth]{/Users/danielweindl/_source/Repositorys/STT-Lernskript/Data/sttvo-k03-Programmiersprachen-4v5-Folien.pdf}    
 \end{figure}
Hinweis: Bei Schleifen mit unbestimmter Laufzeit (WHILE, REPEAT)
sollte ein Zähler mit EXIT-Abbruch eingebaut werden, damit es im
Extremfall zu keiner Zykluszeitverletzung kommt, sondern der Fehler
programmatisch abgefangen werden kann.
\subsubsection{Abarbeitungsverhalten der FOR-Schleife}

\begin{itemize}
    \item Wird \texttt{startVal > endVal} bei positivem \texttt{stepVal},  
    wird die Schleife \textbf{nicht ausgeführt}.
    \item Bei \texttt{startVal = endVal} wird die Schleife \textbf{genau einmal} ausgeführt.
    \item Der Abbruch erfolgt bei \textbf{größer} bzw. \textbf{kleiner},
    nicht bei größer/kleiner gleich.
    \item Der Zählerwert nach Verlassen der Schleife ist
    \texttt{letzter Wert + stepVal}.
    \item Änderungen von \texttt{startVal}, \texttt{endVal} oder \texttt{stepVal}
    innerhalb der Schleife werden berücksichtigt.
    \item \texttt{stepVal = 0} führt zu einer Endlosschleife
    und damit zu einer Zykluszeitverletzung.
    \item Der Schleifenzähler muss ein ganzzahliger Datentyp sein.
\end{itemize}

\subsubsection{EXIT}

Mit \texttt{EXIT} kann eine Schleife vorzeitig verlassen werden.

\begin{verbatim}
FOR i := 1 TO 10 DO
    IF i = 5 THEN
        EXIT;
    END_IF;
END_FOR;
\end{verbatim}
\begin{figure}[H]
    \centering
    \includegraphics[page=127,width=0.8\textwidth]{/Users/danielweindl/_source/Repositorys/STT-Lernskript/Data/sttvo-k03-Programmiersprachen-4v5-Folien.pdf}    
 \end{figure}
\begin{figure}[H]
    \centering
    \includegraphics[page=131,width=0.8\textwidth]{/Users/danielweindl/_source/Repositorys/STT-Lernskript/Data/sttvo-k03-Programmiersprachen-4v5-Folien.pdf}    
 \end{figure}

\subsubsection{Grundstrukturen}
\begin{figure}[H]
    \centering
    \includegraphics[page=134,width=0.8\textwidth]{/Users/danielweindl/_source/Repositorys/STT-Lernskript/Data/sttvo-k03-Programmiersprachen-4v5-Folien.pdf}    
 \end{figure}
\begin{figure}[H]
    \centering
    \includegraphics[page=135,width=0.8\textwidth]{/Users/danielweindl/_source/Repositorys/STT-Lernskript/Data/sttvo-k03-Programmiersprachen-4v5-Folien.pdf}    
 \end{figure}
\begin{figure}[H]
    \centering
    \includegraphics[page=136,width=0.8\textwidth]{/Users/danielweindl/_source/Repositorys/STT-Lernskript/Data/sttvo-k03-Programmiersprachen-4v5-Folien.pdf}    
 \end{figure}


\subsubsection{Determinismus und Laufzeit}

ST-Anweisungen werden innerhalb eines SPS-Zyklus vollständig abgearbeitet.
Endlosschleifen oder sehr große Schleifen können
eine \textbf{Zykluszeitverletzung} verursachen.

\subsubsection{Vor- und Nachteile von ST}

\paragraph{Vorteile}
\begin{itemize}
    \item Sehr gut lesbar und strukturiert
    \item Mächtige Kontrollstrukturen
    \item Ideal für Berechnungen und Algorithmen
    \item Gute Wartbarkeit
\end{itemize}

\paragraph{Nachteile}
\begin{itemize}
    \item Weniger anschaulich für binäre Logik
    \item Fehler durch Endlosschleifen möglich
\end{itemize}

\subsubsection{Einordnung}

ST ist besonders geeignet für:
\begin{itemize}
    \item komplexe Ablauf- und Rechenalgorithmen
    \item Datenverarbeitung
    \item Umsetzung von Hochsprachenkonstrukten
\end{itemize}

In der Praxis wird ST häufig mit grafischen Sprachen kombiniert.

\newpage
\subsection{FBS}

\subsubsection{Grundprinzip}

Die Funktionsbausteinsprache (FBS, engl. Function Block Diagram – FBD)
ist eine grafische Programmiersprache nach IEC~61131-3.
Sie basiert auf der Darstellung von Funktionen und Funktionsbausteinen
als grafische Blöcke, die über Signalverbindungen miteinander verknüpft sind.

Im Mittelpunkt steht der \textbf{Signalfluss} zwischen Ein- und Ausgängen.

\subsubsection{Signalflussorientierung}

In FBS wird die Programmlogik durch die Verbindung von Ausgängen
zu Eingängen anderer Bausteine beschrieben.

\begin{itemize}
    \item Daten fließen von links nach rechts
    \item Ein Ausgang kann mehrere Eingänge speisen
    \item Der Signalfluss bestimmt die logische Abhängigkeit
\end{itemize}

Die zeitliche Ausführung erfolgt dennoch innerhalb des SPS-Zyklus
und wird durch die zugeordnete Task bestimmt.

\subsubsection{Bausteintypen in FBS}

In FBS können folgende Bausteintypen verwendet werden:

\begin{itemize}
    \item Funktionen (stateless)
    \item Funktionsbausteine (mit Gedächtnis)
    \item Erweiterbare Funktionen
\end{itemize}

\paragraph{Funktionen}
Funktionen besitzen kein internes Gedächtnis.
Bei gleichen Eingängen liefern sie immer gleiche Ausgänge.

\paragraph{Funktionsbausteine}
Funktionsbausteine besitzen ein internes Gedächtnis.
Sie benötigen eine Instanzvariable und sind für
zustandsbehaftete Aufgaben geeignet (z.\,B. Timer, Zähler).

\begin{figure}[H]
    \centering
    \includegraphics[page=95,width=0.8\textwidth]{/Users/danielweindl/_source/Repositorys/STT-Lernskript/Data/sttvo-k03-Programmiersprachen-4v5-Folien.pdf}    
 \end{figure}


\subsubsection{Bausteininstanzen}

Jeder Funktionsbaustein wird über eine Instanz aufgerufen.
Die Instanz speichert die internen Zustände des Bausteins
über mehrere SPS-Zyklen hinweg.

\begin{itemize}
    \item Eine Instanz pro Funktionsbaustein-Aufruf
    \item Mehrere Instanzen desselben Bausteintyps möglich
\end{itemize}

Ohne Instanz ist keine Speicherwirkung möglich.

\subsubsection{Abarbeitungsreihenfolge}
\begin{figure}[H]
    \centering
    \includegraphics[page=97,width=0.8\textwidth]{/Users/danielweindl/_source/Repositorys/STT-Lernskript/Data/sttvo-k03-Programmiersprachen-4v5-Folien.pdf}    
 \end{figure}



\subsubsection{EN/ENO-Logik}

Viele Funktionsbausteine und Funktionen besitzen optionale
Enable-Eingänge (EN) und Enable-Ausgänge (ENO).

\begin{itemize}
    \item EN = FALSE: Baustein wird nicht ausgeführt
    \item EN = TRUE: Baustein wird ausgeführt
    \item ENO zeigt an, ob der Baustein aktiv war
\end{itemize}

Ist EN FALSE, bleiben die Ausgänge des Bausteins unverändert.
Dies ist insbesondere bei der Weiterverschaltung zu beachten.

\subsubsection{Weiterverschaltung von Bausteinen}

\begin{itemize}
    \item Bausteine ohne EN/ENO können direkt weiter verschaltet werden
    \item Bausteine mit aktiviertem EN/ENO dürfen
    ausgangsseitig nicht direkt weiter verschaltet werden
    \item Zwischenspeicherung über Variablen ist erforderlich
\end{itemize}

Diese Einschränkung dient der Vermeidung undefinierter Zustände,
wenn Bausteine deaktiviert sind.

\subsubsection{Analoge Signalverarbeitung}

FBS eignet sich besonders gut zur Verarbeitung analoger Werte.

\begin{itemize}
    \item Analoge Werte werden über Linien dargestellt
    \item Keine analogen Kontakte wie in KOP
    \item Verbindung nur über Funktions- oder FB-Ein-/Ausgänge
\end{itemize}

Typische Anwendungen:
\begin{itemize}
    \item Skalierung
    \item Vergleich
    \item Berechnung
    \item Filterung
\end{itemize}

\subsubsection{Erweiterbare Funktionen}

Die IEC~61131-3 definiert erweiterbare Funktionen,
die mehr als zwei Eingänge besitzen können.

Beispiele:
\begin{itemize}
    \item ADD
    \item MUL
    \item AND
    \item OR
    \item MAX / MIN
\end{itemize}

Nicht beschaltete Eingänge sind nicht zulässig.

\subsubsection{Programmflusssteuerung}
\begin{figure}[H]
    \centering
    \includegraphics[page=102,width=0.8\textwidth]{/Users/danielweindl/_source/Repositorys/STT-Lernskript/Data/sttvo-k03-Programmiersprachen-4v5-Folien.pdf}    
 \end{figure}
\begin{figure}[H]
    \centering
    \includegraphics[page=103,width=0.8\textwidth]{/Users/danielweindl/_source/Repositorys/STT-Lernskript/Data/sttvo-k03-Programmiersprachen-4v5-Folien.pdf}    
 \end{figure}

\subsubsection{Vergleich FBS zu KOP und ST}

\begin{center}
\begin{tabular}{|l|c|c|c|}
\hline
\textbf{Merkmal} & \textbf{FBS} & \textbf{KOP} & \textbf{ST} \\ \hline
Darstellung & grafisch & grafisch & textuell \\ \hline
Binäre Logik & gut & sehr gut & gut \\ \hline
Analoge Werte & sehr gut & eingeschränkt & sehr gut \\ \hline
Übersichtlichkeit & hoch & hoch & mittel \\ \hline
Komplexe Algorithmen & eingeschränkt & schlecht & sehr gut \\ \hline
\end{tabular}
\end{center}

\subsubsection{Vor- und Nachteile von FBS}

\paragraph{Vorteile}
\begin{itemize}
    \item Sehr anschauliche Darstellung
    \item Ideal für Signalverarbeitung
    \item Gute Diagnosemöglichkeiten
\end{itemize}

\paragraph{Nachteile}
\begin{itemize}
    \item Bei großen Programmen schnell unübersichtlich
    \item Schleifen und komplexe Abläufe schwer darstellbar
\end{itemize}

\subsubsection{Einordnung}

FBS eignet sich besonders für:
\begin{itemize}
    \item Signalflussorientierte Aufgaben
    \item Analogwertverarbeitung
    \item Kombination mit KOP und ST
\end{itemize}

\newpage
\subsection{KOP}

\subsubsection{Grundprinzip}

Der Kontaktplan (KOP, engl. Ladder Diagram – LD) ist eine grafische
Programmiersprache nach IEC~61131-3.
Er lehnt sich an die Darstellung klassischer Stromlaufpläne an
und kann als um 90° gegen den Uhrzeigersinn gedrehter Stromlaufplan
interpretiert werden.

Eine KOP-Zeile (horizontal) entspricht einem Strompfad (vertikal)
im elektrischen Stromlaufplan.

\begin{itemize}
    \item Sehr gut lesbar für Elektrotechniker
    \item Besonders geeignet für binäre Logik
    \item Weit verbreitet im industriellen Umfeld
\end{itemize}

\subsubsection{Virtueller Stromfluss}

Im KOP existiert kein realer Strom, sondern ein \textbf{virtueller Stromfluss}.
Dieser fließt:
\begin{itemize}
    \item immer von links nach rechts
    \item innerhalb eines Netzwerks
\end{itemize}

Der virtuelle Stromfluss repräsentiert den logischen Zustand TRUE.

\subsubsection{Struktur eines KOP-Netzwerks}

Ein KOP-Programm besteht aus mehreren Netzwerken.
Jedes Netzwerk wird separat ausgewertet.

\paragraph{Linke Stromschiene}
Die linke Stromschiene liefert den logischen Wert TRUE
und ist Startpunkt jedes Netzwerks.

\paragraph{Rechte Stromschiene}
Die rechte Stromschiene schließt das Netzwerk ab.

\paragraph{Horizontale Linien}
Horizontale Linien übertragen den logischen Zustand
von einem Element zum nächsten.

\paragraph{Vertikale Linien}
Vertikale Linien verbinden mehrere horizontale Linien
und implizieren eine logische ODER-Verknüpfung.

\subsubsection{Berechnungsteil und Zuweisungsteil}

Ein KOP-Netzwerk lässt sich logisch in zwei Bereiche gliedern:
\begin{itemize}
    \item \textbf{Berechnungsteil (links)}:  
    Kontakte lesen Variablen und verknüpfen sie logisch.
    
    \item \textbf{Zuweisungsteil (rechts)}:  
    Spulen weisen den berechneten Wert einer Variablen zu.
\end{itemize}

\subsubsection{Kontakte}

Kontakte lesen den Wert einer Variablen
und beeinflussen den virtuellen Stromfluss.
Der Wert der Variablen wird durch Kontakte \textbf{nicht verändert}.

\paragraph{Schließer (NO)}
Der Kontakt leitet Strom, wenn die Variable TRUE ist.

Logisch:
\[
Rechts := Links \land Var
\]

\paragraph{Öffner (NC)}
Der Kontakt invertiert den Variablenwert.

Logisch:
\[
Rechts := Links \land \lnot Var
\]

\subsubsection{Flankenkontakte}

Flankenkontakte liefern TRUE nur für einen SPS-Zyklus
bei einer Zustandsänderung der Variable.

\paragraph{Positive Flanke}
TRUE bei Übergang von FALSE auf TRUE.

\paragraph{Negative Flanke}
TRUE bei Übergang von TRUE auf FALSE.

\paragraph{Positive und negative Flanke}
TRUE bei jeder Zustandsänderung.

Flankenkontakte werden häufig zum Zählen,
Setzen oder Rücksetzen verwendet.

\subsubsection{Spulen}

Spulen weisen den Wert des virtuellen Stromflusses
einer Variablen zu.

\paragraph{Normale Spule}
\[
Var := Links
\]

\paragraph{Negierte Spule}
\[
Var := \lnot Links
\]

\subsubsection{Setz- und Rücksetzspulen}

\paragraph{Setzspule (S)}
Setzt eine Variable auf TRUE,
solange der virtuelle Stromfluss anliegt.
Der Zustand bleibt gespeichert,
bis eine Rücksetzspule aktiv wird.

\paragraph{Rücksetzspule (R)}
Setzt eine Variable auf FALSE,
solange der virtuelle Stromfluss anliegt.

Setz- und Rücksetzspulen realisieren Speicherwirkung
und sind Grundlage für Ablaufsteuerungen.

\subsubsection{Flankenspulen}

\paragraph{Positive Flankenspule}
Setzt die Variable für genau einen Zyklus auf TRUE,
wenn eine positive Flanke am Eingang auftritt.

\paragraph{Negative Flankenspule}
Setzt die Variable für genau einen Zyklus auf TRUE,
wenn eine negative Flanke am Eingang auftritt.

\paragraph{Beide Flanken}
Reagiert auf jede Zustandsänderung.

\subsubsection{Ausführungsreihenfolge}

\begin{itemize}
    \item Der Signalfluss innerhalb eines Netzwerks erfolgt von links nach rechts
    \item Netzwerke werden von oben nach unten abgearbeitet
    \item Kleinere Netzwerknummern vor größeren
\end{itemize}

Explizite Rückkopplungen innerhalb eines Netzwerks
sind nicht erlaubt und werden als Kurzschluss abgelehnt.

Implizite Rückkopplungen sind über Variablen
im nächsten SPS-Zyklus möglich.

\subsubsection{Programmflusssteuerung}

KOP erlaubt die Beeinflussung der Programmausführung durch:
\begin{itemize}
    \item Bedingte Sprünge
    \item Unbedingte Sprünge
    \item RETURN (Verlassen der POE)
\end{itemize}

Sprünge können auf Netzwerknummern oder Labels erfolgen.

\subsubsection{Verwendung von Funktionen und Funktionsbausteinen}

In KOP können Funktionen und Funktionsbausteine verwendet werden.

\begin{itemize}
    \item Funktionsbausteine benötigen Instanzen
    \item Funktionen besitzen kein Gedächtnis
    \item Analoge Werte nur über Block-Ein- und Ausgänge
\end{itemize}

Analoge Werte können nicht über Kontakte verarbeitet werden.

\subsubsection{Vor- und Nachteile von KOP}

\paragraph{Vorteile}
\begin{itemize}
    \item Sehr übersichtlich für binäre Logik
    \item Gute Diagnose durch Anzeige des Stromflusses
    \item Einfache Übertragung von Relaissteuerungen
\end{itemize}

\paragraph{Nachteile}
\begin{itemize}
    \item Numerische Logik nur über Bausteine
    \item Zeichen- und Stringverarbeitung umständlich
    \item Komplexe Algorithmen schwer darstellbar
\end{itemize}

\subsubsection{Einordnung}

KOP eignet sich besonders für:
\begin{itemize}
    \item Binäre Steuerungslogik
    \item Verriegelungen
    \item Klassische Maschinensteuerungen
\end{itemize}
\newpage
\subsection{AS}

\subsubsection{Grundprinzip}

Die Ablaufsprache (AS, engl. Sequential Function Chart – SFC)
ist eine grafische Programmiersprache nach IEC~61131-3
zur Beschreibung \textbf{sequentieller Abläufe}.
Sie eignet sich besonders zur Darstellung von Ablaufsteuerungen
mit klar definierten Zuständen und Übergängen.

SFC basiert auf dem Zustandsmodell und trennt explizit zwischen:
\begin{itemize}
    \item Schritten (Zustände)
    \item Transitionen (Übergangsbedingungen)
    \item Aktionen (auszuführende Operationen)
\end{itemize}

\subsubsection{Schritte}

Ein Schritt repräsentiert einen Zustand des Steuerungsablaufs.
Zu jedem Zeitpunkt ist mindestens ein Schritt aktiv.

\begin{itemize}
    \item Schritte werden als Rechtecke dargestellt
    \item Aktive Schritte sind logisch TRUE
    \item Ein Initialschritt ist beim Start aktiv
\end{itemize}

Ein Schritt selbst führt keine Logik aus,
sondern aktiviert zugeordnete Aktionen.

\subsubsection{Transitionen}

Transitionen beschreiben die Bedingung für den Übergang
von einem oder mehreren Schritten zu einem oder mehreren Folgeschritten.

\begin{itemize}
    \item Transitionen werden logisch ausgewertet
    \item Sie besitzen ausschließlich BOOL-Ergebnisse
    \item Ein Übergang erfolgt nur, wenn alle vorgeschalteten Schritte aktiv sind
\end{itemize}

\medskip
\textbf{Regel:}  
Ein Schritt wird verlassen, wenn die zugehörige Transition TRUE wird.

\subsubsection{Aktivierungs- und Deaktivierungsregeln}

\begin{itemize}
    \item Wird eine Transition TRUE, werden die vorherigen Schritte deaktiviert
    \item Die nachfolgenden Schritte werden gleichzeitig aktiviert
    \item Mehrere Schritte können parallel aktiv sein
\end{itemize}

SFC erlaubt damit auch parallele Abläufe.

\subsubsection{Aktionen}

Aktionen sind die eigentlichen Programmanweisungen
und werden einem Schritt zugeordnet.
Sie können in KOP, FBS, ST oder AWL implementiert sein.

Eine Aktion wird ausgeführt, solange ihr Schritt aktiv ist
und die Aktionsbedingung erfüllt ist.

\subsubsection{Aktionsblöcke}

Aktionen werden in Aktionsblöcken dargestellt.
Ein Aktionsblock besteht aus:
\begin{itemize}
    \item einem Bestimmungszeichen (Qualifier)
    \item optional einer Zeitangabe
    \item dem Aktionsnamen oder einer BOOL-Variablen
\end{itemize}

\subsubsection{Aktionsqualifier}

Der Qualifier bestimmt das zeitliche Verhalten einer Aktion.

\begin{itemize}
    \item \textbf{N} (Normal):  
    Aktion ist aktiv, solange der Schritt aktiv ist
    
    \item \textbf{S} (Set):  
    Aktion wird beim Aktivieren des Schrittes gesetzt
    und bleibt aktiv, bis sie explizit zurückgesetzt wird
    
    \item \textbf{R} (Reset):  
    Setzt eine zuvor gesetzte Aktion zurück
    
    \item \textbf{L} (Limit):  
    Aktion ist nur für eine begrenzte Zeit aktiv
    
    \item \textbf{D} (Delay):  
    Aktion wird zeitverzögert aktiviert
\end{itemize}

Zeitqualifier verwenden entweder konstante Zeiten
oder Variablen vom Typ \texttt{TIME}.

\subsubsection{Parallele und alternative Abläufe}

SFC unterstützt:
\begin{itemize}
    \item alternative Verzweigungen (ODER-Verzweigung)
    \item parallele Verzweigungen (UND-Verzweigung)
\end{itemize}

Bei parallelen Abläufen müssen alle parallelen Zweige
abgeschlossen sein, bevor der Ablauf fortgesetzt wird.

\subsubsection{Einbindung in das SPS-Zyklusmodell}

\begin{itemize}
    \item Schritte und Transitionen werden zyklisch ausgewertet
    \item Aktionen werden innerhalb des SPS-Zyklus ausgeführt
    \item Zeitfunktionen basieren auf der Zykluszeit
\end{itemize}

SFC selbst beschreibt nur die Ablaufstruktur,
nicht die konkrete Ausführungslogik der Aktionen.

\subsubsection{Vor- und Nachteile von SFC}

\paragraph{Vorteile}
\begin{itemize}
    \item Sehr übersichtliche Darstellung komplexer Abläufe
    \item Klare Trennung von Zustand und Aktion
    \item Sehr gut geeignet für Ablaufsteuerungen
\end{itemize}

\paragraph{Nachteile}
\begin{itemize}
    \item Für reine Logik ungeeignet
    \item Zusätzlicher Implementierungsaufwand für Aktionen
\end{itemize}

\subsubsection{Einordnung}

SFC eignet sich besonders für:
\begin{itemize}
    \item Schrittketten
    \item Ablauf- und Prozesssteuerungen
    \item Strukturierung komplexer Programme
\end{itemize}

In der Praxis wird SFC häufig mit KOP, FBS oder ST kombiniert,
wobei SFC die Ablaufstruktur vorgibt
und die Aktionen in anderen Sprachen implementiert werden.

%%%%%%%%%%%%%%%%%%%%%%%%%%%%%%%%%%%%%%%%%%%%%%%%%%%%%%%%%%%%%%%%%%%%%%%%%%%%%%%%%%%%%%%%%%%%%%%

\subsection{Vergleich der Programmiersprachen anhand von Hochsprachenstrukturen}

In diesem Abschnitt werden typische Hochsprachenstrukturen
systematisch den IEC~61131-3-Sprachen gegenübergestellt.
Dadurch wird klar ersichtlich, welche Sprachmittel sich für
welche Aufgaben eignen.

\subsubsection{IF -- THEN -- ELSE -- ELSIF}

\paragraph{Struktur}
\begin{verbatim}
IF cond1 THEN
    x := 1;
ELSIF cond2 THEN
    x := 2;
ELSE
    x := 0;
END_IF;
\end{verbatim}

\paragraph{ST}
\begin{verbatim}
IF cond1 THEN
    x := 1;
ELSIF cond2 THEN
    x := 2;
ELSE
    x := 0;
END_IF;
\end{verbatim}

\paragraph{AWL}
\begin{verbatim}
LD   cond1
JMPC L1
LD   cond2
JMPC L2
LD   0
ST   x
JMP  L_END
L1: LD 1
    ST x
    JMP L_END
L2: LD 2
    ST x
L_END:
\end{verbatim}

\paragraph{KOP}
\begin{itemize}
    \item Mehrere Netzwerke
    \item Parallele Zweige mit Verriegelung
    \item ELSE-Zweig über negierte Bedingungen
\end{itemize}

\paragraph{FBS}
\begin{itemize}
    \item Vergleichsblöcke (GT, EQ, LT)
    \item MUX- oder SELECT-Baustein
\end{itemize}

\paragraph{SFC}
\begin{itemize}
    \item Alternative Transitionen
    \item Je ein Schritt pro IF-Zweig
\end{itemize}

---

\subsubsection{CASE}

\paragraph{Struktur}
\begin{verbatim}
CASE state OF
    0: x := 0;
    1: x := 10;
    2: x := 20;
ELSE
    x := -1;
END_CASE;
\end{verbatim}

\paragraph{ST}
\begin{verbatim}
CASE state OF
    0: x := 0;
    1: x := 10;
    2: x := 20;
ELSE
    x := -1;
END_CASE;
\end{verbatim}

\paragraph{AWL}
\begin{verbatim}
LD state
EQ 0
JMPC S0
LD state
EQ 1
JMPC S1
LD state
EQ 2
JMPC S2
LD -1
ST x
JMP END
S0: LD 0
    ST x
    JMP END
S1: LD 10
    ST x
    JMP END
S2: LD 20
    ST x
END:
\end{verbatim}

\paragraph{KOP}
\begin{itemize}
    \item Parallele Vergleichsnetzwerke
    \item Gegenseitige Verriegelung notwendig
\end{itemize}

\paragraph{FBS}
\begin{itemize}
    \item MUX-Baustein
    \item state als Auswahlvariable
\end{itemize}

\paragraph{SFC}
\begin{itemize}
    \item Zustand entspricht direkt CASE-Wert
\end{itemize}

---

\subsubsection{WHILE -- DO}

\paragraph{Struktur}
\begin{verbatim}
WHILE x < 10 DO
    x := x + 1;
END_WHILE;
\end{verbatim}

\paragraph{ST}
\begin{verbatim}
WHILE x < 10 DO
    x := x + 1;
END_WHILE;
\end{verbatim}

\paragraph{AWL}
\begin{verbatim}
LOOP:
LD x
LT 10
JMPC BODY
JMP END
BODY:
LD x
ADD 1
ST x
JMP LOOP
END:
\end{verbatim}

\paragraph{KOP}
\begin{itemize}
    \item Nicht direkt möglich
    \item Nur über Zustandsmerker und Sprünge
\end{itemize}

\paragraph{FBS}
\begin{itemize}
    \item Nicht vorgesehen
    \item Zyklische Wiederholung über Rückkopplung
\end{itemize}

\paragraph{SFC}
\begin{itemize}
    \item Schleife über Rücksprung-Transition
\end{itemize}

---

\subsubsection{REPEAT -- UNTIL}

\paragraph{Struktur}
\begin{verbatim}
REPEAT
    x := x + 1;
UNTIL x >= 10
END_REPEAT;
\end{verbatim}

\paragraph{ST}
\begin{verbatim}
REPEAT
    x := x + 1;
UNTIL x >= 10
END_REPEAT;
\end{verbatim}

\paragraph{AWL}
\begin{verbatim}
LOOP:
LD x
ADD 1
ST x
LD x
GE 10
JMPC END
JMP LOOP
END:
\end{verbatim}

\paragraph{KOP / FBS}
\begin{itemize}
    \item Nur über Zustandslogik realisierbar
\end{itemize}

\paragraph{SFC}
\begin{itemize}
    \item Schritt bleibt aktiv bis Transition TRUE wird
\end{itemize}

---

\subsubsection{FOR}

\paragraph{Struktur}
\begin{verbatim}
FOR i := 1 TO 5 DO
    sum := sum + i;
END_FOR;
\end{verbatim}

\paragraph{ST}
\begin{verbatim}
FOR i := 1 TO 5 DO
    sum := sum + i;
END_FOR;
\end{verbatim}

\paragraph{AWL}
\begin{verbatim}
LD 1
ST i
LD 0
ST sum
LOOP:
LD sum
ADD i
ST sum
LD i
ADD 1
ST i
LD i
LE 5
JMPC LOOP
\end{verbatim}

\paragraph{KOP}
\begin{itemize}
    \item Sehr aufwendig
    \item Zähler + Vergleich + Rücksprung
\end{itemize}

\paragraph{FBS}
\begin{itemize}
    \item Zählerbaustein + Rückkopplung
\end{itemize}

\paragraph{SFC}
\begin{itemize}
    \item Schleifenstruktur mit Zählvariable
\end{itemize}

---

\subsubsection{Zusammenfassender Vergleich}

\begin{center}
\begin{tabular}{|l|c|c|c|c|c|}
\hline
\textbf{Struktur} & \textbf{ST} & \textbf{AWL} & \textbf{KOP} & \textbf{FBS} & \textbf{SFC} \\ \hline
IF / ELSE & sehr gut & möglich & umständlich & gut & gut \\ \hline
CASE & sehr gut & aufwendig & schlecht & sehr gut & sehr gut \\ \hline
WHILE & gut & möglich & nein & nein & gut \\ \hline
REPEAT & gut & möglich & nein & nein & gut \\ \hline
FOR & sehr gut & möglich & schlecht & schlecht & gut \\ \hline
\end{tabular}
\end{center}

\medskip
\textbf{Merksatz:}  
\emph{ST ist die einzige Sprache, die Hochsprachenstrukturen direkt und sauber abbildet.
KOP und FBS benötigen Zustands- und Speicherlogik, SFC bildet Abläufe strukturell ab.}

%%%%%%%%%%%%%%%%%%%%%%%%%%%%%%%%%%%%%%%%%%%%%%%%%%%%%%%%%%%%%%%%%%%%%%%%%%%%%%%%%%%%%%%%%%%%%%%

\subsection{Umsetzungsmuster}

In diesem Abschnitt werden typische Hochsprachenkonzepte
auf ihre praktische Umsetzung in den IEC~61131-3-Sprachen abgebildet.
Der Schwerpunkt liegt auf wiederkehrenden Mustern
für binäre Logik, Ablaufsteuerungen und Speicherwirkungen.

\subsubsection{Bedingte Logik (IF-Struktur)}

\paragraph{Ziel}
Ausführung einer Aktion nur bei erfüllter Bedingung.

\paragraph{ST}
\begin{verbatim}
IF cond THEN
    y := TRUE;
ELSE
    y := FALSE;
END_IF;
\end{verbatim}

\paragraph{KOP}
\begin{itemize}
    \item Schließerkontakt \texttt{cond}
    \item Normale Spule \texttt{y}
\end{itemize}

\paragraph{AWL}
\begin{verbatim}
LD  cond
ST  y
\end{verbatim}

\paragraph{FBS}
\begin{itemize}
    \item Vergleichs- oder BOOL-Signal
    \item Direkt auf Ausgang verschaltet
\end{itemize}

\paragraph{SFC}
\begin{itemize}
    \item Transition = Bedingung
    \item Aktion im Folgeschritt
\end{itemize}

---

\subsubsection{Speicher (Selbsthaltung)}

\paragraph{Ziel}
Zustand soll erhalten bleiben, auch wenn Bedingung nicht mehr anliegt.

\paragraph{ST}
\begin{verbatim}
IF set THEN
    mem := TRUE;
ELSIF reset THEN
    mem := FALSE;
END_IF;
\end{verbatim}

\paragraph{KOP}
\begin{itemize}
    \item Setzspule (S)
    \item Rücksetzspule (R)
\end{itemize}

\paragraph{AWL}
\begin{verbatim}
LD   set
JMPC SET
LD   reset
JMPC RESET
JMP  END
SET:   LD TRUE
       ST mem
       JMP END
RESET: LD FALSE
       ST mem
END:
\end{verbatim}

\paragraph{FBS}
\begin{itemize}
    \item RS- oder SR-Flipflop-Baustein
\end{itemize}

\paragraph{SFC}
\begin{itemize}
    \item Schritt selbst repräsentiert den Speicher
\end{itemize}

---

\subsubsection{Flankenerkennung}

\paragraph{Ziel}
Aktion soll nur bei Zustandsänderung ausgelöst werden.

\paragraph{ST}
\begin{verbatim}
edge := signal AND NOT signal_old;
signal_old := signal;
\end{verbatim}

\paragraph{KOP}
\begin{itemize}
    \item Positiver oder negativer Flankenkontakt
\end{itemize}

\paragraph{AWL}
\begin{verbatim}
LD   signal
ANDN signal_old
ST   edge
LD   signal
ST   signal_old
\end{verbatim}

\paragraph{FBS}
\begin{itemize}
    \item Flankenbaustein (\texttt{R\_TRIG} / \texttt{F\_TRIG})
\end{itemize}

\paragraph{SFC}
\begin{itemize}
    \item Transition mit Flankenbedingung
\end{itemize}

---

\subsubsection{Zähler}

\paragraph{Ziel}
Ereignisse zählen und auswerten.

\paragraph{ST}
\begin{verbatim}
IF edge THEN
    cnt := cnt + 1;
END_IF;
\end{verbatim}

\paragraph{KOP}
\begin{itemize}
    \item Zählerbaustein (CTU, CTD)
\end{itemize}

\paragraph{AWL}
\begin{verbatim}
LD edge
JMPC INC
JMP END
INC:
LD cnt
ADD 1
ST cnt
END:
\end{verbatim}

\paragraph{FBS}
\begin{itemize}
    \item CTU / CTD / CTUD
\end{itemize}

\paragraph{SFC}
\begin{itemize}
    \item Zähler in Aktionen
\end{itemize}

---

\subsubsection{Zeitverhalten}

\paragraph{Ziel}
Zeitabhängige Steuerung.

\paragraph{ST}
\begin{verbatim}
ton(IN := start, PT := T#5s);
done := ton.Q;
\end{verbatim}

\paragraph{KOP}
\begin{itemize}
    \item Zeitrelais (TON, TOF, TP)
\end{itemize}

\paragraph{AWL}
\begin{verbatim}
LD start
TON t1, PT := T#5s
\end{verbatim}

\paragraph{FBS}
\begin{itemize}
    \item Timer-Funktionsbausteine
\end{itemize}

\paragraph{SFC}
\begin{itemize}
    \item Zeitqualifier (L, D)
\end{itemize}

---

\subsubsection{Ablaufsteuerung}

\paragraph{Ziel}
Mehrere Schritte in definierter Reihenfolge.

\paragraph{ST}
\begin{verbatim}
CASE state OF
    0: state := 1;
    1: state := 2;
END_CASE;
\end{verbatim}

\paragraph{KOP}
\begin{itemize}
    \item Zustandsmerker
    \item Verriegelte Netzwerke
\end{itemize}

\paragraph{AWL}
\begin{itemize}
    \item Sprungmarken
    \item Zustandsvariable
\end{itemize}

\paragraph{FBS}
\begin{itemize}
    \item Zustandsvariable + Logik
\end{itemize}

\paragraph{SFC}
\begin{itemize}
    \item Natürliche Abbildung (Schritte/Transitionen)
\end{itemize}

---

\subsubsection{Zusammenfassung der Umsetzungsmuster}

\begin{center}
\begin{tabular}{|l|c|c|c|c|c|}
\hline
\textbf{Muster} & \textbf{ST} & \textbf{AWL} & \textbf{KOP} & \textbf{FBS} & \textbf{SFC} \\ \hline
IF / Bedingung & sehr gut & gut & sehr gut & gut & gut \\ \hline
Speicher & gut & gut & sehr gut & sehr gut & sehr gut \\ \hline
Flanke & gut & gut & sehr gut & sehr gut & gut \\ \hline
Zähler & gut & gut & sehr gut & sehr gut & gut \\ \hline
Zeit & sehr gut & gut & sehr gut & sehr gut & sehr gut \\ \hline
Ablauf & gut & gut & mittel & mittel & sehr gut \\ \hline
\end{tabular}
\end{center}

\medskip
\textbf{Merksatz:}  
\emph{ST bildet Logik und Algorithmen ab,
KOP/FBS bilden Signale ab,
SFC bildet Abläufe ab.}

%%%%%%%%%%%%%%%%%%%%%%%%%%%%%%%%%%%%%%%%%%%%%%%%%%%%%%%%%%%%%%%%%%%%%%%%%%%%%%%%%%%%%%%%%%%%%%%