\newpage
\section{Diskrete Steuerungen}

\subsection{Begriff der diskreten Steuerung}

Diskrete Steuerungen sind Steuerungen,
bei denen Ein- und Ausgangsgrößen
nur \textbf{diskrete Zustände} annehmen.
In der Praxis handelt es sich überwiegend
um binäre Zustände (0/1, FALSE/TRUE).

Die Verarbeitung erfolgt auf Basis
logischer Verknüpfungen, Speicherwirkungen
und zeitlicher Abfolgen.

\subsection{Hierarchische Abstraktion}
In der Abbildung soll ersichtlich werden wie die Begriffe zusammenhängen. Die Abstraktion nimmt nach unten hin zu.

\vspace*{1cm}
\begin{tikzpicture}[
    box/.style={
        draw,
        rectangle,
        rounded corners,
        align=center,
        minimum width=8cm,
        minimum height=1cm
    },
    node distance=1.2cm
]

% Hierarchie
\node[box] (se) {\textbf{Schaltelemente}\\
Relais, Kontakte, Transistoren};

\node[box, below=of se] (ls) {\textbf{Logische Schaltungen}\\
Boolesche Verknüpfungen};

\node[box, below=of ls] (sn) {\textbf{Schaltnetze}\\
speicherlose logische Systeme};

\node[box, below=of sn] (f) {\textbf{Funktionen}\\
abstrahierte Schaltnetze\\
(ein Ausgang, kein Speicher)};

\node[box, below=of f] (sw) {\textbf{Schaltwerke}\\
logische Systeme mit Speicher};

\node[box, below=of sw, thick] (fb) {\textbf{Funktionsbausteine}\\
gekapselte Schaltwerke};

% Pfeile
\draw[->] (se) -- (ls);
\draw[->] (ls) -- (sn);
\draw[->] (sn) -- (f);
\draw[->] (f) -- (sw);
\draw[->] (sw) -- (fb);

% Schaltsysteme-Klammer (Sammelbegriff)
\draw[dashed, thick, rounded corners]
    ($(sn.north west)+(-0.6,0.3)$) rectangle
    ($(sw.south east)+(0.6,-0.3)$);

\node[align=center] at ($(sn.west)+(-2.3,-1.2)$)
{\textbf{Schaltsysteme}\\
(Oberbegriff)};

\end{tikzpicture}

\subsection{Schaltelemente}

Schaltelemente sind die elementaren
Bausteine diskreter Steuerungen.
Sie besitzen klar definierte Eingänge
und einen oder mehrere Ausgänge.

\paragraph{Grundlegende Schaltelemente}
\begin{itemize}
    \item Schalter
    \item Relais
    \item Schütze
    \item Elektronische Schaltelemente (Transistoren)
\end{itemize}

Schaltelemente können Signale:
\begin{itemize}
    \item schalten
    \item verknüpfen
    \item speichern
\end{itemize}

\subsection{Grundlagen logischer Schaltungen}

Logische Schaltungen verarbeiten binäre Signale
nach den Regeln der Booleschen Algebra.

\paragraph{Logische Grundfunktionen}
\begin{itemize}
    \item TRUE / FALSE
    \item AND
    \item OR
    \item NOT
\end{itemize}

Diese Grundfunktionen können durch
Schaltelemente realisiert werden
(z.\,B. Reihenschaltung, Parallelschaltung).

\paragraph{Erweiterte logische Funktionen}
\begin{itemize}
    \item Exklusiv-ODER (XOR)
    \item NAND
    \item NOR
\end{itemize}

\subsection{Darstellung von Logikfunktionen durch Wertetabellen}

Eine Logikfunktion beschreibt den funktionalen Zusammenhang
zwischen binären Eingangsgrößen und einer binären Ausgangsgröße.
Die Eingänge und der Ausgang können ausschließlich die Werte
\texttt{0} (FALSE) oder \texttt{1} (TRUE) annehmen.

Logikfunktionen bilden die Grundlage aller diskreten Steuerungen
und werden mithilfe der Booleschen Algebra beschrieben.

\subsection{Boolesche Grundfunktionen}

\subsubsection{NICHT-Funktion (NOT)}

Die NICHT-Funktion invertiert den Eingang.

\[
y = \lnot a
\]

\paragraph{Wertetabelle}
\begin{center}
\begin{tabular}{|c|c|}
\hline
$a$ & $y = \lnot a$ \\ \hline
0 & 1 \\ \hline
1 & 0 \\ \hline
\end{tabular}
\end{center}

\subsubsection{UND-Funktion (AND)}

Die UND-Funktion liefert nur dann den Wert 1,
wenn \textbf{alle Eingänge} den Wert 1 besitzen.

\[
y = a \land b
\]

\paragraph{Wertetabelle}
\begin{center}
\begin{tabular}{|c|c|c|}
\hline
$a$ & $b$ & $y = a \land b$ \\ \hline
0 & 0 & 0 \\ \hline
0 & 1 & 0 \\ \hline
1 & 0 & 0 \\ \hline
1 & 1 & 1 \\ \hline
\end{tabular}
\end{center}

\subsubsection{ODER-Funktion (OR)}

Die ODER-Funktion liefert den Wert 1,
wenn \textbf{mindestens ein Eingang} den Wert 1 besitzt.

\[
y = a \lor b
\]

\paragraph{Wertetabelle}
\begin{center}
\begin{tabular}{|c|c|c|}
\hline
$a$ & $b$ & $y = a \lor b$ \\ \hline
0 & 0 & 0 \\ \hline
0 & 1 & 1 \\ \hline
1 & 0 & 1 \\ \hline
1 & 1 & 1 \\ \hline
\end{tabular}
\end{center}

\subsection{Erweiterte Logikfunktionen}

\subsubsection{Exklusiv-ODER-Funktion (XOR)}

Die Exklusiv-ODER-Funktion liefert genau dann den Wert 1,
wenn die Eingänge \textbf{unterschiedlich} sind.

\[
y = a \oplus b
\]

\paragraph{Wertetabelle}
\begin{center}
\begin{tabular}{|c|c|c|}
\hline
$a$ & $b$ & $y = a \oplus b$ \\ \hline
0 & 0 & 0 \\ \hline
0 & 1 & 1 \\ \hline
1 & 0 & 1 \\ \hline
1 & 1 & 0 \\ \hline
\end{tabular}
\end{center}

\subsubsection{NAND-Funktion}

Die NAND-Funktion ist die Negation der UND-Funktion.

\[
y = \lnot (a \land b)
\]

\paragraph{Wertetabelle}
\begin{center}
\begin{tabular}{|c|c|c|}
\hline
$a$ & $b$ & $y$ \\ \hline
0 & 0 & 1 \\ \hline
0 & 1 & 1 \\ \hline
1 & 0 & 1 \\ \hline
1 & 1 & 0 \\ \hline
\end{tabular}
\end{center}

\subsubsection{NOR-Funktion}

Die NOR-Funktion ist die Negation der ODER-Funktion.

\[
y = \lnot (a \lor b)
\]

\paragraph{Wertetabelle}
\begin{center}
\begin{tabular}{|c|c|c|}
\hline
$a$ & $b$ & $y$ \\ \hline
0 & 0 & 1 \\ \hline
0 & 1 & 0 \\ \hline
1 & 0 & 0 \\ \hline
1 & 1 & 0 \\ \hline
\end{tabular}
\end{center}

\subsubsection{Zusammenhang zu Schaltelementen}

Logikfunktionen können physikalisch
durch Schaltelemente realisiert werden:

\begin{itemize}
    \item UND-Funktion → Reihenschaltung von Kontakten
    \item ODER-Funktion → Parallelschaltung von Kontakten
    \item NICHT-Funktion → Öffnerkontakt
\end{itemize}

Diese Abbildungen bilden die Grundlage
für die Darstellung in KOP
sowie für elektrische Steuerungen.

\subsubsection{Zusammenfassung}

Logikfunktionen beschreiben die grundlegenden Zusammenhänge
binärer Steuerungssysteme.
Sie sind Grundlage für Schaltnetze,
Schaltwerke und Funktionsbausteine.

\medskip
\textbf{Merksatz:}  
\emph{Jede diskrete Steuerung lässt sich auf Logikfunktionen zurückführen.}

\subsection{Minterme und Maxterme}

\subsubsection{Grundidee}

Minterme und Maxterme sind standardisierte Darstellungsformen
logischer Funktionen auf Basis von Wertetabellen.
Sie ermöglichen eine systematische und eindeutige Beschreibung
beliebiger Logikfunktionen mithilfe der Booleschen Algebra.

\subsubsection{Minterm}

Ein \textbf{Minterm} ist eine logische UND-Verknüpfung
aller Eingangsvariablen oder ihrer Negationen,
die \textbf{genau für eine bestimmte Eingangskombination den Wert 1} liefert.

\paragraph{Eigenschaften eines Minterms}
\begin{itemize}
    \item Enthält alle Eingangsvariablen genau einmal
    \item Nicht negierte Variable entspricht Eingang = 1
    \item Negierte Variable entspricht Eingang = 0
    \item Ergebnis ist nur für eine Kombination TRUE
\end{itemize}

\paragraph{Beispiel}

Für zwei Eingänge $a$ und $b$:

\[
m_2 = a \land \lnot b
\]

Dieser Minterm ist genau dann TRUE, wenn:
\[
a = 1 \;\land\; b = 0
\]

\subsubsection{Minterme aus der Wertetabelle}

Gegeben sei folgende Wertetabelle:

\begin{center}
\begin{tabular}{|c|c|c|}
\hline
$a$ & $b$ & $y$ \\ \hline
0 & 0 & 0 \\ \hline
0 & 1 & 1 \\ \hline
1 & 0 & 1 \\ \hline
1 & 1 & 0 \\ \hline
\end{tabular}
\end{center}

Die Funktion ist für die Eingangskombinationen
$(0,1)$ und $(1,0)$ gleich 1.

Daraus ergeben sich die Minterme:
\[
m_1 = \lnot a \land b
\]
\[
m_2 = a \land \lnot b
\]

\paragraph{Disjunktive Normalform (DNF)}

Die vollständige Darstellung der Funktion lautet:
\[
y = m_1 \lor m_2
\]

\medskip
\textbf{Merksatz:}  
\emph{Minterme beschreiben alle Eingangskombinationen,
für die die Funktion den Wert 1 annimmt.}

\subsubsection{Maxterm}

Ein \textbf{Maxterm} ist eine logische ODER-Verknüpfung
aller Eingangsvariablen oder ihrer Negationen,
die \textbf{genau für eine bestimmte Eingangskombination den Wert 0} liefert.

\paragraph{Eigenschaften eines Maxterms}
\begin{itemize}
    \item Enthält alle Eingangsvariablen genau einmal
    \item Nicht negierte Variable entspricht Eingang = 0
    \item Negierte Variable entspricht Eingang = 1
    \item Ergebnis ist nur für eine Kombination FALSE
\end{itemize}

\paragraph{Beispiel}

\[
M_2 = a \lor \lnot b
\]

Dieser Maxterm ist genau dann FALSE, wenn:
\[
a = 0 \;\land\; b = 1
\]

\subsubsection{Maxterme aus der Wertetabelle}

Aus der obigen Wertetabelle ergibt sich:
Die Funktion ist für die Eingangskombinationen
$(0,0)$ und $(1,1)$ gleich 0.

Daraus ergeben sich die Maxterme:
\[
M_0 = a \lor b
\]
\[
M_3 = \lnot a \lor \lnot b
\]

\paragraph{Konjunktive Normalform (KNF)}

Die vollständige Darstellung der Funktion lautet:
\[
y = M_0 \land M_3
\]

\medskip
\textbf{Merksatz:}  
\emph{Maxterme beschreiben alle Eingangskombinationen,
für die die Funktion den Wert 0 annimmt.}

\subsubsection{Vergleich Minterm und Maxterm}

\begin{center}
\begin{tabular}{|l|c|c|}
\hline
\textbf{Merkmal} & \textbf{Minterm} & \textbf{Maxterm} \\ \hline
Basierend auf & $y = 1$ & $y = 0$ \\ \hline
Verknüpfung & UND & ODER \\ \hline
Normalform & DNF & KNF \\ \hline
Negation & bei Eingängen mit 0 & bei Eingängen mit 1 \\ \hline
\end{tabular}
\end{center}

\subsubsection{Bedeutung für diskrete Steuerungen}

Minterme und Maxterme bilden die theoretische Grundlage für:
\begin{itemize}
    \item systematische Schaltungsentwürfe
    \item Vereinfachung logischer Funktionen
    \item Umsetzung von Schaltnetzen
    \item Analyse von Steuerungslogik
\end{itemize}

\medskip
\textbf{Zusammenfassender Merksatz:}  
\emph{Minterme führen von der Wertetabelle zur logischen Gleichung,
Maxterme von der logischen Gleichung zur Wertetabelle.}

\begin{figure}[H]
    \centering
    \includegraphics[page=16,width=0.8\textwidth]{/Users/danielweindl/_source/Repositorys/STT-Lernskript/Data/sttvo-k02-diskrete Steuerungen-4v2-Folien_221025.pdf}    
 \end{figure}

\subsection{Schaltnetze}

Schaltnetze sind \textbf{speicherlose}
Schaltsysteme.

\paragraph{Eigenschaft}
Der Ausgang eines Schaltnetzes
hängt ausschließlich von den
aktuellen Eingangswerten ab.

\[
a = f(e)
\]

\paragraph{Merkmale}
\begin{itemize}
    \item Keine Speicherwirkung
    \item Keine Zustandsabhängigkeit
    \item Rein kombinatorisches Verhalten
\end{itemize}

\paragraph{Beispiele}
\begin{itemize}
    \item Logische Verriegelungen
    \item Freigabeschaltungen
    \item Sicherheitsabfragen
\end{itemize}

\begin{figure}[H]
    \centering
    \includegraphics[page=37,width=0.8\textwidth]{/Users/danielweindl/_source/Repositorys/STT-Lernskript/Data/sttvo-k02-diskrete Steuerungen-4v2-Folien_221025.pdf}    
\end{figure}

\subsection{Funktionen}

Funktionen sind abstrahierte logische
Zusammenfassungen von Schaltnetzen.

Sie besitzen:
\begin{itemize}
    \item definierte Eingänge
    \item genau einen Ausgang
    \item keine Speicherwirkung
\end{itemize}

Funktionen beschreiben damit
eine reine logische Abbildung
von Eingang auf Ausgang.

\subsection{Schaltwerke}

Schaltwerke sind \textbf{schaltende Systeme mit Speicherwirkung}.

\paragraph{Eigenschaft}
Der Ausgang hängt von:
\begin{itemize}
    \item den aktuellen Eingängen
    \item dem internen Zustand
\end{itemize}
ab.

\[
a = f(e, s)
\]

\paragraph{Merkmale}
\begin{itemize}
    \item Speicherwirkung vorhanden
    \item Zustandsabhängiges Verhalten
    \item Grundlage für Ablaufsteuerungen
\end{itemize}

\paragraph{Typische Speicher}
\begin{itemize}
    \item Selbsthaltung
    \item Flip-Flops (RS, SR)
    \item Merker
\end{itemize}


\begin{figure}[H]
    \centering
    \includegraphics[page=48,width=0.8\textwidth]{/Users/danielweindl/_source/Repositorys/STT-Lernskript/Data/sttvo-k02-diskrete Steuerungen-4v2-Folien_221025.pdf}    
\end{figure}
\begin{figure}[H]
    \centering
    \includegraphics[page=49,width=0.8\textwidth]{/Users/danielweindl/_source/Repositorys/STT-Lernskript/Data/sttvo-k02-diskrete Steuerungen-4v2-Folien_221025.pdf}    
\end{figure}
\begin{figure}[H]
    \centering
    \includegraphics[page=50,width=0.8\textwidth]{/Users/danielweindl/_source/Repositorys/STT-Lernskript/Data/sttvo-k02-diskrete Steuerungen-4v2-Folien_221025.pdf}    
\end{figure}
\begin{figure}[H]
    \centering
    \includegraphics[page=51,width=0.8\textwidth]{/Users/danielweindl/_source/Repositorys/STT-Lernskript/Data/sttvo-k02-diskrete Steuerungen-4v2-Folien_221025.pdf}    
\end{figure}
\begin{figure}[H]
    \centering
    \includegraphics[page=52,width=0.8\textwidth]{/Users/danielweindl/_source/Repositorys/STT-Lernskript/Data/sttvo-k02-diskrete Steuerungen-4v2-Folien_221025.pdf}    
\end{figure}

\subsection{Schaltsysteme}

Ein Schaltsystem ist die
\textbf{Zusammenfassung mehrerer Schaltelemente}
zu einer funktionalen Einheit.

Schaltsysteme lassen sich grundsätzlich
in zwei Klassen einteilen:
\begin{itemize}
    \item Schaltnetze
    \item Schaltwerke
\end{itemize}

Diese Unterscheidung ist grundlegend
für das Verständnis diskreter Steuerungen.
\begin{figure}[H]
    \centering
    \includegraphics[page=5,width=0.8\textwidth]{/Users/danielweindl/_source/Repositorys/STT-Lernskript/Data/sttvo-k02-diskrete Steuerungen-4v2-Folien_221025.pdf}    
\end{figure}

\subsection{Funktionsbausteine}

Funktionsbausteine sind
\textbf{gekapselte Schaltwerke}
mit definierten Schnittstellen.

Sie kombinieren:
\begin{itemize}
    \item logische Verknüpfungen
    \item Speicherfunktionen
    \item gegebenenfalls Zeitfunktionen
\end{itemize}

\paragraph{Eigenschaften}
\begin{itemize}
    \item Mehrere Ein- und Ausgänge
    \item Interner Zustand
    \item Wiederverwendbarkeit
\end{itemize}

\paragraph{Beispiele}
\begin{itemize}
    \item Zeitglieder
    \item Zähler
    \item Speicherbausteine
\end{itemize}

\subsection{Zusammenhang der Begriffe}

\begin{center}
\begin{tabular}{|l|l|}
\hline
\textbf{Begriff} & \textbf{Charakteristik} \\ \hline
Schaltelement & Physikalisches Grundelement \\ \hline
Logische Schaltung & Logische Verknüpfung \\ \hline
Schaltnetz & Speicherloses System \\ \hline
Funktion & Abstraktes Schaltnetz \\ \hline
Schaltwerk & System mit Speicher \\ \hline
Funktionsbaustein & Gekapseltes Schaltwerk \\ \hline
\end{tabular}
\end{center}

\medskip
\textbf{Merksatz:}  
\emph{Schaltnetze kennen keinen Zustand,
Schaltwerke benötigen Speicher.}

\subsection{Zahlensysteme}

\begin{figure}[H]
    \centering
    \includegraphics[page=24,width=0.8\textwidth]{/Users/danielweindl/_source/Repositorys/STT-Lernskript/Data/sttvo-k02-diskrete Steuerungen-4v2-Folien_221025.pdf}    
\end{figure}\begin{figure}[H]
    \centering
    \includegraphics[page=26,width=0.8\textwidth]{/Users/danielweindl/_source/Repositorys/STT-Lernskript/Data/sttvo-k02-diskrete Steuerungen-4v2-Folien_221025.pdf}    
\end{figure}
