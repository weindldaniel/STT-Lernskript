\section{Einleitung}

\subsection{Einordnung der Steuerungstechnik}

Die Steuerungstechnik ist eine Teildisziplin der Automatisierungstechnik und beschäftigt sich
mit dem gezielten Beeinflussen technischer Prozesse durch vorgegebene Vorschriften,
Algorithmen oder Gesetzmäßigkeiten.
Sie umfasst den \textbf{Entwurf}, die \textbf{Realisierung}, das \textbf{Testen},
die \textbf{Inbetriebnahme} sowie die \textbf{Wartung} von Steuerungssystemen.

Ein technischer Prozess ist ein Vorgang zur \textbf{Umformung}, \textbf{Speicherung}
oder zum \textbf{Transport} von Material, Energie oder Information.
Der Zustand eines Prozesses wird durch sogenannte \emph{Zustandsgrößen} beschrieben,
die technisch erfassbar und beeinflussbar sind.

Die Automatisierungstechnik gliedert sich unter anderem in:
\begin{itemize}
    \item Sensorik (Informationsgewinnung)
    \item Informationsverarbeitung (z.\,B. SPS, Industrie-PC)
    \item Aktorik (Informationsumsetzung)
    \item Mensch-Maschine-Schnittstelle (Visualisierung, Bedienung)
\end{itemize}

\subsection{Steuern und Regeln}

\paragraph{Steuern}
Unter Steuern versteht man einen Vorgang, bei dem Eingangsgrößen
Ausgangsgrößen eines Systems beeinflussen, \textbf{ohne dass die Ausgangsgrößen
rückgeführt und fortlaufend verglichen werden}.
Kennzeichnend ist der \textbf{offene Wirkungsablauf}.

Die Steuerungstechnik ist die gezielte Anwendung von Steuerungsmechanismen
(Vorschriften, Algorithmen, Gesetzmäßigkeiten) unter Zuhilfenahme der verfügbaren
technischen Mittel zur zielgerichteten Erfüllung von Aufgaben in Prozessen.

\paragraph{Regeln}
Beim Regeln wird eine Regelgröße kontinuierlich erfasst,
mit einer Führungsgröße verglichen und in Richtung dieser Führungsgröße beeinflusst.
Kennzeichnend ist der \textbf{geschlossene Wirkungsablauf (Rückkopplung)}.

\medskip
\textbf{Merksatz:}  
\emph{Steuern arbeitet ohne Rückführung, Regeln immer mit Rückführung.}

\subsection{Offene und geschlossene Steuerungen}

\begin{itemize}
    \item \textbf{Offene Steuerung}:  
    Keine Rückmeldung aus dem Prozess.  
    Beispiel: Zeitgesteuerte Ampel.
    
    \item \textbf{Geschlossene Steuerung}:  
    Rückmeldung beeinflusst den Steuerungsablauf, jedoch nicht kontinuierlich.  
    Beispiel: Abschaltung über Endschalter.
    
    \item \textbf{Regelung}:  
    Kontinuierliche Rückführung und Korrektur.  
    Beispiel: Drehzahlregelung eines Motors.
\end{itemize}

 
\begin{figure}[H]
    \centering
    \includegraphics[page=12,width=0.8\textwidth]{/Users/danielweindl/_source/Repositorys/STT-Lernskript/Data/sttvo-k01-Einführung-4v3-Folien_011025.pdf}    
 \end{figure}


\subsection{Grundstruktur eines Steuerungssystems}

Ein Steuerungssystem besteht grundsätzlich aus:
\begin{itemize}
    \item Führungsgrößen (z.\,B. Start, Stop, Sollwerte)
    \item Steuereinrichtung (z.\,B. SPS)
    \item Steuerstrecke (technischer Prozess)
    \item Stellgrößen (Ausgangsgrößen der Steuerung)
    \item Meldegrößen (Rückmeldungen, Anzeigen)
    \item Störgrößen (unerwünschte Einflüsse)
\end{itemize}

\subsection{Klassifikation von Steuerungen}

\subsubsection{Nach Informationsdarstellung}

\begin{itemize}
    \item \textbf{Analoge Steuerungen}:  
    Stetige Signalverarbeitung (z.\,B. Spannungen, Ströme)
    
    \item \textbf{Digitale Steuerungen}:  
    Verarbeitung von Zahlenwerten und Codes
    
    \item \textbf{Binäre Steuerungen}:  
    Verarbeitung logischer Zustände (TRUE/FALSE)
\end{itemize}

\begin{figure}[H]
    \centering
    \includegraphics[width=0.6\textwidth]{/Users/danielweindl/_source/Repositorys/STT-Lernskript/Bilder/BinäreSignalverarbeitung.png}
\end{figure}

\subsubsection{Nach Art der Signalverarbeitung}
\begin{figure}[H]
    \centering
    \includegraphics[width=0.8\textwidth]{/Users/danielweindl/_source/Repositorys/STT-Lernskript/Bilder/ArtenSignalverarbeitung.png}
\end{figure}

\begin{itemize}
    \item \textbf{Synchrone Steuerungen}:  
    Abtastung und Verarbeitung im festen Zeitraster
    
    \item \textbf{Asynchrone Steuerungen}:  
    Ereignisgesteuerte Verarbeitung
    
\begin{figure}[H]
    \centering
    \includegraphics[width=1\textwidth]{/Users/danielweindl/_source/Repositorys/STT-Lernskript/Bilder/SynchronAsynchron.png}
\end{figure}

    \item \textbf{Verknüpfungssteuerungen}:  
    Ausgang hängt nur vom aktuellen Eingang ab
    
    \item \textbf{Ablaufsteuerungen}:  
    Ausgang hängt von Eingang \emph{und} Zustand ab
\end{itemize}

\subsection{Kategorien von Steuerungen}

Die Realisierung einer Steuerung beschreibt,
\textbf{mit welchen technischen Mitteln}
die Steuerungsfunktion umgesetzt wird.
Man unterscheidet mehrere grundlegende Realisierungsarten,
die historisch und technisch unterschiedliche Eigenschaften besitzen.

\begin{figure}[H]
    \centering
    \includegraphics[width=0.9\textwidth]{/Users/danielweindl/_source/Repositorys/STT-Lernskript/Bilder/UnterscheidungEnergieübertragung.png}
\end{figure}


\subsubsection{Mechanische Steuerungen}

Mechanische Steuerungen realisieren Steuerungsfunktionen
ausschließlich durch mechanische Elemente.
Die Steuerinformation wird durch Kräfte, Wege oder Bewegungen erzeugt.

\paragraph{Typische Elemente}
\begin{itemize}
    \item Anschläge
    \item Nocken
    \item Kurvenscheiben
    \item Gestänge
\end{itemize}

\paragraph{Eigenschaften}
\begin{itemize}
    \item Kein elektrischer Energiebedarf
    \item Ablauf direkt an Bewegung gekoppelt
    \item Sehr robust
    \item Kaum flexibel
\end{itemize}

\paragraph{Beispiele}
\begin{itemize}
    \item Waschmaschinen mit mechanischem Programmwerk
    \item Nockensteuerungen in Werkzeugmaschinen
    \item Drehorgeln, Spieluhren
\end{itemize}

\paragraph{Einordnung}
Mechanische Steuerungen sind historisch bedeutsam,
werden heute jedoch kaum noch neu eingesetzt,
da Änderungen nur mechanisch möglich sind.

\subsubsection{Pneumatische Steuerungen}

Pneumatische Steuerungen nutzen Druckluft
zur Signalübertragung und zur Energieversorgung der Aktoren.

\paragraph{Typische Elemente}
\begin{itemize}
    \item Pneumatikzylinder
    \item Wegeventile
    \item Drosseln
    \item Pneumatische Logikelemente
\end{itemize}

\paragraph{Eigenschaften}
\begin{itemize}
    \item Schnelle Schaltzeiten
    \item Relativ einfache Logik möglich
    \item Begrenzte Genauigkeit
    \item Kompressibilität der Luft
\end{itemize}

\paragraph{Beispiele}
\begin{itemize}
    \item Handhabungssysteme
    \item Einfache Montageautomaten
\end{itemize}

\paragraph{Einordnung}
Pneumatische Steuerungen werden heute meist
in Kombination mit SPS eingesetzt
(SPS steuert Ventile, Pneumatik führt aus).

 

\subsubsection{Hydraulische Steuerungen}

Hydraulische Steuerungen verwenden
nicht oder kaum kompressible Flüssigkeiten
zur Kraft- und Signalübertragung.

\paragraph{Typische Elemente}
\begin{itemize}
    \item Hydraulikzylinder
    \item Steuerventile
    \item Pumpen
    \item Druckspeicher
\end{itemize}

\paragraph{Eigenschaften}
\begin{itemize}
    \item Sehr hohe Stellkräfte möglich
    \item Gute Regelbarkeit
    \item Hoher technischer Aufwand
    \item Wartungsintensiv
\end{itemize}

\paragraph{Beispiele}
\begin{itemize}
    \item Pressen
    \item Spritzgussmaschinen
\end{itemize}

\paragraph{Einordnung}
Hydraulische Steuerungen sind heute
meist Teil der Steuerstrecke,
während die Steuerlogik elektrisch oder per SPS realisiert wird.

 

\subsubsection{Elektrische Steuerungen}

Elektrische Steuerungen realisieren die Steuerungsfunktion
durch elektrische Schaltgeräte und Leitungsverknüpfungen.

\paragraph{Typische Elemente}
\begin{itemize}
    \item Relais
    \item Schütze
    \item Schalter
    \item Zeitrelais
\end{itemize}

\paragraph{Eigenschaften}
\begin{itemize}
    \item Binäre Logik gut realisierbar
    \item Hohe Übersichtlichkeit im Stromlaufplan
    \item Änderungen nur durch Umverdrahtung
    \item Mechanischer Verschleiß
\end{itemize}

\paragraph{Beispiele}
\begin{itemize}
    \item Motorsteuerungen
    \item Aufzugssteuerungen (klassisch)
\end{itemize}

\paragraph{Einordnung}
Elektrische Steuerungen sind die direkte
\textbf{Vorgänger-Technologie der SPS}
und Grundlage für KOP-Darstellungen.

 

\subsubsection{Elektronische Steuerungen}

Elektronische Steuerungen verwenden
kontaktlose elektronische Bauelemente
zur Realisierung der Steuerungslogik.

\paragraph{Typische Elemente}
\begin{itemize}
    \item Transistoren
    \item Logikgatter
    \item Mikrocontroller
    \item FPGA
\end{itemize}

\paragraph{Eigenschaften}
\begin{itemize}
    \item Hohe Schaltgeschwindigkeit
    \item Kein mechanischer Verschleiß
    \item Geringe Leistungsfähigkeit für Aktoren
    \item Hoher Entwicklungsaufwand
\end{itemize}

\paragraph{Einordnung}
Elektronische Steuerungen werden heute
vor allem in eingebetteten Systemen eingesetzt
und bilden die Hardwarebasis moderner SPS.

\subsection{Speicherprogrammierbare Steuerung (SPS)}

\begin{figure}[H]
    \centering
    \includegraphics[width=0.6\textwidth]{/Users/danielweindl/_source/Repositorys/STT-Lernskript/Bilder/Automatisierungspyramide.png}
\end{figure}
\subsubsection{Begriff}

Eine speicherprogrammierbare Steuerung (SPS) ist eine
digital arbeitende Steuerungseinrichtung zur Steuerung von Maschinen
und Anlagen.
Die Steuerungsfunktion wird nicht durch Verdrahtung,
sondern durch ein im Speicher abgelegtes Programm realisiert.


Unter einer SPS versteht man einen speziellen Computer mit mehreren (in der Regel) digitalen Ein‐/Ausgängen,
der logische Verknüpfungen (programmierbar) zwischen den Ein‐ und Ausgängen herstellt.
Im Prinzip können Steuerungsaufgaben auch von ganz normalen PCs erfolgen. Entsprechende Software (Soft‐SPS für
PCs) ist für diesen Zweck auch verfügbar. Vom Standpunkt der Programmierung aus ist es unerheblich, auf welchem
System die SPS‐Funktion umgesetzt wird. Dennoch werden Steuerungen überwiegend auf speziellen SPS‐Computern
verwirklicht. Folgende Gründe sprechen für den Einsatz spezieller SPS‐Geräte:
\subsubsection{Grundelemente einer SPS}

Eine SPS besteht aus folgenden grundlegenden Elementen:

\begin{itemize}
    \item \textbf{Zentraleinheit (CPU)}  
    Führt das Steuerungsprogramm aus, verwaltet Speicher,
    steuert den Zyklus und koordiniert die Kommunikation.

    \item \textbf{Netzteil}  
    Versorgt CPU und Baugruppen mit der erforderlichen Betriebsspannung
    (typisch 24\,V DC).

    \item \textbf{Eingangskarte}  
    Erfassen binäre oder analoge Signale aus der Feldebene
    (Sensoren).

    \item \textbf{Ausgangskarte}  
    Stellen binäre oder analoge Signale zur Ansteuerung von Aktoren bereit.

    \item \textbf{Kommunikationsbaugruppen}  
    Ermöglichen den Datenaustausch mit anderen SPSen,
    Feldgeräten oder Leitsystemen.

    \item \textbf{Rückwandbus / internes Bussystem}  
    Verbindet CPU und Baugruppen und ermöglicht den internen Datenaustausch.
\end{itemize}

\subsubsection{Baugruppen einer SPS}

\paragraph{Digitale Eingangskarte}
\begin{itemize}
    \item Erfassen binäre Zustände (0/1)
    \item Typische Signale: Schalter, Taster, Endschalter
    \item Galvanische Trennung zur Feldebene
\end{itemize}

\paragraph{Digitale Ausgangskarte}
\begin{itemize}
    \item Schalten binäre Lasten
    \item Typisch: Relais-, Transistor- oder Triac-Ausgänge
\end{itemize}

\paragraph{Analoge Eingangskarte}
\begin{itemize}
    \item Erfassen kontinuierliche Messwerte
    \item Typische Signale: 0–10\,V, 4–20\,mA
\end{itemize}

\paragraph{Analoge Ausgangskarte}
\begin{itemize}
    \item Geben kontinuierliche Stellgrößen aus
    \item Ansteuerung von Frequenzumrichtern, Ventilen, Reglern
\end{itemize}

\subsubsection{Aufbauarten von SPS}

Speicherprogrammierbare Steuerungen lassen sich
nach ihrem konstruktiven Aufbau
in verschiedene Aufbauarten einteilen.
Diese unterscheiden sich insbesondere hinsichtlich
Leistungsfähigkeit, Erweiterbarkeit und Einsatzgebiet.

\paragraph{Nano-SPS}

Nano-SPS sind sehr kompakte Steuerungen
mit fest integrierten Ein- und Ausgängen.
Sie besitzen nur eine geringe Rechenleistung
und eingeschränkte Programmiermöglichkeiten.
Typische Einsatzgebiete sind einfache Steuerungsaufgaben,
Gebäudetechnik sowie Ausbildungs- und Übungszwecke.

\paragraph{Kompakt-SPS}

Bei Kompakt-SPS sind CPU, Netzteil
und Ein-/Ausgabebaugruppen
in einem gemeinsamen Gehäuse integriert.
Eine Erweiterung ist nur eingeschränkt möglich.
Kompakt-SPS werden häufig in kleinen bis mittleren Maschinen eingesetzt.

\paragraph{Mini-SPS}

Mini-SPS stellen eine leistungsstärkere Form
kompakter Steuerungen dar.
Sie verfügen über größere Programmspeicher
und erweiterte Kommunikationsmöglichkeiten.
Der Einsatz erfolgt typischerweise
in mittleren Automatisierungsanlagen.

\paragraph{Rackbasierte Klein-SPS}

Rackbasierte Klein-SPS sind modular aufgebaute Systeme,
bei denen CPU und Baugruppen
in einem gemeinsamen Rack angeordnet sind.
Die Kommunikation erfolgt über einen Rückwandbus.
Diese Aufbauart bietet eine gute Flexibilität
bei gleichzeitig kompakter Bauweise.

\paragraph{Rackbasierte SPS}

Rackbasierte SPS stellen die leistungsfähigste
klassische Aufbauart dar.
Sie bestehen aus einer CPU
und frei kombinierbaren Baugruppen
für Ein-/Ausgänge, Kommunikation und Spezialfunktionen.
Sie werden in komplexen Maschinen
und großen Industrieanlagen eingesetzt.

\paragraph{Soft-SPS}

Eine Soft-SPS ist eine softwarebasierte Steuerung,
die auf einem Industrie-PC oder Standard-PC läuft.
Die Anbindung der Peripherie erfolgt
über Feldbusse oder dezentrale I/O-Systeme.
Soft-SPS ermöglichen die Kombination
von Steuerung, Visualisierung und Datenverarbeitung
auf einer Plattform.

\medskip
\textbf{Merksatz:}  
\emph{Die Aufbauart bestimmt Flexibilität,
Leistungsfähigkeit und Einsatzbereich einer SPS.}
\begin{figure}[H]
    \centering
    \includegraphics[page=84,width=0.8\textwidth]{/Users/danielweindl/_source/Repositorys/STT-Lernskript/Data/sttvo-k01-Einführung-4v3-Folien_011025.pdf}    
\end{figure}

\subsubsection{Eigenschaften von SPS}

\begin{itemize}
    \item Zyklische, deterministische Programmausführung
    \item Hohe Zuverlässigkeit im industriellen Umfeld
    \item Gute Diagnose- und Fehlermeldemöglichkeiten
    \item Softwarebasierte Anpassbarkeit
    \item Lange Lebensdauer
\end{itemize}

\paragraph{Determinismus}
Eine deterministische Programmausführung liegt vor,
wenn ein Steuerungsprogramm bei gleichen Eingangszuständen,
gleichem Startzustand und identischen Randbedingungen
bei jedem Durchlauf denselben Ablauf und dieselben Ergebnisse liefert.
In SPS-Systemen wird dies durch den festen Zyklus
(Eingänge einlesen, Programm abarbeiten, Ausgänge schreiben)
und eine definierte Abarbeitungsreihenfolge erreicht.
Determinismus gewährleistet,
dass Reaktionszeiten vorhersehbar sind
und das Systemverhalten reproduzierbar bleibt,
was für sichere und zuverlässige Automatisierung
unverzichtbar ist.

\subsubsection{Abgrenzung zu klassischen Steuerungen}

Im Vergleich zu fest verdrahteten elektrischen Steuerungen
bietet die SPS:
\begin{itemize}
    \item geringeren Änderungsaufwand
    \item bessere Wartbarkeit
    \item höhere Funktionalität
\end{itemize}

\medskip
\textbf{Merksatz:}  
\emph{Die SPS ersetzt die Verdrahtungslogik durch Softwarelogik.}


\subsection{Zentrale und dezentrale Steuerungen}

Steuerungssysteme lassen sich nach der räumlichen Anordnung von
Steuereinheit und Ein-/Ausgabebaugruppen in zentrale und dezentrale
Steuerungen einteilen.

\subsubsection{Zentrale Steuerungen}

Bei zentralen Steuerungen befinden sich
CPU, Netzteil und Ein-/Ausgabebaugruppen
in einem gemeinsamen Schaltschrank.

\paragraph{Merkmale}
\begin{itemize}
    \item Zentrale Verarbeitung und zentrale Peripherie
    \item Direkte Verdrahtung aller Sensoren und Aktoren zur SPS
    \item Übersichtliche Systemstruktur
\end{itemize}

\paragraph{Vorteile}
\begin{itemize}
    \item Einfache Projektierung
    \item Geringe Systemkomplexität
    \item Gute Übersicht im Schaltschrank
\end{itemize}

\paragraph{Nachteile}
\begin{itemize}
    \item Hoher Verdrahtungsaufwand bei großen Anlagen
    \item Lange Leitungswege zur Feldebene
\end{itemize}

\paragraph{Typische Anwendung}
\begin{itemize}
    \item Kleine bis mittlere Maschinen
    \item Kompakte Anlagen
\end{itemize}

\begin{figure}[H]
    \centering
    \includegraphics[page=86,width=0.8\textwidth]{/Users/danielweindl/_source/Repositorys/STT-Lernskript/Data/sttvo-k01-Einführung-4v3-Folien_011025.pdf}    
\end{figure}


\subsubsection{Dezentrale Steuerungen}

Bei dezentralen Steuerungen werden
Ein- und Ausgabebaugruppen räumlich
in die Nähe des Prozesses verlagert.
Die Kommunikation mit der zentralen CPU
erfolgt über Feldbussysteme.

\paragraph{Merkmale}
\begin{itemize}
    \item Verteilte Ein-/Ausgabemodule
    \item Kommunikation über Feldbus (z.\,B. Profinet, Profibus, CAN)
    \item CPU kann zentral oder verteilt angeordnet sein
\end{itemize}

\paragraph{Vorteile}
\begin{itemize}
    \item Reduzierter Verdrahtungsaufwand
    \item Kürzere Sensor- und Aktorleitungen
    \item Hohe Skalierbarkeit
\end{itemize}

\paragraph{Nachteile}
\begin{itemize}
    \item Höherer Planungsaufwand
    \item Abhängigkeit von der Feldbuskommunikation
\end{itemize}

\paragraph{Typische Anwendung}
\begin{itemize}
    \item Große und räumlich verteilte Anlagen
    \item Modulare Maschinenkonzepte
\end{itemize}

\begin{figure}[H]
    \centering
    \includegraphics[page=87,width=0.8\textwidth]{/Users/danielweindl/_source/Repositorys/STT-Lernskript/Data/sttvo-k01-Einführung-4v3-Folien_011025.pdf}    
\end{figure}


\subsubsection{Vergleich zentrale vs. dezentrale Steuerung}

\begin{center}
\begin{tabular}{|l|c|c|}
\hline
\textbf{Merkmal} & \textbf{Zentral} & \textbf{Dezentral} \\ \hline
Verdrahtungsaufwand & hoch & gering \\ \hline
Systemkomplexität & gering & höher \\ \hline
Erweiterbarkeit & begrenzt & sehr gut \\ \hline
Typische Größe & klein/mittel & mittel/groß \\ \hline
\end{tabular}
\end{center}

\medskip
\textbf{Merksatz:}  
\emph{Zentrale Steuerungen sind übersichtlich,
dezentrale Steuerungen sind flexibel und skalierbar.}

\begin{figure}[H]
    \centering
    \includegraphics[page=88,width=0.8\textwidth]{/Users/danielweindl/_source/Repositorys/STT-Lernskript/Data/sttvo-k01-Einführung-4v3-Folien_011025.pdf}    
\end{figure}



\subsection{Zuordnungstabelle}
\begin{itemize}
    \item \textbf{Signalbezeichnung}  
    (eindeutiger, sprechender Name des Signals)

    \item \textbf{Signalart}  
    (digitaler Eingang, digitaler Ausgang, analoger Eingang, analoger Ausgang)

    \item \textbf{Physikalische Adresse}  
    (z.\,B. \texttt{E0.0}, \texttt{A2.3}, Analogkanal)

    \item \textbf{SPS-Baugruppe / Steckplatz / Kanal}  
    (Zuordnung zur konkreten Hardware)

    \item \textbf{Feldgerät}  
    (Sensor oder Aktor, z.\,B. Endschalter, Ventil, Motor)

    \item \textbf{Signalbeschreibung / Funktion}  
    (was bewirkt das Signal im Prozess)
\end{itemize}
\begin{figure}[H]
    \centering
    \includegraphics[width=0.9\textwidth]{/Users/danielweindl/_source/Repositorys/STT-Lernskript/Bilder/Zuordnungstabelle.png}
\end{figure}

\subsection{Feldebene}

Die Feldebene bildet die unterste Ebene der Automatisierungspyramide
und umfasst alle technischen Einrichtungen,
die \textbf{direkt mit dem Prozess gekoppelt} sind.
Dazu zählen Sensoren zur Erfassung von Prozessgrößen
sowie Aktoren zur Beeinflussung des Prozesses.
Die Kommunikation auf der Feldebene erfolgt über
binäre oder analoge Signale,
die von Ein-/Ausgabebaugruppen oder Feldbussystemen
(z.\,B. Profibus, Profinet, CAN)
zur übergeordneten Steuerung übertragen werden.
Die Feldebene stellt damit die Schnittstelle
zwischen realem Prozess und Steuerungssystem dar.

\begin{itemize}
    \item \textbf{Sensoren}  
    (z.\,B. Endschalter, Näherungsschalter, Lichtschranken, Druck-, Temperatur- und Wegsensoren)
    
    \item \textbf{Aktoren}  
    (z.\,B. Elektromotoren, Ventile, Zylinder, Relais, Schütze)
    
    \item \textbf{Ein- und Ausgabegeräte}  
    (digitale und analoge E/A-Baugruppen, dezentrale I/O-Module)
    
    \item \textbf{Feldgeräte}  
    (Messumformer, Stellgeräte, intelligente Sensoren)
    
    \item \textbf{Feldbussysteme und Signalübertragung}  
    (z.\,B. Profibus, Profinet, CAN, IO-Link)
\end{itemize}