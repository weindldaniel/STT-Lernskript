\newpage
\subsection{AS}

\subsubsection{Grundprinzip}

Die Ablaufsprache (AS, engl. Sequential Function Chart – SFC)
ist eine grafische Programmiersprache nach IEC~61131-3
zur Beschreibung \textbf{sequentieller Abläufe}.
Sie eignet sich besonders zur Darstellung von Ablaufsteuerungen
mit klar definierten Zuständen und Übergängen.

SFC basiert auf dem Zustandsmodell und trennt explizit zwischen:
\begin{itemize}
    \item Schritten (Zustände)
    \item Transitionen (Übergangsbedingungen)
    \item Aktionen (auszuführende Operationen)
\end{itemize}
\begin{figure}[H]
    \centering
    \includegraphics[page=145,width=0.8\textwidth]{/Users/danielweindl/_source/Repositorys/STT-Lernskript/Data/sttvo-k03-Programmiersprachen-4v5-Folien.pdf}    
 \end{figure}
\begin{figure}[H]
    \centering
    \includegraphics[page=146,width=0.8\textwidth]{/Users/danielweindl/_source/Repositorys/STT-Lernskript/Data/sttvo-k03-Programmiersprachen-4v5-Folien.pdf}    
 \end{figure}

\subsubsection{Schritte}

Ein Schritt repräsentiert einen Zustand des Steuerungsablaufs.
Zu jedem Zeitpunkt ist mindestens ein Schritt aktiv.

\begin{itemize}
    \item Schritte werden als Rechtecke dargestellt
    \item Aktive Schritte sind logisch TRUE
    \item Ein Initialschritt ist beim Start aktiv
\end{itemize}

Ein Schritt selbst führt keine Logik aus,
sondern aktiviert zugeordnete Aktionen.

\subsubsection{Transitionen}

Transitionen beschreiben die Bedingung für den Übergang
von einem oder mehreren Schritten zu einem oder mehreren Folgeschritten.

\begin{itemize}
    \item Transitionen werden logisch ausgewertet
    \item Sie besitzen ausschließlich BOOL-Ergebnisse
    \item Ein Übergang erfolgt nur, wenn alle vorgeschalteten Schritte aktiv sind
\end{itemize}

\medskip
\textbf{Regel:}  
Ein Schritt wird verlassen, wenn die zugehörige Transition TRUE wird.

\begin{figure}[H]
    \centering
    \includegraphics[page=151,width=0.8\textwidth]{/Users/danielweindl/_source/Repositorys/STT-Lernskript/Data/sttvo-k03-Programmiersprachen-4v5-Folien.pdf}    
 \end{figure}


\subsubsection{Aktivierungs- und Deaktivierungsregeln}

\begin{itemize}
    \item Wird eine Transition TRUE, werden die vorherigen Schritte deaktiviert
    \item Die nachfolgenden Schritte werden gleichzeitig aktiviert
    \item Mehrere Schritte können parallel aktiv sein
\end{itemize}

SFC erlaubt damit auch parallele Abläufe.

\subsubsection{Aktionen}

Aktionen sind die eigentlichen Programmanweisungen
und werden einem Schritt zugeordnet.
Sie können in KOP, FBS, ST oder AWL implementiert sein.

Eine Aktion wird ausgeführt, solange ihr Schritt aktiv ist
und die Aktionsbedingung erfüllt ist.

\begin{figure}[H]
    \centering
    \includegraphics[page=154,width=0.8\textwidth]{/Users/danielweindl/_source/Repositorys/STT-Lernskript/Data/sttvo-k03-Programmiersprachen-4v5-Folien.pdf}    
 \end{figure}


\subsubsection{Aktionsblöcke}

Aktionen werden in Aktionsblöcken dargestellt.
Ein Aktionsblock besteht aus:
\begin{itemize}
    \item einem Bestimmungszeichen (Qualifier)
    \item optional einer Zeitangabe
    \item dem Aktionsnamen oder einer BOOL-Variablen
\end{itemize}

\subsubsection{Aktionsqualifier}

Der Qualifier bestimmt das zeitliche Verhalten einer Aktion.

\begin{itemize}
    \item \textbf{N} (Normal):  
    Aktion ist aktiv, solange der Schritt aktiv ist
    
    \item \textbf{S} (Set):  
    Aktion wird beim Aktivieren des Schrittes gesetzt
    und bleibt aktiv, bis sie explizit zurückgesetzt wird
    
    \item \textbf{R} (Reset):  
    Setzt eine zuvor gesetzte Aktion zurück
    
    \item \textbf{L} (Limit):  
    Aktion ist nur für eine begrenzte Zeit aktiv
    
    \item \textbf{D} (Delay):  
    Aktion wird zeitverzögert aktiviert
\end{itemize}

Zeitqualifier verwenden entweder konstante Zeiten
oder Variablen vom Typ \texttt{TIME}.

\begin{figure}[H]
    \centering
    \includegraphics[page=155,width=0.8\textwidth]{/Users/danielweindl/_source/Repositorys/STT-Lernskript/Data/sttvo-k03-Programmiersprachen-4v5-Folien.pdf}    
 \end{figure}
\begin{figure}[H]
    \centering
    \includegraphics[page=156,width=0.8\textwidth]{/Users/danielweindl/_source/Repositorys/STT-Lernskript/Data/sttvo-k03-Programmiersprachen-4v5-Folien.pdf}    
 \end{figure}
\begin{figure}[H]
    \centering
    \includegraphics[page=158,width=0.8\textwidth]{/Users/danielweindl/_source/Repositorys/STT-Lernskript/Data/sttvo-k03-Programmiersprachen-4v5-Folien.pdf}    
 \end{figure}


\subsubsection{Parallele und alternative Abläufe}

SFC unterstützt:
\begin{itemize}
    \item alternative Verzweigungen (ODER-Verzweigung)
    \item parallele Verzweigungen (UND-Verzweigung)
\end{itemize}

Bei parallelen Abläufen müssen alle parallelen Zweige
abgeschlossen sein, bevor der Ablauf fortgesetzt wird.

\subsubsection{Einbindung in das SPS-Zyklusmodell}

\begin{itemize}
    \item Schritte und Transitionen werden zyklisch ausgewertet
    \item Aktionen werden innerhalb des SPS-Zyklus ausgeführt
    \item Zeitfunktionen basieren auf der Zykluszeit
\end{itemize}

SFC selbst beschreibt nur die Ablaufstruktur,
nicht die konkrete Ausführungslogik der Aktionen.



\subsubsection{Verzweigungen}
\begin{figure}[H]
    \centering
    \includegraphics[page=162,width=0.8\textwidth]{/Users/danielweindl/_source/Repositorys/STT-Lernskript/Data/sttvo-k03-Programmiersprachen-4v5-Folien.pdf}    
 \end{figure}
\begin{figure}[H]
    \centering
    \includegraphics[page=163,width=0.8\textwidth]{/Users/danielweindl/_source/Repositorys/STT-Lernskript/Data/sttvo-k03-Programmiersprachen-4v5-Folien.pdf}    
 \end{figure}
\begin{figure}[H]
    \centering
    \includegraphics[page=164,width=0.8\textwidth]{/Users/danielweindl/_source/Repositorys/STT-Lernskript/Data/sttvo-k03-Programmiersprachen-4v5-Folien.pdf}    
 \end{figure}


\subsubsection{Beispiele}
\begin{figure}[H]
    \centering
    \includegraphics[page=173,width=0.8\textwidth]{/Users/danielweindl/_source/Repositorys/STT-Lernskript/Data/sttvo-k03-Programmiersprachen-4v5-Folien.pdf}    
 \end{figure}
\begin{figure}[H]
    \centering
    \includegraphics[page=177,width=0.8\textwidth]{/Users/danielweindl/_source/Repositorys/STT-Lernskript/Data/sttvo-k03-Programmiersprachen-4v5-Folien.pdf}    
 \end{figure}
\begin{figure}[H]
    \centering
    \includegraphics[page=178,width=0.8\textwidth]{/Users/danielweindl/_source/Repositorys/STT-Lernskript/Data/sttvo-k03-Programmiersprachen-4v5-Folien.pdf}    
 \end{figure}

\subsubsection{Vor- und Nachteile von SFC}

\paragraph{Vorteile}
\begin{itemize}
    \item Sehr übersichtliche Darstellung komplexer Abläufe
    \item Klare Trennung von Zustand und Aktion
    \item Sehr gut geeignet für Ablaufsteuerungen
\end{itemize}

\paragraph{Nachteile}
\begin{itemize}
    \item Für reine Logik ungeeignet
    \item Zusätzlicher Implementierungsaufwand für Aktionen
\end{itemize}

\subsubsection{Einordnung}

SFC eignet sich besonders für:
\begin{itemize}
    \item Schrittketten
    \item Ablauf- und Prozesssteuerungen
    \item Strukturierung komplexer Programme
\end{itemize}

In der Praxis wird SFC häufig mit KOP, FBS oder ST kombiniert,
wobei SFC die Ablaufstruktur vorgibt
und die Aktionen in anderen Sprachen implementiert werden.