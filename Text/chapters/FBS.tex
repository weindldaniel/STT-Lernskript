\newpage
\subsection{FBS}

\subsubsection{Grundprinzip}

Die Funktionsbausteinsprache (FBS, engl. Function Block Diagram – FBD)
ist eine grafische Programmiersprache nach IEC~61131-3.
Sie basiert auf der Darstellung von Funktionen und Funktionsbausteinen
als grafische Blöcke, die über Signalverbindungen miteinander verknüpft sind.

Im Mittelpunkt steht der \textbf{Signalfluss} zwischen Ein- und Ausgängen.

\subsubsection{Signalflussorientierung}

In FBS wird die Programmlogik durch die Verbindung von Ausgängen
zu Eingängen anderer Bausteine beschrieben.

\begin{itemize}
    \item Daten fließen von links nach rechts
    \item Ein Ausgang kann mehrere Eingänge speisen
    \item Der Signalfluss bestimmt die logische Abhängigkeit
\end{itemize}

Die zeitliche Ausführung erfolgt dennoch innerhalb des SPS-Zyklus
und wird durch die zugeordnete Task bestimmt.

\subsubsection{Bausteintypen in FBS}

In FBS können folgende Bausteintypen verwendet werden:

\begin{itemize}
    \item Funktionen (stateless)
    \item Funktionsbausteine (mit Gedächtnis)
    \item Erweiterbare Funktionen
\end{itemize}

\paragraph{Funktionen}
Funktionen besitzen kein internes Gedächtnis.
Bei gleichen Eingängen liefern sie immer gleiche Ausgänge.

\paragraph{Funktionsbausteine}
Funktionsbausteine besitzen ein internes Gedächtnis.
Sie benötigen eine Instanzvariable und sind für
zustandsbehaftete Aufgaben geeignet (z.\,B. Timer, Zähler).

\begin{figure}[H]
    \centering
    \includegraphics[page=95,width=0.8\textwidth]{/Users/danielweindl/_source/Repositorys/STT-Lernskript/Data/sttvo-k03-Programmiersprachen-4v5-Folien.pdf}    
 \end{figure}


\subsubsection{Bausteininstanzen}

Jeder Funktionsbaustein wird über eine Instanz aufgerufen.
Die Instanz speichert die internen Zustände des Bausteins
über mehrere SPS-Zyklen hinweg.

\begin{itemize}
    \item Eine Instanz pro Funktionsbaustein-Aufruf
    \item Mehrere Instanzen desselben Bausteintyps möglich
\end{itemize}

Ohne Instanz ist keine Speicherwirkung möglich.

\subsubsection{Abarbeitungsreihenfolge}
\begin{figure}[H]
    \centering
    \includegraphics[page=97,width=0.8\textwidth]{/Users/danielweindl/_source/Repositorys/STT-Lernskript/Data/sttvo-k03-Programmiersprachen-4v5-Folien.pdf}    
 \end{figure}



\subsubsection{EN/ENO-Logik}

Viele Funktionsbausteine und Funktionen besitzen optionale
Enable-Eingänge (EN) und Enable-Ausgänge (ENO).

\begin{itemize}
    \item EN = FALSE: Baustein wird nicht ausgeführt
    \item EN = TRUE: Baustein wird ausgeführt
    \item ENO zeigt an, ob der Baustein aktiv war
\end{itemize}

Ist EN FALSE, bleiben die Ausgänge des Bausteins unverändert.
Dies ist insbesondere bei der Weiterverschaltung zu beachten.

\subsubsection{Weiterverschaltung von Bausteinen}

\begin{itemize}
    \item Bausteine ohne EN/ENO können direkt weiter verschaltet werden
    \item Bausteine mit aktiviertem EN/ENO dürfen
    ausgangsseitig nicht direkt weiter verschaltet werden
    \item Zwischenspeicherung über Variablen ist erforderlich
\end{itemize}

Diese Einschränkung dient der Vermeidung undefinierter Zustände,
wenn Bausteine deaktiviert sind.

\subsubsection{Analoge Signalverarbeitung}

FBS eignet sich besonders gut zur Verarbeitung analoger Werte.

\begin{itemize}
    \item Analoge Werte werden über Linien dargestellt
    \item Keine analogen Kontakte wie in KOP
    \item Verbindung nur über Funktions- oder FB-Ein-/Ausgänge
\end{itemize}

Typische Anwendungen:
\begin{itemize}
    \item Skalierung
    \item Vergleich
    \item Berechnung
    \item Filterung
\end{itemize}

\subsubsection{Erweiterbare Funktionen}

Die IEC~61131-3 definiert erweiterbare Funktionen,
die mehr als zwei Eingänge besitzen können.

Beispiele:
\begin{itemize}
    \item ADD
    \item MUL
    \item AND
    \item OR
    \item MAX / MIN
\end{itemize}

Nicht beschaltete Eingänge sind nicht zulässig.

\subsubsection{Programmflusssteuerung}
\begin{figure}[H]
    \centering
    \includegraphics[page=102,width=0.8\textwidth]{/Users/danielweindl/_source/Repositorys/STT-Lernskript/Data/sttvo-k03-Programmiersprachen-4v5-Folien.pdf}    
 \end{figure}
\begin{figure}[H]
    \centering
    \includegraphics[page=103,width=0.8\textwidth]{/Users/danielweindl/_source/Repositorys/STT-Lernskript/Data/sttvo-k03-Programmiersprachen-4v5-Folien.pdf}    
 \end{figure}

\subsubsection{Vergleich FBS zu KOP und ST}

\begin{center}
\begin{tabular}{|l|c|c|c|}
\hline
\textbf{Merkmal} & \textbf{FBS} & \textbf{KOP} & \textbf{ST} \\ \hline
Darstellung & grafisch & grafisch & textuell \\ \hline
Binäre Logik & gut & sehr gut & gut \\ \hline
Analoge Werte & sehr gut & eingeschränkt & sehr gut \\ \hline
Übersichtlichkeit & hoch & hoch & mittel \\ \hline
Komplexe Algorithmen & eingeschränkt & schlecht & sehr gut \\ \hline
\end{tabular}
\end{center}

\subsubsection{Vor- und Nachteile von FBS}

\paragraph{Vorteile}
\begin{itemize}
    \item Sehr anschauliche Darstellung
    \item Ideal für Signalverarbeitung
    \item Gute Diagnosemöglichkeiten
\end{itemize}

\paragraph{Nachteile}
\begin{itemize}
    \item Bei großen Programmen schnell unübersichtlich
    \item Schleifen und komplexe Abläufe schwer darstellbar
\end{itemize}

\subsubsection{Einordnung}

FBS eignet sich besonders für:
\begin{itemize}
    \item Signalflussorientierte Aufgaben
    \item Analogwertverarbeitung
    \item Kombination mit KOP und ST
\end{itemize}
